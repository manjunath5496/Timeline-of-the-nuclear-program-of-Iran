\documentclass{report}
\usepackage{fullpage}
\begin{document}
\title{3.185 Lectures}
\author{Adam Powell}
\date{Fall Semester, 2003}
\maketitle

\begin{abstract}
  This represents the sum total of lecture material presented in 3.185, {\em
    Transport Phenomena in Materials Engineering}, in the Fall of 2003.  It is
  not meant to be a ``reader'' for the course, but more of an ``electronic
  notebook'' of my own, a set of bullet points if you will, to guide the
  content of each lecture.  As such, it reads very poorly, but does contain all
  of the equations and derivations presented in the course.

  This document is copyright 2003 by Adam Powell, and as part of MIT Open
  CourseWare (OCW), is licensed in the same manner as other OCW documents (see
  {\tt http://ocw.mit.edu/global/terms-of-use.html} for licensing terms).
\end{abstract}

\tableofcontents

\chapter{Introduction}

\section{September 3, 2003}

Handouts: syllabus, ABET, Diffusion, PS1 due Monday 9/8.

Circulate signup sheet: name, username, year, course

\begin{itemize}
\item Introductions: me, Albert.

\item What is covered: review stuff on general overview.  Lots of complexity:
  from single ODE to five fully-coupled nonlinear second-order PDEs in five
  field variables.

\item Intro to the general transport methodology: conservation and
  constitutive equations
  \begin{equation}
    \label{eq:conservation}
    \rm accumulation = in - out + generation
  \end{equation}
  Discuss the ``terms'', with a flying eraser.

  Microscopic and macroscopic.

\item Necessary for all classes of materials, ask how many are interested in
  each:
  \begin{itemize}
  \item Polymers: synthesis, injection molding, membranes (Bo)
  \item Bio: drug delivery, anisotropic diffusion, blood flow (cool)
  \item Ceramics: powder synthesis, separation, drying, sintering
  \item Electronic: crystal growth, CVD, diffusion
  \item Metals: smelting, refining, casting, heat treatment
  \end{itemize}

\item Why 3.185 is important: processing-(structure)-properties-performance.
  We do low-cost, high-quality processing, low environment overhead, which is
  one of the two important aspects of this triad/tetrahedron.

  Sponsors of our work care about two things: low-cost high-quality processes
  and high performance.  They don't care about structure.  Andy Groves,
  chairman of Intel, could care less about the electronic structure of titanium
  silicide-titanium aluminide diffusion barriers in aluminum interconnects, he
  wants cheap high-quality processes that result in high performance.  Closer
  to home, parents' eyes glaze over at talk of ``Kinetics of eta phase
  precipitation in nickel superalloys,'' but not at ``Avoiding catastrophic
  failure of jet engine turbine blades in service.''  Structure provides an
  important way to model the relationship between processing and properties,
  without which a black box, not a science.
\end{itemize}

\paragraph{Mechanics}

\begin{itemize}
\item Discuss grading: HW points and collab, double-session tests, mixed final.
\item Get test conflict dates, aim for Weds. 10/15--17, 11/19--21.
\item Make sure everyone has a recitation.
\item Schedule office hours.
\item Discuss travel: three trips, none of which should impact 3.185.  If one
  more, tradition of having previous TA give a lecture.
\end{itemize}

\paragraph{Required math}
\begin{itemize}
\item Vector arithmetic (dot product, cross product, outer product)
\item Vector calculus (gradient, divergence, curl)
\item Solving homogeneous linear ordinary differential equations, e.g.
  \begin{equation}
    \label{eq:ODEs}
    y'' = k,\ \ {\rm or}\ \ y'' - ky = 0
  \end{equation}
\item What partial differential equations look like, e.g.
  \begin{equation}
    \label{eq:pdes}
    \nabla^2 C = 0
  \end{equation}
\item The error function and derivatives:
  \begin{eqnarray}
    \label{eq:erfdef}
    {\rm erf}(x) & = & \frac{2}{\sqrt{\pi}}\int_0^x e^{-\xi^2}d\xi \\
    \frac{d}{dx}{\rm erf}(x) & = &
    \frac{d}{dx}\frac{2}{\sqrt{\pi}}\int_0^x e^{-\xi^2}d\xi =
    \frac{2}{\sqrt{\pi}}e^{-x^2}
  \end{eqnarray}
\item The substantial derivative: the time derivative in a moving frame.
  \begin{equation}
    \label{eq:substantialdef}
    \frac{D}{Dt} = \frac{\partial}{\partial t} + \vec{u}\cdot\nabla
  \end{equation}
  Kind of like moving vector $x(t),y(t),z(t)$:
  \begin{equation}
    \label{eq:substantialexplan}
    \left.\frac{dC}{dx}\right|_{(x,y,z)} = \frac{\partial C}{\partial t} +
    \frac{\partial C}{\partial x}\frac{\partial x}{\partial t} +
    \frac{\partial C}{\partial y}\frac{\partial y}{\partial t} +
    \frac{\partial C}{\partial z}\frac{\partial z}{\partial t}.
  \end{equation}
\end{itemize}

\paragraph{Previous feedback}

\begin{itemize}
\item Prof. Powell is cool, lectures are great, double tests are neat!
\item This course spent way too much time on diffusion.  Okay, will cut a bit
  shorter this time.  But non-3.01s will be lost, need to see TA or me.
\item Too much busy algebra on problem sets.  Okay, will cut quite a bit, some
  computer.
\item Problem sets should be due on Friday instead of Monday, for last-minute
  recitation help.  Poll class, incl. PS1 Mon or Weds?
\item Textbook is awful.  It covers things in the wrong order, and is hard to
  read.  Changing to new textbook, better readings, but still wrong order.
  Also see Incropera and DeWitt, in Reading Room; old text there too.
\item TLL: ``muddiest part'', index cards for each lecture from Friday, started
  last year, need more.
\item Too much online.  But taking it off would only hurt those without
  Bibles.  Sorry, won't do.
\item Prof. Powell lets ``dumb'' questions slow things down.  No dumb
  questions.  Very often correct mistakes or omissions, ten others have the
  same question.  If anything, MIT juniors and seniors need to be much more
  vocal!  (Last mid-term evaluation, dreadful lecture...)
\end{itemize}
\newpage


\chapter{Diffusion}

\section{September 5, 2003: 1-D Cartesian and Cylindrical Steady-State}
\label{diff1dss}

TODO:
\begin{itemize}
\item Check reading room to make sure texts are there.
\item Bring: cards, class list.
\item Check text to make sure chap 25 units are consistent with mine.
\end{itemize}

\noindent Opener: Colleen's facility with names... my advisee!

Mechanics:
\begin{itemize}
\item Diffusion handout typo: should be erfc($y$) = 1--erf($y$).
\item Choose new recitation times!
\item Finalize test dates.
\item Invite example processes.
\item Index cards for muddiest part.
\item Try names.
\end{itemize}

\paragraph{Diffusion}
Stuff they've learned before, new twist.  Steady state 1-D cartesian,
cylindrical coordinates.

Steady-state: accumulation=0.  In today's case, species isn't generated or
consumed inside the glass, so in--out=0.  (Monday: generation by homogeneous
chemical reaction.)

1-D: concentration varies only in one direction.

My style: start with a motivating example, introduce the physics along the
way.  When we're done, we have the physics, and an example of how to use it.

Yesterday: 3.185 is about low-cost high-quality processes.  Here a process, not
for material but for helium, maximize productivity a.k.a. throughput subject to
process constraints.

Example: helium diffusion through pyrex glass, enormously higher $D$ than any
other gas (25x hydrogen!).  Some helium in natural gas, can flow through pyrex
tubes to separate, 2mm OD 1mm ID.  Generally diffusion-limited.  Want to
calculate the production rate, more important understand how works, because
from understanding flows design recommendations.  ASSUME diffusion-limited, so
this is the slow-step, not adsorption/desorption etc.

Simple solution: unroll to a plate, with $C_{in}$ on one side (equilibrium with
partial pressure in natural gas) and $C_{out}$ on the other (pumped away into
tanks), be sure to use thickness $\delta$.  We know how to do this:
\begin{quotation}
  \noindent Constitutive law: Fick's first, gradient points up, diffusion goes
  down, proportionality constant $D$:
  \begin{equation}
    \label{eq:ficksfirst}
    \vec{J} = -D\nabla C, J_x = -D \frac{\partial C}{\partial x}
  \end{equation}
  Units of each term.

  \noindent 1-D: no difference in $y$- or $z$-direction, so those partials are
  zero.  When varies in only one direction and not time, not partial but total
  $dC/dx$.

  \noindent Conservation with no accumulation or generation: $dJ_x/dx = 0$,
  substitute to get
  \begin{equation}
    \label{eq:ficks1.5}
    \frac{d}{dx}\left(D\frac{dC}{dx}\right) = 0,
  \end{equation}
  ASSUME constant $D$ this is
  \begin{equation}
    \label{eq:ficksecond}
    \frac{d^2 C}{dx^2} = 0.
  \end{equation}
  General solution in 1-D:
  \begin{equation}
    \label{eq:linear}
    C = Ax + B.
  \end{equation}
  Boundary conditions (limited by diffusion):
  \begin{equation}
    \label{eq:linbcs}
    x=0 \Rightarrow C=C_{in}, x=\delta \Rightarrow C=C_{out}.
  \end{equation}
  Result: flux
  \begin{equation}
    \label{eq:linflux}
    J=-D dC/dx=-D \frac{\Delta C}{\delta}.
  \end{equation}

  At 500$^\circ$C, $D_{\rm He-pyrex}=2\times 10^{-8}\ \frac{\rm cm^2}{\rm s}$,
  for some steady gas/helium mixture $C_{in}=10^{-5}\frac{\rm g}{\rm cm^3}$,
  say $C_{out}\simeq 0$.  For $\delta={\rm 0.5mm=0.05cm}$, this gives
  \begin{equation}
    \label{eq:realflux}
    J = {\rm 2\times 10^{-8}\ \frac{cm^2}{s} \cdot
    \frac{10^{-5}\frac{g}{cm^3}}{0.05cm} =
    4\times10^{-12}\frac{g}{cm^2\cdot s}}.
  \end{equation}
  Tube array with total length 10m=1000cm (e.g. 100 tubes each 10 cm long),
  $R_2={\rm OD}/2=0.1$cm, so throughput is
  \begin{equation}
    \label{eq:realfluxarea}
    JA = 2\pi R_2 L J = 8\times10^{-10}\pi =
  2.5\times10^{-9}\frac{\rm g}{\rm s}
  \end{equation}
  Or do we use $R_1$?  That would give $1.2\times10^9$.  How far off is the
  flux?  A dilemma.

  \noindent Timescale: $\delta^2 \simeq Dt \Rightarrow$ steady state.  Here
  $t\simeq\delta^2/D=125000$ seconds, about a day and a half.
\end{quotation}

\noindent Design: what to do to improve throughput?
\begin{itemize}
\item Smaller $\delta$: possible breakage
\item Higher $D$: change glass, raise temperature
\item Higher $\Delta C$: raise/lower temperature, change glass
\end{itemize}
Okay, that was the braindead 1-D solution.  What about the real cylinder?

\paragraph{Cylindrical coordinates}

So, $C_{out}$ at outside, $C_{in}$ at inside, what to do between?  Use $R_1$
and $R_2$ for inner, outer radii.  Fick's first, assume 1-D, so $C$ is function
of $r$ only.
\begin{equation}
  \label{eq:cylfick1}
  J_r = -D\frac{dC}{dr}
\end{equation}
Conservation: in at $r+\Delta r$, out at $r$, no gen or accum, area $2\pi r L$:
\begin{equation}
  \label{eq:cylconserv}
  0 = [2\pi r L J_r]_r - [2\pi r L J_r]_{r+\Delta r},
\end{equation}
divide by $2\pi L$, $\Delta r\rightarrow 0$:
\begin{equation}
  \label{eq:cyleq}
  0 = -\frac{d}{dr}[r J_r]
\end{equation}
Plug in flux:
\begin{equation}
  \label{eq:cyloverall}
  0 = \frac{d}{dr}\left(rD\frac{dC}{dr}\right).
\end{equation}
Now solve:
\begin{eqnarray}
  \label{eq:cyleqint}
  A' & = & r D\frac{dC}{dr} \\
  \frac{A'}{Dr} & = & \frac{dC}{dr} \\
  C & = & A \ln r + B
\end{eqnarray}
where $A=A'/D$.  From BCs:
\begin{equation}
  \label{eq:cylintsol}
  \frac{C-C_{in}}{C_{out}-C_{in}} = \frac{\ln (r/R_1)}{\ln (R_2/R_1)}
\end{equation}
Check at $R_1$ and $R_2$, units.

\noindent Flux$=-D dC/dr$:
\begin{equation}
  \label{eq:cylflux}
  J_r = -D\frac{dC}{dr} = -D \frac{d}{dr}
  \left[C_in + (C_{out}-C_{in})\frac{\ln (r/R_1)}{\ln (R_2/R_1)}\right] =
  D\frac{C_{in}-C_{out}}{\ln (R_2/R_1)}\frac{1}{r}.
\end{equation}
Important result: not flux, but flux times area.
\begin{equation}
  \label{eq:cylfluxarea}
  AJ_r = -2\pi r L D\frac{dC}{dr} =
  2\pi r L D \frac{C_{in}-C_{out}}{\ln (R_2/R_1)}\frac{1}{r}.
\end{equation}
Note $r$s cancel, so $AJ_r$ is constant for all $r$.  Make sure units work.
Cool.

\noindent Numbers, result: 1.8$\times10^{-9}$.  Between the 1-D estimates.
Overestimated flux, at $R_2$ is really $1.44\times10^{-12}$, twice that at
$R_1$, so $R_1$ value is closer.

\noindent More important: cartesian gave {\em wrong design criterion}!  Not
minimize $\delta$, but minimize $R_2/R_1$!  Double production by going from 2
to $\sqrt{2}$ because $\ln(\sqrt{2}) = \frac{1}{2}\ln(2)$, e.g. 3 mm OD with no
change in thickness!

\noindent Other design issues: helium on inside or outside?  Inside means glass
is in tension, outside compression.  But if gas is dirty, inside is easier to
clean.

\noindent Note: on problem set 2, will derive this for a sphere for a drug
delivery device.  Pretty cool.
\newpage


\section{September 8, 2003: Steady-State with Homogeneous Chemical Reaction}

\noindent Mechanics:
\begin{itemize}
\item New recitations: R12, F2, both in 8-306.
\item Fri: very different lecture, ABET stuff.
\end{itemize}
Names.

\noindent Muddy stuff:
\begin{itemize}
\item Recitations. :-)
\item Dislike cgs units.
\item Clearer writing and neater presentation.  (Big chalk...)
\item Why both ways?  One is simple but wrong, other is complex and right.
\item Flux in cylindrical coordinates.  Give full gradient. (Next time.)
\item Integrating to get solution: $d/dr(rdC/dr)=0\Rightarrow C=A'\ln r+B$.

\item General$\longrightarrow$particular solution in cylindrical coords.  Start
  with general, plug in BCs:
  \begin{equation}
    \label{eq:cylstdgen}
    C = A\ln r + B
  \end{equation}
  \begin{eqnarray}
    \label{eq:cylstdconcderive}
    C_{in} &=& A'\ln R_1 + B \\
    C_{out} &=& A'\ln R_2 + B \\
    C_{out} - C_{in} &=& A'(\ln R_2 - \ln R_1) \\
    A' &=& \frac{C_{out} - C_{in}}{\ln (R_2/R_1)} \\
    C_{in} &=& \frac{C_{out} - C_{in}}{\ln (R_2/R_1)}\ln R_1 + B \\
    B &=& C_{in} - \frac{C_{out} - C_{in}}{\ln (R_2/R_1)}\ln R_1 \\
    C &=& \frac{C_{out} - C_{in}}{\ln (R_2/R_1)}\ln r + 
    C_{in} - \frac{C_{out} - C_{in}}{\ln (R_2/R_1)}\ln R_1 \\
    C-C_{in} &=& \frac{C_{out} - C_{in}}{\ln (R_2/R_1)} \ln (r/R_1) \\
    \frac{C-C_{in}}{C_{out}-C_{in}} &=& \frac{\ln (r/R_1)}{\ln (R_1/R_2)}
  \end{eqnarray}

\item Didn't finish: timescale to steady-state $\tau\sim L^2/D$, in this case
  125,000 seconds, about a day and a half.  Will explore this more rigorously
  on Friday.
\end{itemize}
Summarize: illustrates 3.185 methodology
\begin{itemize}
\item Problem statement: maximize throughput = flux$\times$area
\item Conservation equation
\item Constitutive equation
\item Combine to give (partial) differential equation
\item General solution with integration constants
\item Boundary conditions give values of integration constants
\item Use solution to get problem objective: flux$\times$area
\item Design recommendation follows from solution
\end{itemize}
Like p. 465 of W$^3$R.

\paragraph{Generation} Homogeneous chemical reactions.  RCC carbon
fiber-reinforced graphite composite!  Very high-temperature, high-strength.
Carbon fiber preform, model as a porous material, diffusion of acetylene to the
surfaces of the fibers, at high temp it decomposes and deposits graphite.

\noindent Problem: as it deposits, it seals off the entrances, non-constant
diffusivity.  Generation of acetylene $G = -kC$.  UNITS!

\noindent Set up problem in center, symmetry, sheet of material.  1-D equation:
\begin{equation}
  \label{eq:genbalance}
  0 = D\frac{d^2C}{dx2} + G = D\frac{d^2C}{dx^2} - kC
\end{equation}
General solution, using polynomials $e^{Rx}$, $R=\pm\sqrt{k/D}$, so
\begin{equation}
  \label{eq:gensolution}
  C = A\exp\left(x\sqrt{\frac{k}{D}}\right) +
  B\exp\left(-x\sqrt{\frac{k}{D}}\right)
\end{equation}
BCs: at $x=\pm\frac{L}{2}$, $C=C_0$, so $A=B$,
\begin{equation}
  \label{eq:genbcfit}
  C_0 = A \left[\exp\left(\frac{L}{2}\sqrt{\frac{k}{D}}\right) +
    \exp\left(-\frac{L}{2}\sqrt{\frac{k}{D}}\right)\right] =
  A \cosh\left(\frac{L}{2}\sqrt{\frac{k}{D}}\right)
\end{equation}
Result:
\begin{equation}
  \label{eq:gendone}
  \frac{C}{C_0} = \frac{\cosh\left(x\sqrt{\frac{k}{D}}\right)}
  {\cosh\left(\frac{L}{2}\sqrt{\frac{k}{D}}\right)}
\end{equation}

What does this look like?  Pay attention to $\frac{L}{2}\sqrt{\frac{k}{D}}$, or
more generally, $\frac{L^2 k}{D}$.  Can either be sorta uniform, or VERY
non-uniform, uniform if that number is small (thin sheet, slow reaction, fast
diffusion), nonuniform if it's large (thick sheet, fast reaction, slow
diffusion).  Makes sense.

So, have process, want to double thickness with same uniformity, can't change
$D$ much, how much change $k$?  Drop by factor of 4.  Problem: takes four times
as long!!

The real solution: blow acetylene through it!
\newpage


\section{September 10, 2003: Unsteady-State Diffusion}

TODO:
\begin{itemize}
\item Check reading room to make sure texts are there.
\item Bring: cards, class list.
\item Check W$^3$R pages for this lecture.
\end{itemize}
NAMES!

\noindent Mechanics:
\begin{itemize}
\item Get office hours together.
\item Pump the zephyr instance.
\item Fri: very different lecture, ABET stuff.
\end{itemize}

\noindent Muddy stuff:
\begin{itemize}
\item Cylindrical gradient (W$^3$R appendices A-B, p. 695--700):
  \begin{equation}
    \label{eq:cylgrad}
    \nabla C = \frac{\partial C}{\partial r}\hat{r}
    + \frac{1}{r}\frac{\partial C}{\partial\theta}\hat{\theta}
    + \frac{dC}{dz}\hat{z},
  \end{equation}
  \begin{eqnarray}
    \label{eq:cyllap}
    \nabla^2 C &=& \frac{\partial^2C}{\partial r^2}
    + \frac{1}{r}\frac{\partial C}{\partial r}
    + \frac{1}{r^2}\frac{\partial^2C}{\partial\theta^2}
    + \frac{\partial^2C}{\partial z^2},\ {\rm or} \\
    &=& \frac{1}{r}\frac{\partial}{\partial r}\left(
      r\frac{\partial C}{\partial r}\right)
    + \frac{1}{r^2}\frac{\partial^2C}{\partial\theta^2}
    + \frac{\partial^2C}{\partial z^2},\ {\rm or} \\
  \end{eqnarray}
\item During derivations, important points are obscured.  ``What we've
  learned'' summaries help.  Basically following outline mentioned before, on
  p. 465 of W$^3$R.  Also feel free to snooop around Athena directory...
\item How to draw exponentials in concentration profile?
\item Plotting hyperbolic trig functions---necessary?  No.  If needed, use
  calc/table.
\item Too quick jump from integ consts to carbon fiber material, missed lots.
\item Keep C$_2$H$_2$ conc constant for constant rate?  Not quite, keep it
  uniform.
\item How to calculate C$_2$H$_2$ consumption rate?  Use concentration (or
  partial pressure) and reaction constant.  But often don't know reaction
  constant, need to try at various temperatures, or just do what we did---find
  something which works, and use this methodology to understand why and how to
  make work for new designs.
\item How do we get:
  \begin{equation}
    \label{eq:secondderivative}
    \lim_{\Delta x\rightarrow 0}-D\frac{\left.\frac{dC}{dx}\right|_x -
      \left.\frac{dC}{dx}\right|_{x+\Delta x}}{\Delta x} + G =
    D\frac{d^2C}{dx^2} + G?
  \end{equation}
  Derivative of the derivative is the second derivative?
\item Arrhenius plot: which part is diffusion-limited, which part
  reaction-limited?  Can't really compare because of different units.  But can
  sort of make a plot of $\log(kL^2/D)$ vs. $1/T$, look at different parts.
  Want: low-temperature reaction-limited case, with fast diffusion to wipe out
  conc gradients.  Diffusion-limited means it doesn't diffuse in very far.

  Next Monday: reaction and diffusion in series, which dominates is more
  straightforward, can easily compare the different coefficient.
\item How did $\sqrt{k/D}L/2$ become $kL^2/D$?  What's important for design: if
  two designs have the same $kL^2/D$, then have the same $\sqrt{k/D}L/2$, same
  uniformity.  So use the simpler one to guide design decisions.  Get into
  further with dimensional analysis next week.
\item Why no flux at $x=0$?  Because of symmetry: on left, flux goes right; on
  right, flux goes left; in middle, flux goes... nowhere!  Symmetry, or
  zero-flux boundary condition, like PS2 \#3.
\end{itemize}

\paragraph{Unsteady Diffusion}

Last two times: stories to take home: increasing production rate of helium from
natural gas, high-quality manufacturing of reinforced carbon-carbon for space
shuttle wing, nose leading-edge tiles.  This time: math first, examples later,
because three different solutions to the diffusion equation, examples can use
one or more.

Now accumulation != 0, rate also in mol/sec = $V \partial C/\partial t$.
Chapter 27 material.  Resulting equation in 1-D:
\begin{equation}
  \label{eq:unsdiffusion}
  \frac{\partial C}{\partial t} = D\frac{\partial^2C}{\partial x^2} + G.
\end{equation}
Physical intuition: concentration curvature is either due to generation, or
leads to time evolution, or both.  Upward curvature means neighbors diffuse in,
either $G$ negative or $C$ increases; downward means diffuse out, either $G$
positive or $C$ decreases.

Today focus on zero-generation solutions which you've seen before, to be used
in this class (book derives them using Laplace transforms...):
\begin{itemize}
\item Error function:
  \begin{eqnarray}
    \label{eq:erfsol}
    C &=& A{\rm erf(c)} \left(\frac{x}{2\sqrt{Dt}}\right) + B, \\
    \frac{\partial C}{\partial t} &=& -\frac{x}{2\sqrt{\pi Dt^3}}
    \exp\left(-\frac{x^2}{4DT}\right), \\
    \frac{\partial C}{\partial x} &=& \frac{2}{\sqrt{\pi}}\frac{1}{2\sqrt{Dt}}
    \exp\left(-\frac{x^2}{4DT}\right), \\
    \frac{\partial^2C}{\partial x^2} &=& -\frac{1}{\sqrt{\pi Dt}}\frac{2x}{4Dt}
    \exp\left(-\frac{x^2}{4DT}\right) \\
    &=& -\frac{x}{2\sqrt{\pi D^3t^3}} \exp\left(-\frac{x^2}{4DT}\right).
  \end{eqnarray}
  So this satisfies equation \ref{eq:unsdiffusion}.

  What it looks like: graph erf, erfc; discuss constant C BC $x=0\Rightarrow
  C=C_0$, uniform C IC $t=0\Rightarrow C=C_i$.  Results:
  \begin{equation}
    \label{eq:erfit}
    C_i>C_s \Rightarrow
    C = C_s + (C_i-C_s) {\rm erf}\left(\frac{x}{2\sqrt{Dt}}\right)
  \end{equation}
  \begin{equation}
    \label{eq:erfcfit}
    C_s>C_i \Rightarrow
    C = C_i + (C_s-C_i) {\rm erfc}\left(\frac{x}{2\sqrt{Dt}}\right)
  \end{equation}

  Semi-infinite.  But what if your part is not semi-infinite, but thickness
  $L$?  (Not many parts are semi-infinite...)  Since erf(2)=0.995 and
  erfc(2)=0.005, we can approximate $\infty\simeq 2$, if
  $\frac{L}{2\sqrt{Dt}}>2$, then erf is about 1 and erfc about 0, can consider
  semi-infinite.  Solve for $t$:
  \begin{equation}
    \label{eq:seminf}
    t < \frac{L^2}{16D} \Rightarrow {\rm semi-infinite}.
  \end{equation}
  (Recall $t>L^2/D\Rightarrow$ steady-state...)

\item ``Shrinking Gaussian'':
  \begin{eqnarray}
    \label{eq:shrgaus}
    C &=& \frac{A\delta}{\sqrt{\pi Dt}}\exp\left(-\frac{x^2}{4Dt}\right) + B \\
    \frac{\partial C}{\partial t} &=& ...
  \end{eqnarray}
  Graph: initial layer of height $A+B$ in background of $B$, spreads out.
  Note: not valid for short times, only for
  \begin{equation}
    \label{eq:gaussvalid}
    2\sqrt{Dt}>\delta \Rightarrow t > \frac{\delta^2}{4D}.
  \end{equation}
  Note also (semi-)infinite, like erf valid for $t<L^2/16D$.  Note one-sided,
  two-sided; either way, BC at $x=0\Rightarrow \partial C/\partial x=0$.

  Not necessarily square initial condition!  Can even start with erf, drive in
  to shr Gaussian.
\end{itemize}
\newpage


\section{September 12, 2003: 9/11 remembered, ABET}

TODO: Bring ABET handout!

\noindent Schedule office hours.
\vspace{\baselineskip}

September 11.  The day meant a lot of things to a lot of people.  Yesterday the
occasion was commemorated in a number of ways, here in Boston, in my hometown
of New York City, and around the country and the world.  I can't hope to be as
profound as some of the speakers at those services, but can talk about a few
things it meant to me personally, in particular as I have reflected on my
decision to become an engineer, and my purpose in the profession.  Perhaps some
of it will resonate with one or two of you; I invite your comments or
questions, and we'll take as long for this as we have to.

I'd like to start nine days before the tragedy, when I was in New York for my
sister-in-law's wedding.  My wife's parents live in Brooklyn, which is where
the ceremony was held, but we were staying with her aunt and uncle in Long
Island.  At least twice a day in the few days beforehand, we drove the Belt
Parkway and Brooklyn-Queens Expressway, wrapping around Brooklyn, passing under
the Verazzano bridge and entering New York Harbor, with the view of the Statue
of Liberty and the majestic buildings rising ahead, the skyline dominated, of
course, by the World Trade Center.

During those drives, I recalled the experience of my High School French teacher
Mr. Schwartzbart, an Austrian Jew who survived World War II in a rural Belgian
boys' camp which, unknown to him at the time, was made up entirely of Jewish
boys, and in fact, was set up to keep them safe throughout the Nazi occupation.
He described the terror he felt under the occupation, and then the arrival of
the American soldiers, ``All of them giants,'' he said, then pointed to me,
``like Adam,'' they had come to set the continent free.

And he described the journey to America as a young teenager, a transforming
experience.  Most amazing was the entry of his ship carrying scores of poor
immigrants like himself into New York Harbor, this impossibly enormous bridge
which just got bigger and bigger as they approached (the Verazzano was the
longest span in the world for about 50 years), the tranquility of the harbor
within, with the great buildings visible ahead including the Empire State, and
the Statue of Liberty to his left as they steamed toward Ellis Island (the
World Trade Center's construction was still 20 years away).  There was an
awe-inspiring sense of the magnitude of this great nation of impossible size
which had overwhelmed some of the greatest evil the world had ever known, and
his heart swelled with joy at the thought that there was such power on the side
of liberty.

These days in is fashionable to reflexively cringe at the identification of
this country with freedom, and this teacher in particular very frequently
commented cynically on the deficiencies in American culture and education.
Having come to know this side of him, when we asked why he came to this
country, Mr. Schwartzbart's reply surprised us: ``The land of the free and the
home of the brave.''  Then after a pause, ``It really is true.''  His personal
experience of this gave great weight to these words.

During these drives along the Belt Parkway, my thoughts also turned toward the
fragility of the grand edifices, and in particular to the 1993 bombing of the
underground parking lot of one of the Twin Towers.  Fortunately the towers
withstood that attempt to destroy them, but there would surely be more
attempts, and no amount of devastation was too horrible for the perpetrators to
dream up.  Should anything happen, I was grateful for the opportunity to see
this beauty, and even to feel a small piece of what Paul Schwartzbart had felt
some fifty years earlier.  I thought of how fortunate is his generation which
came through the Depression, fought that terrible war, and lived to see the
nation preside over such a long and prosperous peacetime as the world had
perhaps never before known.

So you can imagine my shock when just nine days later, as I sat in my office,
my wife called from home to say that while watching CNN, they announced that a
plane had crashed into the World Trade Center.  Well, I thought, about 60 years
ago a small plane hit the Empire State building, I'm sure there was a lot of
damage and many people killed, but the rest of the building should be fine.
Just a few minutes later she called again to tell me about the second plane,
and suddenly I was afraid.  Then the Pentagon, and the missing plane in
Pennsylvania.  My thoughts turned to the Mid-East, and this administration's
policy of deliberate neglect in the Israeli-Palestinian peace process.  Then
the last call, one tower had collapsed.  With her voice choking from the tears,
she described its fall as ``like a house of cards,'' and could say little more.
Immediately, I logged out, got on my bike, and pedaled home as fast as I could.

I'm sure each of us can tell a story about where we were when it happened.
Being from New York, I was immediately concerned for friends and family.  My
wife's grandmother went to the roof of her building in Queens, from where she
saw the second plane hit the south tower, and that tower's collapse.  I had
shared this view every day growing up as I rode the Roosevelt Island tramway to
school and saw these buildings which seemed as permanent as mountains.  My
wife's best friend in College, who lives in the Prospect Heights section of
Brooklyn and works in the southern tip of Manhattan, noticed people in his
neighborhood looking up and saw some smoke, but rushed into the subway as he
was late for work; the packed subway stopped after it left Brooklyn and waited
in the tunnel for about 20 minutes before it turned back and he got out and
learned what had happened.  My Elementary School best friend worked in the 17th
floor of Tower 1, and had a bad back which would have made it painful and
difficult to get out---if he hadn't been home sick that day.

Then there was my father's friend whom I know well and whose business had just
finished moving into the 89th floor of Tower 1.  His staff had been told not to
come in that morning until 10 AM, because their carpets, freshly washed during
the 1-7 AM shift, would need to finish drying.  As he drove north on the New
Jersey Turnpike, he saw the first plane crash right through the windows of his
new office, then took the next exit and went right back to his daughter's
kindergarten class.

The previous Spring and Summer I had a course 6 UROP student in my group.  His
brother worked above the 90th floor of Tower 2, and on the first and second day
afterward without a word to anyone in his family, my student grew panicked,
then desparate, then increasingly hopeless.  His brother finally called to say
that a friend had literally dragged him from the office after the first plane
hit, and they ran out of the building together just as the second plane smashed
into it.  He described bits of the hell that was the area around him, but at
the time had no other thought than to get away, go home and lie down in shock,
not even thinking about his relatives who were trying to reach him.  My student
described the moment when they connected as one of the happiest of his life.

Another friend was not so fortunate.  Her father worked in an upper floor of
Tower 2, and was one of just two in his company who didn't make it out.  To
make matters worse, she was trapped in L.A. because of grounded planes, unable
to get back and try to locate him, so day upon day she was not only uncertain
and hurting but frustrated at being far from anyone who could help her.  She is
still grieving, as it's hard to accept that she lost the closest person to her
in the world because a handful of maniacs decided to crash a plane into his
building.

There are of course tens of thousands more stories like these, so many people
were affected directly or knew someone who was.  But even if you were not so
directly involved, if you're like me, the tragedy didn't end on that day, but
played out over and over again in your mind.  I can't count how many mornings
in the ensuing months I woke up at 3 AM thinking about the towers' collapse,
feeling hurt, afraid, angry, and much as I hate to admit it, somewhat
vindictive as we learned of the total destruction of the Al Queda camps and
cave complexes in Afghanistan.

Then thoughts turned to my own life.  What can I do, what's my role in the
world, how can I help?

I turned to the motivations I had for entering science and engineering, and
materials science in particular, which I came up with in High School.
Motivations for studying these things vary greatly, from interest in the
subject matter, elegance of the equations, beauty of nature etc., to being able
to earn a stable income and support a household, or perhaps a large income, to
serving society in some way.  My own motivations fell somewhat in the first
category, but if I had followed that alone I would have been 6-3 (computer
science); it was the last of these categories, serving society, which steered
me into Materials Science.

As a high school student, I verbalized this service as follows.  As a scientist
or engineer, I would be helping to solve the world's little problems, which I
listed as:
\begin{itemize}
\item Agriculture, to feed a growing planet.
\item Medicine, allowing people to lead longer, healthier lives.
\item Transportation and communications, to bring people together and lessen
  the chances of conflict.  For example, much of the reason war between France
  and Germany today is unthinkable is because there are so many more personal
  cross-border relationships now than in 1940 or 1914, it's very difficult for
  a propagandist to castigate an entire people as ``the enemy'' and it's
  becoming more difficult every year.
\item Human interactions with the environment, for sustainable living.
\item A recent addition, information access, with implications for democracy,
  as the biggest enemy of an authoritarian state is the truth.
\end{itemize}

All of these are important in themselves, but even more important, if we do our
jobs well and make a difference in these areas, we help the artists,
politicians, economists, philosophers and theologians to solve the big
problems, which I would list as:
\begin{itemize}
\item World peace.
\item Averting famines, and their relief.  Almost all famines can be avoided
  without resort to international aid, and are the result of poor resource
  management or hoarding.
\item Real public health, made available to those who need it around the world.
\item Justice, including somewhat equitable economic distribution.
\item Truth in journalism and history.
\item Human happiness and fulfillment.
\item Purpose and meaning for our lives.
\item Artistic expression of emotions, of values, of that purpose and meaning.
\item Last year Ross Benson added: Tolerance of differences.
\end{itemize}

An important consequence of this understanding of ``little problems'' and ``big
problems'' is that being a scientist or engineer requires a lot of {\em faith},
faith that our knowledge and our inventions will be used wisely, for good and
not for evil.  The more ``sciencey'' our contributions, the more faith is
necessary, with the ultimate example perhaps being nuclear science, which can
be used to produce lots of cheap power or cure diseases, or destroy entire
cities in an instant.  If we work on weapons, they can of course be used for
defense or for aggression.

But even if we're not working on nuclear science or weaponry, one of the
lessons of September 11 for me is that no matter how careful we are to focus on
purely non-military technologies, this tragedy showed that even a civilian
jetliner---built to bring people together---can be abused by people with
sufficient hatred as a weapon of mass destruction.  This is truly frightening
for us, and requires us to have that much more faith in the people,
institutions and systems surrounding the technology in whose development we
participate.

So what should we do?  Shall we abandon technology altogether and go back to
rubbing sticks together?  Perhaps we should join the peace corps?  For some of
us that will be the answer, but I think there's a lot more that can be done
with the little problems that can help to make a real impact on the big ones.
So how can we put ourselves in positions to do as much good as possible?

I can think of a few ways, but at your age and even at mine, perhaps the most
important is to take a step back and examine what we're doing and why.  I have
an advisee taking this class now who took off all of last Spring for that very
purpose, and ended up returning to MIT (and in fact to Materials Science) that
much more focused than the previous December for the experience.  Of course,
you don't have to take off a semester to do this, there are very good ways to
do some of this right here.

First, the HASS and HASS-D subjects present outstanding opportunities for this
kind of exploration.  MIT is no longer just about training technology leaders,
but also about training world leaders who know about technology, and this
school has put enormous resources into building world class departments in the
Humanities, Arts and Social Sciences.  For example, I've heard tremendous
things about our Anthropology department from a variety of external sources,
and even within our department we offer a HASS subject called Materials in the
Human Experience (3.094) every Spring.

Second, I've made a point of suggesting to all of my advisees that they get to
know the MISTI programs (I think that stands for MIT Internships in Science,
Technology and Industry), which do an outstanding job not just of sending
students to companies, universities and government labs in foreign countries,
but also preparing them for the trip, even culturally and psychologically.

Third, develop a habit of using your wealth to support organizations and causes
which effectively promote what you view as positive values.  You may not have
much now, but you will later, and getting into this habit is not hard;
furthermore, membership in many of these organizations requires a contribution
of as little as \$30.  If you like I can discuss offline some of the
organizations I've given to since my undergraduate years, one even since high
school.

Fourth, look for opportunities to participate in the process of improving lives
yourselves.  Whether tutoring or mentoring, or working in a social justice
organization, or writing to Congress, participating in society in a meaningful
way is important to making it all work, and I believe important to improving
ourselves too.  Believe it or not, time is actually one resource which you will
{\em not} have more of later in life than you have now, particularly if
children become part of your life.

Fifth and perhaps most importantly, get to know your fellow students.  This
buzzword is repeated over and over again, but it's worth repeating yet again:
because MIT attracts the best and brightest from all over the world, the {\em
  diversity} of the students on this campus is truly extraordinary, it's almost
certainly broader and deeper than anything you've experienced before college,
and almost certainly broader and deeper than anything you will ever experience
later in life.  That goes for Cambridge as well, and many other universities,
though much less so on the graduate level and beyond.  And by getting to know
your colleagues, I don't just mean hanging out and eating pizza, nor even
getting to know what spices they use to cook lamb, though food is of course an
important part of social interaction.  I'd encourage you to learn something
about your friends' lives, their families, their values---and be willing to
discuss these aspects of yourselves too.

And given its importance, I'd encourage you to learn something about your
friends' faith.  Human institutions, organizations, systems and even nations
are terrific, but never perfect, as we learned in a powerful way on September
11.  Participating in and strengthening them is an important and honorable
activity, but I believe that placing all of our hope on them is not viable in
the long term.  At some point they're going to let us down, as this country has
in some ways let down my French teacher Mr. Schwartzbart.  Furthermore,
evidence abounds for forces at work in the universe beyond those of physics,
and even grows with the increase of human knowledge about this universe;
perhaps the most significant example is the Anthropic Principle in Cosmology,
which some of you may have heard of and I'd be happy to discuss offline.

That concludes what I said last year: the tragedy reminds us that our work here
is very important, but must be viewed in context, and done with faith that it
will be used for the broad purposes for which we intend it.  Since last year,
time has passed and some of the emotions have subsided just a bit, also several
important things have happened, or have not happened, causing my own feelings
about this to be somewhat more complicated.

For one thing, the message from Washington continues to urge us to live out
lives as if nothing had happened, because if we changed anything, we'd be
giving the terrorists what they want.  But this is foolishness, important
things have changed, and as citizens there are things we can do on a daily
basis to improve our country's security, and the silence from Washington has
been deafening.

A few months ago I saw a book provocatively titled, ``When you ride alone, you
ride with bin Laden.''  The cover art was derived from a World War II poster,
``When you ride alone, you ride with Hitler,'' whose point was that the
practices of avoiding driving, carpooling, and using public transportation save
gasoline needed for the war effort.  In that vein, an important thing which has
{\em not} happened is that there has been no effort whatsoever on a national
level to reduce our dependence on imported oil, which has been a huge factor in
our problems in the Middle East.  In fact, we've seen the opposite in this
administration's rollback in fuel economy standards, and heard talk about the
costs to the auto manufacturers and consumers of requiring increases in
efficiency, with no mention whatsoever of the multitude of costs of continuing
to burn fossil fuels as extravegantly as we like.

Another thing which has changed my view of the world was the war in Iraq.  For
months, the administration hyped any evidence at all for Iraqi connections to
Al Qaeda and possession or development of weapons of mass destruction.  Then
inspections were allowed (to be fair, largely due to U.S. pressure, and no
thanks to the posturing of certain countries like France), and one-by-one the
inspections eliminated every piece of purported WMD evidence save the rumor
about uranium purchase from Niger.  And so with that one rumor as
justification, we sent an invasion force to Kuwait, and used technological
superiority to destroy the Iraqi army in about four weeks.

I believe that the war was wrong for a number of reasons.  First, as mentioned,
it was completely unjustified.  There was no correlation to terrorism, none to
weapons of mass destruction, and there are a dozen dictators around the world
whose human rights violations could similarly motivate action.  If violation of
U.N. resolutions was the motive, then it's up to the U.N. to act.

Second, this war was an extremely imprudent waste of resources.  It is unwise
to spend a couple of hundred billion dollars on a flimsily-justified war in a
time of record deficits, and even more imprudent to tie down fifteen of the
U.S. Army's thirty-three combat brigades in that occupation during a time of
multiple threats from WMD in North Korea and Iran, to shaky stability in
Afghanistan, to low-level al Qaeda activity from Indonesia to Somalia.  And
most significant is the loss of hundreds of American and thousands of Iraqi
lives, very nearly including that of my own brother, a Captain in the U.S.
Army third infantry division's second brigade, whose unit was hit by an Iraqi
missile just after their capture of downtown Baghdad.  Like my Course 6 UROP
student two years ago, the days between learning of the attack and confirming
my brother's safety were some of the longest of my life.  I cannot imagine the
terror and grief of hundreds of thousands of loved ones of Iraqi soldiers who
did not know for weeks or months whether their sons, brothers, husbands or
fathers were dead or alive, nor the pain of those whose worst fears were in the
end confirmed.

Finally, the war has cost us diplomatically, and will continue to cost us
potentially for many generations.  The arrogance with which the administration
shrugged off overwhelming international opposition was shocking, prticularly as
it involved three of the five Security Council permanent members, some of our
most important allies, and both of our neighbors.  Even more shocking was the
pathetically childish pettiness with which the administration spoke of
``punishing'' the opponents to the war---particularly the French---and then
turned around to ask them to contribute money and troops to the occupation and
reconstruction to reduce the resource burden on us.  But in the long term, the
most diplomatically destructive aspect of the war is the precedent it sets for
accusing a nation of violations of one sort or another, putting aside
international outcry for restraint, and using military superiority to crush the
weaker victim.  This precedent can easily be abused by, say, Turkey, Syria,
Egypt, Jordan and Saudi Arabia against Israel, North Korea against the South,
Russia against the former Soviet Republics, China against Taiwan---any nation
with a fight to pick can say, ``But of course it's been done before, by the
Land of the Free and Home of the Brave!''

And so we are reminded of our duty as citizens to speak out about matters of
importance to our country.  And in particular, as scientists and engineers we
have the duty to speak out with authority on certain issues such as the small
cost of reducing energy consumption, and the enormous costs of not doing so.
Most of all, in the changed world our need for faith is greater than ever, in
our work as well as our outlook for the future.  I welcome any comments,
contributions, or questions.

\paragraph{ABET Sheet Review}

See the ABET sheet.

\paragraph{Unsteady Diffusion Continued}

Muddy stuff:
\begin{itemize}
\item Symbol conventions: $C_i$ is initial, $C_0$ I use as surface and
  sometimes other things; $C_s$ always surface concentration.
\item Physical meaning of graphs, interpretation of these things: coming soon.
\item Exact criteria for each solution: will be summarized.
\item Semi-infinite and erf validity.  First, semi-infinite applies to
  one-sided erf and one-sided shrinking Gaussian.  Fully infinite applies to
  two-sided both.  Second, the criterion is derived from the error function:
  $t<L^2/16D \Rightarrow {\rm erf}(x/2\sqrt{Dt}) > 0.995$ and erfc is below
  0.005.  Same criterion for erf, erfc, shrinking Gaussian.
\item How do the M and top-hat initial conditions give the same equation?  Over
  a long time, the details of the initial condition smooth out and become
  Gaussian.
\end{itemize}

\noindent Continuing where we left off last time (did the erf and shrinking
  Gaussian):
\begin{itemize}
\item Fourier Series:
  \begin{eqnarray}
    \label{eq:fourier}
    C &=& a \exp(-b^2Dt) \sin(bx) ({\rm or }\ \cos), \\
    \frac{\partial C}{\partial t} &=& -ab^2D \exp(-b^2Dt) \sin(bx), \\
    \frac{\partial C}{\partial x} &=& ab \exp(-b^2Dt) \cos(bx), \\
    \frac{\partial^2 C}{\partial x^2} &=& -ab^2 \exp(-b^2Dt) \sin(bx). \\
  \end{eqnarray}
  Elegant and simple.  Separation of variables: all of the $t$ in one term, all
  of the $x$ in the other $f(t)g(x)$.  [Eigenfunctions...]

  Graph: sine wave decays with time.
\end{itemize}
\newpage


\section{September 15, 2003: Wrapup unsteady, boundary conditions}
\label{unstdiff}

Muddy from last time:
\begin{itemize}
\item Appreciate 9/11 reflections, not political opinions.  Apologies to those
  affected, particularly language.  Forms a strong part of 9/11 feelings,
  particularly as administration has done its best to tie them together.
  Uncomfortable with talking about one side and not the other.  War must always
  be a last resort when other means e.g. negotiation have been tried.  Defining
  objectives as narrowly as ``regime change'' with neither aggression nor clear
  and present danger (Bush {\em and} Clinton) is as misguided and wrong as
  extremists in the region who call for destrurction of a certain state ``by
  any means necessary''.  Open for last words...
\end{itemize}

\noindent Continuing where we left off last time (did the erf and shrinking
  Gaussian):
\begin{itemize}
\item Fourier Series:
  \begin{eqnarray}
    \label{eq:fourier2}
    C &=& a \exp(-b^2Dt) \sin(bx) ({\rm or }\ \cos), \\
    \frac{\partial C}{\partial t} &=& -ab^2D \exp(-b^2Dt) \sin(bx), \\
    \frac{\partial C}{\partial x} &=& ab \exp(-b^2Dt) \cos(bx), \\
    \frac{\partial^2 C}{\partial x^2} &=& -ab^2 \exp(-b^2Dt) \sin(bx). \\
  \end{eqnarray}
  Elegant and simple.  Separation of variables: all of the $t$ in one term, all
  of the $x$ in the other $f(t)g(x)$.  [Eigenfunctions...]

  Graph: sine wave decays with time.  Like a sinusoidal layered material being
  annealed in time.  If period=2$L$, then $b=\pi/L$, get:
  \begin{equation}
    \label{eq:sinefit}
    C = a\exp\left(-\frac{\pi^2Dt}{L^2}\right)\sin\left(\frac{\pi x}{L}\right).
  \end{equation}

  But what use is a sine wave?  Note linearity of diffusion equation: any sum
  of solutions is also a solution.  So we can add sine waves to get something
  more useful, use that.

  Fourier transform: express {\em any} initial condition as sum of sine waves.
  We'll do one: square wave of period $2L$ for multilayer material annealing,
  each term has period $2L/n$ so $b_n=n\pi/L$:
  \begin{equation}
    \label{eq:squarefourier}
    C = \sum_{n=0}^\infty a_n \exp\left(-\frac{n^2\pi^2Dt}{L^2}\right)
    \sin\left(\frac{n\pi x}{L}\right).
  \end{equation}
  The Fourier transform:
  \begin{equation}
    \label{eq:fft}
    a_n = \frac{4}{n\pi},\ n\ {\rm odd},\ 0,\ n\ {\rm even}.
  \end{equation}
  Illustrate different sine functions, how they add to a square wave.  Result:
  \begin{equation}
    \label{eq:squarefourierfit}
    C = C_0 + (C_{max}-C_0) \sum_{n=1, n\ {\rm odd}}^\infty \frac{4}{n\pi}
    \exp\left(-\frac{n^2\pi^2Dt}{L^2}\right) \sin\left(\frac{n\pi x}{L}\right).
  \end{equation}

  So we start with all these sine waves, then what happens?  Higher-order terms
  shrink real fast.  Graph amplitude vs. time, show $n=3$ drops out nine times
  faster, $n=5$ twenty-five times faster.  Do it as $t/\tau$, where
  $\tau=L^2/\pi^2D$, so first term is $\exp(-t/\tau)$, second $\exp(-9t/\tau)$.
  One term remains real fast.

  Consider: $t=L^2/D$...  What is single-term max concentration at that time?
  $4/\pi \times \exp(-\pi^2) = 6.5\times10^{-5}$, quite close to zero!

  Other application: finite system with thickness $L$, uniform IC, constant $C$
  boundary conditions. Graph, show the ``virtual'' wave outside.  Note that at
  short times, erf is easier; long times, one-term sine wave is easy.

  TA wanted to do 2-D and 3-D separation $C=f(t)g(x)h(y)$...

  Suppose something more like first lecture, section \ref{diff1dss} (p.
  \pageref{diff1dss}): initial flat, one side raised?  Then linear plus Fourier
  series, with odd and even $n$; even so, at $t=L^2/D$, first term, max conc is
  a tiny fraction of original.  So, very close to steady-state!
\end{itemize}

\paragraph{Which to use?} Summary of criteria:

\begin{itemize}
\item Error functions erf, erfc:
  \begin{itemize}
  \item (Semi-)infinite
  \item Uniform initial condition at $t=0$ equal to boundary condition at
    $x=\infty$
  \item Constant concentration boundary condition at $x=0$, infinite
    source/sink backing it up.
  \end{itemize}

\item Shrinking Gaussian $\frac{A\delta}{\sqrt{\pi Dt}}e^{-x^2/4Dt} + B$:
  \begin{itemize}
  \item (Semi-)infinite
  \item Fixed amount of material already in solid and diffusing into infinity
  \item Gaussian ``width'' $2\sqrt{Dt}$ much larger than $\delta$
  \end{itemize}

\item Fourier series:
  \begin{itemize}
  \item Infinite 1-D sine or square wave initial condition, {\em or}
  \item Finite layer thickness $L$
  \item Uniform initial condition at $t=0$
  \item Constant concentration boundary conditions at $x=0$, $x=L$
  \end{itemize}
\end{itemize}

\noindent Examples:
\begin{itemize}
\item Decarburizing steel sheet: initial concentration $t=0\Rightarrow C=C_i$,
  boundaries $x=0,L\Rightarrow C=C_s=0$ due to oxidation.  Start erf, go to
  Fourier.
\item Semiconductor devices: initial treatment with phosphorous-containing gas,
  initially no phosphorous so $t=0\Rightarrow C_P=0$; fixed concentration at
  surface $C=C_eq$ or $C_s$ or $C_0$.  Erf.  Then seal the top, no more gas,
  drive-in diffusion gives shrinking Gaussian.
\item Galvanizing steel: thin layer of zinc on iron.  Initial:
  $x<\delta\Rightarrow C=C_0$, $x>\delta\Rightarrow C=0$.  Start erf (diffusion
  couple) but centered at $x=\delta$
  \begin{equation}
    \label{eq:galvcouple}
    C=\frac{C_0}{2}{\rm erfc}\left(\frac{x-\delta}{2\sqrt{Dt}}\right).
  \end{equation}
  This holds for as long as the zinc is semi-infinite, i.e. $t<\delta^2/16D$.
  For long times $t>\delta^2/D$, finite amount of zinc means shrinking
  Gaussian:
  \begin{equation}
    \label{eq:galvgauss}
    C=\frac{C_0\delta}{\sqrt{\pi Dt}} \exp\left(-\frac{x}{4Dt}\right).
  \end{equation}
\end{itemize}
\newpage


\section{September 17, 2003: Boundary conditions, layer growth}

\noindent Fun opener: Michael Dixon on calculus in physics in Dorchester.

Muddy from last time:
\begin{itemize}
\item Square wave: why is $a_n$ zero for even $n$?  That's the Fourier
  transform; would throw off symmetry.
\item How erfs in infinite square wave IC?  Like diffusion couple...
\item What is plotted on the graph of amplitude vs. time?
\item Steady Fourier: $\frac{4}{\pi}\exp(-\pi^2)$ constant?  No, wave with that
  amplitude at $t=L^2/D$; for longer time asymptotically approaches $C=C_{av}$.
\item In galvanizing situation, why erf for small times?  Isn't it a finite
  amount of zinc?  At very small times, can consider even the zinc as
  semi-infinite.  Also, at small times $t<\delta^2/D$, shrinking Gaussian
  doesn't work, goes to infinitely tall and thin.
\item What to use in example 3 between $\delta^2/16D<t<\delta^2/D$?  Nothing
  covered here; perhaps another Fourier transform...
\item Write down initial conditions for examples...  Added to lecture notes, in
  my home directory.
\end{itemize}

\paragraph{Diffusion boundary conditions}

Types:
\begin{itemize}
\item Constant $C$: what's that C?
\item Constant flux: sealed means zero, sometimes ion beams, etc. (not often)
\item Flux vs. C, for chemical reaction or mass transfer coefficient.  $J=k
  (C-C_{eq})$ (note ChemEs' $k''$...)  Units: $k$ in cm/s.  Also mass transfer
  coefficient through fluid film: $J=h_D (C-C_{eq})$, where $h_D = \delta/D$.
\end{itemize}

Example: gas diffusion in metal.  Oxygen, nitrogen, hydrogen in metals:
diffuses monoatomically, $\rm 2 O(m) \Leftrightarrow O_2(g)$, equation:
\begin{equation}
  \label{eq:diatomic}
  K_{eq} = \frac{p_{\rm O_2}}{[O]^2}.
\end{equation}
So oxygen concentration at equilibrium is proportional to the square root of
pressure.  (Not the same for helium, argon, etc.)  Then if flux is proportional
to difference in concentrations, it's proportional to square root of difference
in partial pressures!

\noindent Misconception 1: Equilibrium is {\bf NOT} steady-state.  Global
equilibrium here would be a brick of SiO$_2$.  Local equilibrium gives
concentrations at interface sometimes.  Can be out of local equilibrium and at
steady state too.

\noindent Misconception 2: kinetic reaction rate order is {\bf NOT}
thermodynamic order.

\paragraph{Layer growth}

Motivating example: silicon oxidation (example W$^3$R pp. 487-489).  Electronic
components, this is the ``gate oxide'' in MOSFETs, draw p-n-p structure.
Thickness must be tightly controlled for the FET to work---to switch at the
right voltage.  How to make the oxide?  Expose it to air, and it just grows.
Cool.  Wet oxidation too: expose it to steam and it grows differently
(slower?).  How much does it grow, how long to leave it there?

Go to the phase diagram, this one is pretty simple.  Start with equilibrium.
Many times, can use the phase diagram.  Others, with impure phases (like air),
or partial pressure or total pressure dependency, need activity to get
equilibrium concentration at interface.

Out of equilibrium, but interfaces at local equilibrium: diffusion-limited,
explain the concept of pseudo-steady-state.  Growth is really slow because,
although $D$ might be high leading to steady-state, $C_3-C_1$ is really small,
leading to slow flux, really slow growth of the layer.  Growth rate:
\begin{equation}
  \label{eq:fluxgrow}
  J_O = \frac{\rm 2\ moles\ O}{\rm 1\ mole\ SiO_2}
\frac{\rho_{\rm SiO_2}}{MW_{\rm SiO_2}} \frac{dY}{dt}.
\end{equation}
Why?  Proportional to $J_O$, then it's just a matter of working out units.

No.  $J\Delta t$ is the amount over a certain amount of time, $(C_1-C_0) \Delta
Y$ is the area on the graph needed to transform that much Si to SiO$_2$.

\noindent For equilibrium, set $J=\Delta C/Y$, solve for $Y$:
\begin{eqnarray}
  \label{eq:parabolicgrowth}
  \frac{C_3-C_1}{Y} &=& \frac{\rm 2\ moles\ O}{\rm 1\ mole\ SiO_2}
  \frac{\rho_{\rm SiO_2}}{M_{\rm SiO_2}} \frac{dY}{dt} \\
  Y dY &=& \frac{\rm 1\ mole\ SiO_2}{\rm 2\ moles\ O}
  \frac{M_{\rm SiO_2} (C_3-C_1)}{\rho_{\rm SiO_2}} dt \\
  \frac{Y^2-Y_0^2}{2} &=& \frac{\rm 1\ mole\ SiO_2}{\rm 2\ moles\ O}
  \frac{M_{\rm SiO_2} (C_3-C_1)}{\rho_{\rm SiO_2}} (t-t_0).
\end{eqnarray}
Parabolic growth.  Plotted in text, p. 489.  But the plot doesn't quite work!
\newpage


\section{September 19, 2003: Layer Growth, Dimensional Analysis}

Mechanics etc.
\begin{itemize}
\item Forgot last time: test 1 October 13?
\item Monday: no class, but I will be here for office hours 2:30-3:30 as usual;
  Albert?
\item Next Thurs 9/25: GE CEO Jeffrey Immelt at Bartos theater (downstairs
  Media Lab)...
\end{itemize}

\noindent Muddy from last time:
\begin{itemize}
\item Pseudo-steady-state of what?  Diffusion: growth is slow, so conc profile
  reaches steady-state.
\item Why $J_0 = \frac{\Delta Y}{\Delta t}(C_1-C_0)$?  Explain: $J_O\Delta t$
  is the amount of oxygen which the flux has fed through.  $(C_1-C_0)\Delta Y$
  is the amount of oxygen needed to transform $\Delta Y$ worth of silicon into
  silicon dioxide.
\item Why $Y(t)$?  Is all diffusion-limited growth proportional to $\sqrt{t}$?
\end{itemize}

\paragraph{Layer growth}

Motivating example: silicon oxidation (example W$^3$R pp. 487-489).  How much
does it grow, how long to leave it there?

Back to phase diagram: $C_0$, $C_1$ and $C_3$ as three equilibrium
concentrations at operating temperature.  Out of equilibrium, but interfaces at
local equilibrium: pseudo-steady-state.

For equilibrium, set $J=\Delta C/Y$, solve for $Y$:
\begin{eqnarray}
  \label{eq:parabolicgrowth2}
  \frac{Y^2-Y_0^2}{2} &=& \frac{\rm 1\ mole\ SiO_2}{\rm 2\ moles\ O}
  \frac{M_{\rm SiO_2} (C_3-C_1)}{\rho_{\rm SiO_2}} (t-t_0).
\end{eqnarray}
Parabolic growth.  Plotted in text, p. 489.  But the plot doesn't quite work!

Next out of local equilibrium: say 1st order chemical reaction as the slow
step.  (Note: kinetic order vs. thermodynamic equilibrium exponents!)  Then
introduce $C_2$ between $C_1$ and $C_3$, say the rate is proportional to
$C_2-C_1$, proportionality constant is the reaction rate coefficient which we
call $k$.  Very short times:
$$J = k(C_2-C_1) \simeq k(C_3-C_1),$$
constant growth rate, {\em linear} film growth.

\label{thecalc}
Suppose $C_3$ at oxide outer surface, $C_2$ at back interface, $C_1$ in
equilibrium with silicon, reaction limit with constant $k$, $J=k(C_2-C_1)$.
Want $J(C_3,C_1)$.  Solve all together:
$$J = k(C_2-C_1) = \frac{D}{Y}(C_3-C_2) \left[\Rightarrow
  \frac{C_3-C_2}{C_2-C_1} = \frac{kY}{D}\right]$$
$$\left(k+\frac{D}{Y}\right) C_2 = k C_1 + \frac{D}{Y} C_3$$
$$C_2 = \frac{kC_1 + \frac{D}{Y} C_3}{k+\frac{D}{Y}} =
\frac{\frac{kY}{D} C_1 + C_3}{\frac{kY}{D} + 1}$$
Now get $J$:
$$J = k\left(\frac{kC_1 + \frac{D}{Y} C_3}{k+\frac{D}{Y}} - C_1\right)$$
$$J = k\left(\frac{\frac{D}{Y} C_3 - \frac{D}{Y} C_1}{k+\frac{D}{Y}}\right)$$
$$J = \frac{C_3 - C_1}{\frac{Y}{D} + \frac{1}{k}} =
\frac{k(C_3-C_1)}{\frac{kY}{D} + 1}$$
Resistances in series.  What dominates?  Biot number!  $\frac{kY}{D}$  Ratio of
resistances.  Small times and small $Y$ mean small Biot, reaction-limited.
Long times and large $Y$ mean large Biot, diffusion-limited.

\paragraph{Dimensional analysis}

W$^3$R pp. 140--142.  Definitions:
\begin{itemize}
\item Base units: m, s, mol
\item Derived units: cm/s, N, mol/cm$^3$
\item Dimensions: $L$, $t$, $C_O$
\end{itemize}

\paragraph{Step 1} Postulate desired behavior as a function of the other
variables, {\em e.g.} $J_O = f(C_1-C_3, k, D, Y)$, or $f (J_O, C_1-C_3, k, D,
Y) = 0$.  The number of parameters is the number of dimensions $n$, in this
case $n=5$.

This is done by intuition, and is very often the hardest step in the process.

\paragraph{Step 2} Find the number of base units in the system $r$.  Here: cm,
s, mol, so $r=3$.

\paragraph{Step 3} Buckingham Pi theorem: number of dimensionless groups =
$n-r$.

\paragraph{Step 4} Choose $r$ dimensionally-independent variables to eliminate,
which will make the others dimensionless.  Here we'll choose $C_1-C_3$, $D$,
and $Y$.

Counterexample: can't use $k$, $D$ and $Y$ because they're not independent!
Very often there are multiple ``right answers'' (fluid dynamics), choose the
one which is most convenient.

\paragraph{Step 5}  Form the $\pi$ groups from what's left, which are unitless
versions of the parameters.  Dimensionless $J$, called $\pi_J$, is $J \cdot
[C_3-C_1]^a \cdot [D]^b \cdot [Y]^c$.  Easy way: make a table with base units
across the top, start with dimensions of $J$.  Which of the eliminated units
have moles?  $C_3-C_1$, so we can say $a=-1$ and moles are cancelled.  Then
which have seconds?  $D$, so we can say $b=-1$ and seconds are cancelled.  Now
there's just cm$^{-1}$, so $c=1$ and we're done:
$$\pi_J = \frac{J_O Y}{(C_3-C_1) D}.$$

Likewise, $\pi_k$ starts with $k$ in m/s, so $a=0$, $b=-1$, $c=1$.  Look, it's
the mass transfer Biot number!

\paragraph{Step 6} Rewrite Step 1 in dimensionless terms, and we're done:
$$\pi_J = f(\pi_k).$$
What's this?  So simple?  Can't be.

Let's test:
$$J_O=\frac{C_3-C_1}{\frac{1}{k} + \frac{Y}{D}}$$
Mult by $\frac{Y}{(C_3-C_1)D}$ to give
$$\frac{J_O Y}{(C_3-C_1)D} = \frac{1}{\frac{D}{kY} + 1} =
\frac{1 + \frac{D}{kY} - \frac{D}{kY}}{\frac{D}{kY} + 1} =
1 - \frac{1}{1+\frac{kY}{D}}$$
So,
$$\pi_J = 1 - \frac{1}{1+\pi_k}$$
Limiting cases: large $\pi_k$ means $\pi_J = 1-0 = 1$, so $J_O =
\frac{D}{Y}(C_3-C_1)$.

For small $\pi_k$, use $\frac{1}{1+x}\simeq1-x$ near $x=0$, so $\pi_J = \pi_k$,
$J_O = \frac{D}{Y}(C_3-C_1)\frac{kY}{D} = k(C_3-C_1)$.  Excellent!

Purpose: simplify down to an easier expression, single graph.  If couldn't
solve equation, single graph could be obtained from one experiment, generalized
to any other reaction-diffusion problem of the same nature.  Physical modeling,
e.g. wind tunnel: get the dimensionless numbers right, every detail of flow is
the same, dimensionless drag force is identical!

\noindent This ends diffusion, next week heat conduction!



\chapter{Heat Conduction}

\section{September 24, 2003: Wrap up dimensional analysis, start heat
  conduction}

Mechanics:
\begin{itemize}
\item Handout: heat conduction equation solutions.
\item GE CEO tomorrow noon Bartos
\item Tests 1, 2 (10/10, 11/19) first part in 2-143.
\item Final Mon 12/15 1:30-4:30 ``2-105''...
\end{itemize}

\noindent Muddy stuff from last time:
\begin{itemize}
\item How steady-state diffusion in oxide?
\item How is $C_1-C_0=2\frac{\rho}{M}$?  $C_1-C_0\gg C_3-C_1$ and $C_1-C_0\gg
  C_0$, so for this purpose, $C_1=C_3=$moles oxygen/unit volume in SiO$_2$ and
  $C_0\simeq 0$.  Molar density of SiO$_2$ is $\rho/M$, molar density of oxygen
  is twice that.
\item Dimensional analysis was fast.  Yes, learning the steps is easy, but
  ``How to choose which variables to `postulate desired behavior'?''  Not
  easy, learn by example---we'll do this many more times this term...
\item How to form dimensionless quantities?  If counted $i=n-r$ correctly, and
  chose dimensionally-independent parameters to eliminate, then like
  simultaneous equations: units of $J$ * (units of $\Delta C)^a$... etc.  Table
  as an easier way of doing that.  Will do an example today with $\pi_k$.
\item (Multiple people) How is $\pi_J$ a function of $\pi_k$?  Stay tuned for
  the dramatic conclusion of dimensional analysis...
\end{itemize}

\paragraph{Dimensional Analysis}

Recap last time:
\begin{enumerate}
\item Postulate desired behavior as a function of the other variables, {\em
    e.g.} $J_O = f(C_1-C_3, k, D, Y)$, or $f (J_O, C_1-C_3, k, D, Y) = 0$.  The
  number of parameters is $n$, in this case $n=5$.

\item Find the number of base units in the system $r$.  Here: cm, s, mol, so
  $r=3$.  (Rank of the dimensional matrix...)

\item Buckingham Pi theorem: number of dimensionless groups = $n-r$.

\item Choose $r$ dimensionally-independent variables to eliminate, which will
  make the others dimensionless.  Here we'll choose $C_1-C_3$, $D$, and $Y$
  (NOT $k$, $D$ and $Y$ because they're not independent!)

\item Form the $\pi$ groups from what's left, which are unitless versions of
  the parameters kept:
  $$\pi_J = \frac{J_O Y}{(C_3-C_1) D}, \pi_k = \frac{kY}{D}.$$
\item Rewrite Step 1 in dimensionless terms, and we're done: $\pi_J =
  f(\pi_k).$
\end{enumerate}
What's this?  So simple?  Can't be.  Let's test:
$$J_O=\frac{C_3-C_1}{\frac{1}{k} + \frac{Y}{D}}$$
Mult by $\frac{Y}{(C_3-C_1)D}$ to give
$$\frac{J_O Y}{(C_3-C_1)D} = \frac{1}{\frac{D}{kY} + 1} =
\frac{1 + \frac{D}{kY} - \frac{D}{kY}}{\frac{D}{kY} + 1} =
1 - \frac{1}{1+\frac{kY}{D}}$$
So,
$$\pi_J = 1 - \frac{1}{1+\pi_k}$$
Limiting cases: large $\pi_k$ means $\pi_J = 1-0 = 1$, so $J_O =
\frac{D}{Y}(C_3-C_1)$.

For small $\pi_k$, use $\frac{1}{1+x}\simeq1-x$ near $x=0$, so $\pi_J = \pi_k$,
$J_O = \frac{D}{Y}(C_3-C_1)\frac{kY}{D} = k(C_3-C_1)$.  Excellent!

Purpose: simplify down to an easier expression, single graph.  If couldn't
solve equation, single graph could be obtained from one experiment, generalized
to any other reaction-diffusion problem of the same nature.  Physical modeling,
e.g. wind tunnel: get the dimensionless numbers right, every detail of flow is
the same, dimensionless drag force is identical!

\paragraph{Heat Conduction}

Conservation of math (in one ear, out the other).  But
seriously, conservation of thermal energy, for us enthalpy.  Usual equation:
$$\rm accumulation = in - out + generation$$
$$V\frac{dH}{dt} = Aq_{\rm in} - A q_{\rm out} + V\dot{q}$$
Note on the accumulation term: when temperature changes, enthalpy changes
according to the heat capacity, build up units from dT/dt (Kelvin/sec) adding
$c_p$ and $\rho$ to get to Joules/sec.

What's heat flux $\vec{q}$?  Like diffusion goes down the conc gradient
(actually, chem potential gradient), heat goes down the temperature gradient,
proportionality constant $k$:
\begin{equation}
  \label{eq:fouriersfirst}
  \vec{q} = -k \nabla T.
\end{equation}
Using that in-out and that accumulation term, derive the 1-D heat equation,
same as diffusion in section \ref{unstdiff} (p. \pageref{unstdiff}).  Simplify
constant $k$, 1-D, so:
\begin{equation}
  \label{eq:fouriersecond}
  \rho c_p\frac{\partial T}{\partial t} = k \frac{\partial^2T}{\partial x^2} +
  \dot{q}.
\end{equation}
Define thermal diffusivity $\alpha = k/\rho c_p$, with no gen reduces to
diffusion equation, and give 1-D solutions:
\begin{itemize}
\item 1-D steady-state: linear temperature.
\item Cylindrical steady-state: $T=A\ln r +B$; with uniform generation: $T=A\ln
  r+B-Gr^2/2$
\item 1-D semi-infinite uniform initial, constant $T$ boundary:
  \begin{equation}
    \label{eq:erftemp}
    \frac{T-T_s}{T_\infty-T_s} =
    {\rm erf} \left(\frac{x}{2\sqrt{\alpha t}}\right).
  \end{equation}
\item 1-D finite, uniform initial $T$, boundary constant $T$: Fourier series
  \begin{equation}
    \label{eq:fouriertemp}
    \frac{T-T_s}{T_i-T_s} = \sum_{n=1}^\infty a_n
    \exp\left(-\frac{n^2\pi^2\alpha t}{L^2}\right)
    \sin\left(\frac{n\pi x}{L}\right)
  \end{equation}
\end{itemize}
Even more on the handout, not responsible for any further than handout (and not
asterisks either).

\noindent Timescale to steady-state... any guesses?

Optional: Why does heat go down the temp gradient, and diffusion down the chem
potential gradient?  Thermo: increasing $S$ or decreasing $G$.  Spinodal
decomposition: negative $\partial^2G/\partial C^2$, uphill diffusion!  Fourier
series in reverse...
\newpage


\section{September 26, 2003: Heat conduction: boundary layers, multilayer wall}

Opener: Christine Ortiz talk on how inquisitive this class is...

\noindent Mechanics:
\begin{itemize}
\item PS3 due today, PS4 due Monday 10/6.
\item Last test 1 material next Wednesday, following math quiz in 2-143!
\item Tests 1, 2 (10/10, 11/19) first part in 2-143.
\item Final Mon 12/15 1:30-4:30 ``2-105''...
\item Zeiss materials microscopy truck at Chapel Turnaround 10/2 9-4.
\end{itemize}

\noindent Muddy from last time:
\begin{itemize}
\item Why is $\pi_J=\pi_k$ at small $\pi_k$?  Okay.  For $x\simeq 1$,
  $\frac{1}{x}\simeq 2-x$.  So, for $\pi_k\simeq 0$, $1+\pi_k\simeq 1$, and
  $$\frac{1}{1+\pi_k}\simeq2-(1+\pi_k)=1-\pi_k$$
  $$\pi_J = 1-\frac{1}{1+\pi_k} \simeq 1-(1-\pi_k) = \pi_k$$
\end{itemize}

\paragraph{Boundary conditions}
\begin{itemize}
\item Constant temperature.
\item Constant flux $\vec{q} \cdot \hat{n}$ const, in 1-D $q_x =
  -k\frac{\partial T}{\partial x}$.
\item Heat transfer coefficient: $q_x = h (T-T_{env})$.  UNITS!
\end{itemize}

On the last, think about a boundary layer of thickness $\delta$ in a fluid,
model $h$ as $k_f/\delta$.  Then we can look at steady-state heat conduction
through a plate, in particular the heat flux ($T_1$ BC on bottom,
$h(T_2-T_{env})$ BC on top):
$$q = \frac{k}{L}(T_1-T_2) = h (T_2-T_{env})$$
$$q = \frac{T_1-T_{env}}{\frac{1}{h} + \frac{L}{k}}$$
Awesome!  Now you know W$^3$R chapters 17 and 18---well, mostly.

\paragraph{Multilayer wall}

With lots of layers, just add up the resistances...
$$q = \frac{T_0-T_n}{\frac{1}{h} + \frac{L_1}{k_1} + \frac{L_2}{k_2} + ... +
  \frac{L_n}{k_n} + \frac{1}{h_2}}$$
Same $q_x$ everywhere implies that layers with higher $k$ have lower $\partial
  T/\partial x$.

Cylindrical is slightly different, uses flux-area product, based on log
solution:
$$Q=qA = \frac{2\pi L(T_1-T_4)}{\frac{1}{k_1}\ln\frac{R_2}{R_1}+
  \frac{1}{k_2}\ln\frac{R_3}{R_2}+\frac{1}{hR_3}}$$
Temperature trick: use Biot number equivalent:
$$\frac{T_0-T_2}{T_2-T_n} = \frac{{\rm resistances\ bet\ 2\ and\ }n}
{\rm resistances\ bet\ 0\ and\ 2}$$
\newpage


\section{September 29, 2003: Finally, the graphs!}

Fun opener: typeset homework and course evaluation handwriting...

\noindent Mechanics:
\begin{itemize}
\item PS4 due Monday 10/6.
\item Last test 1 material Wednesday, following math quiz in 2-143!
\item Tests 1, 2 (10/10, 11/19) first part in 2-143.
\item Final Mon 12/15 1:30-4:30 ``2-105''...
\item Zeiss materials microscopy truck at Chapel Turnaround 10/2 9-4.
\end{itemize}

\noindent Moddy from last time:
\begin{itemize}
\item How did we get:
  $$q_x = \frac{T_1-T_{fl}}{\frac{1}{h} + \frac{L}{k_s}}?$$
  Yes, skipped some steps because the math here is the same as the math for
  diffusion.  See notes of 9/19 (section \ref{thecalc}, page \pageref{thecalc})
  for the derivation.
\item Temperature for large Biot, small Biot...
\item Albert: parallel composite wall...
\item Blackboard technique...
\end{itemize}

Today's motivating example: powder metallurgy by spray/gas atomization.  Small
droplets, very rapid cooling, rapid solidification microstructures, solute
trapping.

So, suppose initial condition $T=T_i$, outside fluid at $T_{fl}$.  Boundary
conditions: $r=R \Rightarrow q_r = h (T-T_{fl})$.  Want to know temperature
distribution through time, or temperature history.  This requires a Bessel
function series!!  How to do understand?
\begin{itemize}
\item Dimensional analysis!
\item Qualitative description of behavior.
\item Graphs in text.
\item Simpified low Biot number behavior: Newtonian cooling.
\end{itemize}

Dimensional analysis:
\begin{enumerate}
\item Formulation: $T-T_{fl} = f(t, r, R, T_i-T_{fl}, h, k, \rho c_p)$.  $n=8$
  parameters!
\item Base units: K, s, m, kg so $m=4$.
\item Buckingham pi: four dimensionless parameters.
\item What to eliminate?  Want to keep $T-T_{fl}$, $t$, $r$; choose $h$ also.
  Eliminate $R$, $T_i-T_{fl}$, $k$, $\rho c_p$.
\item $\pi_T$ is easy, as is $\pi_r$.  $\pi_h$: eliminated by $k$ and $R$.
  $\pi_t$ is funny, use $k$ for seconds, $\rho c_p$ for Joules, $R$ for
  remaining meters.  Result is the Fourier number, the ratio of $t/t_{SS}$.

  Note: could have used $h$ to eliminate seconds, but result wouldn't have been
  as cool: $\pi_t = ht/\rho c_p R$.
\item Dimensionless equation:
  $$\frac{T-T_{fl}}{T_i-T_{fl}} = f\left(\frac{r}{R}, \frac{\alpha t}{R^2},
    \frac{hL}{k}\right)$$
\end{enumerate}
The solution to this requires a Bessel function series!!  No simple solution we
can fit to, so qualitative.

Now can graph $\pi_T$ vs. $\pi_r$ for various $\pi_t$, different graphs for
different $\pi_h$.  Large ($>100$) reverts to the constant temperature boundary
condition $T=T_{fl}$.

Had to end there, continuing after the Math Quiz on Wednesday...
\newpage


\section{October 1, 2003: Math Quiz, Graphs Wrapup, Newtonian Cooling}

Mechanics:
\begin{itemize}
\item Zeiss Materials Microscopy Truck scheduled tomorrow: cancelled!
\end{itemize}

\noindent Muddy stuff:
\begin{itemize}
\item Mass transfer: diffusion/reaction-limited.  Heat transfer:
  conduction/convection-limited.  Mass transfer can also be convection-limited
  if we replace reaction constant $k$ with mass transfer coefficient $h_D$.
\end{itemize}

\paragraph{Wrapup of The Graphs}

Now can graph $\pi_T$ vs. $\pi_r$ for various $\pi_t$, different graphs for
different $\pi_h$.  Large ($>100$) reverts to the constant temperature boundary
condition $T=T_{fl}$, small ($<0.1$) we'll get to in a moment, intermediate
Biot number graphs.

\paragraph{Newtonial cooling}
Small Biot number ($<0.1$): temperture is roughly uniform.  Let's say it {\em
  is} uniform.  Then we just have $T(t)$, $\pi_T(\pi_t,\pi_h)$.  Cool.

Balance over the entire object: accumulation = -out.
$$V \rho c_p \frac{dT}{dt} = -A q_r = A h (T-T_{fl})$$
Rearrange:
$$\frac{dT}{T-T_{fl}} = -\frac{A h}{V\rho c_p} dt$$
Integrate, with initial condition $T_i$ at $t=0$:
$$\ln(T-T_{fl}) - \ln(T_i-T_{fl}) = -\frac{A h t}{V\rho c_p}$$
$$\frac{T-T_{fl}}{T_i-T_{fl}} = \exp\left(-\frac{A h t}{V\rho c_p}\right)$$
Plug in $V/A$:
\begin{itemize}
\item Sphere: $R/3$
\item Cylinder: $R/2$
\item Plate: $``R'' = L/2$
\end{itemize}
Had to end there...
\newpage


\section{October 3, 2003: Moving on...}

Mechanics:
\begin{itemize}
\item Test 1 next Friday 2-143; handout, answer any questions.
\item Regular office hours; zephyrable (instance) most of next Tuesday.
\item PS4 due next Monday 10/6, correction: \#2a in BTU/hr not kW.  Corrected
  version on Stellar.
\item PS2\#3c solution error: ``at $t=1$ second, $x=9.6\times10^{-5}$cm, or
  just under one micron.  At $t=4$ seconds, $x=1.92\times10^{-4}$cm, just under
  two microns.'' (was $\times10^{-5}$...)  Corrected version on Stellar.
\item 3B Symposium Wednesday November 5.
\end{itemize}

\noindent Muddy from last time:
\begin{itemize}
\item What's this equation $V\rho c_p\partial T/\partial t = -Aq_r$?  We've had
  that before, it looked like $V\partial H/\partial t = Aq_x|_x -
  Aq_x|_{x+\Delta x} + V\dot{q}$.  I just skipped a step and went straight to
  accum=$V\rho c_p\partial T/\partial t$.  Sorry about that.
\item What's this bit about applying to different shapes?  We left everything
  general, with volume and area, so whether a sphere, rod, plate, or crumpled
  up piece of paper, it just works.
\end{itemize}

The book takes a different approach to the graphs in Appendix F: $\pi_T$ vs.
$\pi_t$ for various $\pi_h$, graphs at different $\pi_r$.  Useful for
temperature histories like PS4\#3 (but skip past the early graphs...), and also
for TTT diagrams, like our metal spray.
$$Y=\frac{T_\infty-T}{T_\infty-T_0} = f\left(X=\frac{\alpha t}{x_1^2},
  n=\frac{x}{x_1}, m=\frac{k}{hx_1}\right)$$

\paragraph{Wrapup Newtonian cooling}

Last time we did $\rm accum = --out$ for the whole shape, got to:
$$\frac{T-T_{fl}}{T_i-T_{fl}} = \exp\left(-\frac{A h t}{V\rho c_p}\right)$$
First, examine terms, timescale, larger/smaller $h$, rho $c_p$, $V/A$.  Plug in
$V/A$:
\begin{itemize}
\item Sphere: $R/3$
\item Cylinder: $R/2$
\item Plate: $``R'' = L/2$
\item Other shapes: varies...
\end{itemize}
Can instead define alternate Biot and Fourier numbers: Bi$'=\frac{hV}{kA}$,
Fo$'=\frac{\alpha A^2}{V^2}t$, then:
$$\frac{T-T_{fl}}{T_i-T_{fl}} = \exp\left(-\frac{hV}{kA}\frac{kA^2}{\rho
    c_pV^2}t\right) = \exp\left(\rm -Bi'Fo'\right)$$

So, all set for PS4, test1?

\paragraph{Thermal conductivity}

Diffusion is straightforward: atoms move, right?  Well, not quite: gases in
straight lines, liquid atoms move in chains, vacanices, interstitials,
dislocations, etc.  For heat, various mechanisms:
\begin{itemize}
\item Collisions
\item Phonons
\item Photons---radiation, which is spontaneous emission from hot body
\item Electrons
\end{itemize}
On electrons, Wiedmann-Franz law:
$$k_{el} = L \sigma_{el} T,
L=\frac{\pi}{3} \left(k_B/e\right)^2 =
2.45\times10^{-8}\frac{\rm W ohm}{K^2}$$
where $e$=electron charge.

Metals: $\sigma_{el}$ goes down with temperature.  What about electrons is
semiconductors?

Liquids: water .615 20-100$^\circ$C, O$_2$ $3.4\times10^{-4}$, H$_2$
$1.77\times10^{-3}$ (both 300K)

Influence of porosity and humidity/water absorption.  Gases are very bad
conductors, water not quite as bad but has very high specific heat!  (PS4 \#1d,
water has four times $c_p$ of aluminum which is highest there.)

Typical conductivity values: 0.1 to 300 $\rm\frac{W}{m\cdot K}$.
Porous$\rightarrow$less, metals high, gases {\em really} small!

Note: at conference, diamond-aluminum composite for microelectronics, 45 vol\%
diamond but isotropic conductivity of 550 W/mK!  Nearly twice copper,
squeeze-castable into heat sink parts.  Q: why no diamond-iron composite?
\newpage


\section{October 6, 2003: Finite Differences}

Mechanics:
\begin{itemize}
\item Test 1 next Friday 2-143; handout, answer any questions.
\item Regular office hours; zephyrable (instance) tomorrow 9-12, 1-5.
\item Albert review session Thursday 7 PM in 8-302 (next door to recitation).
\end{itemize}

\noindent Muddy from last time:
\begin{itemize}
\item Why $\pi_T=\exp(-\pi_h'\pi_t')$ with no $\pi_r$?  Because at low Biot
  number, $T$ is uniform, not a function of $r$ or $\pi_r$.
\end{itemize}

\paragraph{Finite differences}
Very often no analytical solution to a system.  (Or if there is one, it's
impossibly complex.)  So, use a computer, make some approximations.
\begin{itemize}
\item Discretize space: calculate temperature at a finite number of points on a
  grid (here 1-D).  Choose $x_i$, calculate $T_i$.  For simplicity, we'll
  choose evenly-spaced points, so $x_{i+1}-x_i=\Delta x$.
\item Discretize time: calcluate temperature at a finite number of
  ``timesteps'' at times $t_n$, so with both, we have $T_{i,n}$.  For
  simplicity, $\Delta t$ uniform.
\item Make some approximations about derivatives:
  $$\left.\frac{\partial T}{\partial t}\right|_{x_i, t_{n+1/2}} \simeq
  \frac{T_{i,n+1} - T_{i,n}}{\Delta t}$$
  $$\left.\frac{\partial T}{\partial x}\right|_{x_{i+1/2}, t_n} \simeq
  \frac{T_{i+1}-T_i}{\Delta x}$$
  $$\left.\frac{\partial^2 T}{\partial x^2}\right|_{x_i, t_n} \simeq
  \frac{\left.\frac{\partial T}{\partial x}\right|_{x_{i+1/2}, t_n} -
    \left.\frac{\partial T}{\partial x}\right|_{x_{i-1/2}, t_n}}
  {\Delta x} \simeq
  \frac{T_{i-1,n} - 2T_{i,n} + T_{i+1,n}}{(\Delta x)^2}$$
\end{itemize}

So, let's look at the energy equation, and substitute approximations:
$$\frac{\partial T}{\partial x} = \alpha\frac{\partial^2 T}{\partial x^2} +
\frac{\dot{q}}{\rho c_p}$$
$$\frac{T_{i,n+1} - T_{i,n}}{\Delta t} =
\alpha \frac{T_{i-1,n} - 2T_{i,n} + T_{i+1,n}}{(\Delta x)^2} +
\frac{\dot{q}}{\rho c_p}$$
$$T_{i,n+1} = T_{i,n} + \Delta t \left[
  \frac{T_{i-1,n} - 2T_{i,n} + T_{i+1,n}}{(\Delta x)^2} +
  \frac{\dot{q}}{\rho c_p}\right] =
T_{i,n} + {\rm Fo}_M (T_{i-1,n} - 2T_{i,n} + T_{i+1,n} +
\frac{\Delta t}{\rho c_p}\dot{q}$$
This is the ``forward Euler'' algorithm, a.k.a. ``explicit'' time stepping.
Nice, efficient, easy to put in a spreadsheet.  Problems: inaccurate because
time and space derivatives not co-located, also unstable.  Inaccuracy later.

Demonstrate instability for ${\rm Fo}_M > \frac{1}{2}$:
$$T_{i,n+1} = T_{i,n} (1-2{\rm Fo}_M) +
2 {\rm Fo}_M \frac{T_{i-1,n} + T_{i+1,n}}{2} +
\frac{\Delta t}{\rho c_p}\dot{q}$$
So, it's like a weighted average between $T_{i,n}$ and the average of the two
(show graphically).  When ${\rm Fo}_M>\frac{1}{2}$, the $T_{i,n}$ part is
negative, so we shoot past it!  So, the criterion is that it must be
$\leq\frac{1}{2}$, larger timestep means less work, so use $\frac{1}{2}$.

Exercise: cut length step in half, for same total time, how many more
timesteps?  How much more computational work?  Spreadsheet area...

To get rid of this instability, we have the ``backward Euler'' algorithm,
a.k.a. ``fully implicit'' time stepping.
$$\frac{T_{i,n+1} - T_{i,n}}{\Delta t} =
\alpha \frac{T_{i-1,n+1} - 2T_{i,n+1} + T_{i+1,n+1}}{(\Delta x)^2} +
\frac{\dot{q}}{\rho c_p}$$
Cool!  But, requires simultaneous equation solution for the next timestep.  But
it is unconditionally stable: infinite timestep means we solve the steady-state
problem.

Solving the simultaneous equations:
$$-{\rm Fo}_M T_{i-1,n+1} + (1+2{\rm Fo}_M) T_{i,n+1} - {\rm Fo}_M T_{i+1,n+1}
= T_{i,n} + \frac{\dot{q}\Delta t}{\rho c_p}$$
$$\begin{array}{cccccccc}T_{0,n+1} & & & & & & & =\\
  -{\rm Fo}_MT_{0,n+1}&+&(1+2{\rm Fo}_M)T_{1,n+1}&+&-{\rm Fo}_MT_{2,n+1} &&&=\\
  && -{\rm Fo}_MT_{1,n+1}&+&(1+2{\rm Fo}_M)T_{2,n+1}&+&-{\rm Fo}_MT_{3,n+1}&=\\
  & & & & & & T_{4,n+1} &= \end{array}
\begin{array}{c} T_{0,BC} \\ T_{1,n}+\frac{\dot{q}_1\Delta t}{\rho c_p} \\
  T_{2,n}+\frac{\dot{q}_1\Delta t}{\rho c_p} \\ T_{3,BC}\end{array}$$
$$\left(\begin{array}{cccc}
    1 & & & \\ -{\rm Fo}_M& (1+2{\rm Fo}_M)& -{\rm Fo}_M& \\
    & -{\rm Fo}_M& (1+2{\rm Fo}_M)& -{\rm Fo}_M \\ & & & 1 \end{array}\right)
\left(\begin{array}{c} T_0 \\ T_1 \\ T_2 \\ T_3 \end{array}\right) =
\left(\begin{array}{c} T_{0,BC} \\ T_{1,n}+\frac{\dot{q}_1\Delta t}{\rho c_p}\\
  T_{2,n}+\frac{\dot{q}_1\Delta t}{\rho c_p} \\ T_{3,BC}\end{array}\right)$$
Now just use 18.06 matrix techniques: Gaussian elimination, LU decomposition,
eigenvalues, etc.
\newpage


\section{October 8, 2003: More Finite Differences}

Mechanics:
\begin{itemize}
\item Graded math quizzes back.  Avg 96$\frac{1}{4}$, std dev 4.85, 11 100s!
  Warm-up, next is the race.
\item Test 1 next Friday 2-143; handout, answer any questions.
\item Review session tomorrow 7 PM 8-302.
\item I can have office hours Monday, but would much rather be available
  Wednesday 2:30-3:30.
\item PS4 solution error: Newtonian cooling eq has just one t!  Also in 10/1
  and 10/3 lecture notes; corrections in Stellar and on Athena respectively.
\end{itemize}

\noindent Muddy from last time:
\begin{itemize}
\item Top and bottom rows in RHS last time were wrong, should have been
  $T_{0,BC}$ and $T_{3,BC}$.  Sorry...
\item ``How... theoretically interesting.

  ``You said you were going to start each lecture with a `motivating
  factor'---a real example to tie things to so the lecture isn't just so many
  symbols and numbers---where was today's motivating factor?

  ``I'm hoping to at least be able to see a problem being solved where all this
  is useful.  Otherwise, this makes no sense, sorry.''

  Okay, two examples today on the laptop.
\end{itemize}

Encourage to think of test as checkpoint, first evaluation (except Math quiz,
but that doesn't count).  And remember, you can make it up in the second
sitting.  Will not be straightforward, won't see PS problems, but will apply
same techniques to new situations.  You will have to think, but you can all do
that, that's why you're here.

\paragraph{Finite differences}

Last time: Forward Euler/explicit, and Backward Euler/implicit timestepping.
But both of these are integrating in time using the value at previous or next
timestep.  Like rectangles in numerical integration.  Graphically show error as
proportional to $\Delta t$.  To increase accuracy, use trapezoids, right?  Then
error is proportional to $(\Delta t)^2$.  That works like:
$$\frac{T_{i,n+1} - T_{i,n}}{\Delta t} =
\alpha \frac{T_{i-1,n} - 2T_{i,n} + T_{i+1,n} +
T_{i-1,n+1} - 2T_{i,n+1} + T_{i+1,n+1}}{2(\Delta x)^2} +
\frac{\dot{q}_{i,n} + \dot{q}_{i,n+1}}{2\rho c_p}$$
This is ``semi-implicit'', or ``Crank-Nicholson'' time integration, also need
to solve simultaneous equations.

Error goes as $\Delta t^2$ for Crank-Nicholson, $\Delta t$ for
explicit/implicit (forward/backward Euler), like trapezoid rule vs. simple
rectangle Riemann integration.

2-D: two second derivatives in $x$ and $y$, $T_{i,j,n}$ at $x_i,y_j,t_n$;
explicit form:
$$\frac{T_{i,n+1} - T_{i,n}}{\Delta t} =
\alpha \left(\frac{T_{i-1,j,n} - 2T_{i,j,n} + T_{i+1,j,n}}{(\Delta x)^2} +
\frac{T_{i,j-1,n} - 2T_{i,j,n} + T_{i,j+1,n}}{(\Delta y)^2}\right)$$
With $\Delta x=\Delta y$, Fo$_M=\alpha\Delta t/(\Delta x)^2$, we have:
$$T_{i,n+1} = (1-4{\rm Fo}_M)T_{i,j} +
4{\rm Fo}_M \frac{T_{i-1,j,n} + T_{i+1,j,n} + T_{i,j-1,n} + T_{i,j+1,n}}{4}$$
So the stability criterion is:
$${\rm Fo}_m\leq\frac{1}{4}\Rightarrow\Delta t\leq\frac{\Delta x^2}{4\alpha}.$$
In 3-D:
$${\rm Fo}_m\leq\frac{1}{6}\Rightarrow\Delta t\leq\frac{\Delta x^2}{6\alpha}.$$

Laptop spreadsheet demos: iron conduction ps5.gnumeric, freezing water
lecture1008.gnumeric.
\newpage


\section{October 15, 2003: Moving Body}

Mechanics:
\begin{itemize}
\item Test 1 a bit too long, which is average for me, but not good.  Will aim
  for shorter next time.
\item Test typo: m, n switch in equation sheet graph descr.
\item Ambiguous wording in 4b: clarified on board, but take any self-consistent
  answer.
\item Misleading wording in one test question!  2d: strike ``---that is, what's
  a more realistic shape for this region''.
\item The graph: perhaps not big enough.  Good news: after initial behavior,
  $\pi_t\propto\exp(-t)$ (works for Newt cooling always, $n=1$ term of
  Fourier).  So on log-linear graph, straight lines, can extrapolate.
\item New version on Stellar (minus the graph), will be used in retake.  Sorry!
\item Office hours: Today 2:30-3:30.
\item SOFCs and energy today 12:15 Marlar Lounge (37-252), Ashley Predith, MIT.
\item Magnetic nanodots Monday 3-4 Chipman, Igor Roshchin, UCSD.
\end{itemize}

\paragraph{Moving body}
Example: VAR of titanium alloys, nickel superalloys.  Start, during operation.
Nickel: 6-8 kA, 17$\rightarrow$20''; Ti around 30 kA, 30$\rightarrow$36''.

Competition: thermal diffusion up vs. drive down.  Suggest steady-state, sketch
$T$ vs. $z$.

Temperatures in ingot real complicated, flow, etc.  But can analyze electrode
now.  Question: how much of the electrode is heated?  What's the temperature
profile?

Choose frame of reference of melt interface on the bottom of the electrode.
Solid is moving with respect to frame of reference.  Now conductive and
convective heat fluxes: $\vec{q} + \rho c_p T \vec{u}$ (not really, but this is
valid for the difference).

In and out have motion component!  Important thing: in-out.
${\rm in}=u_x \rho c_p T$, out too.  Result when goes to zero:
$${\rm in - out} = -\frac{\partial}{\partial x}(q_x + \rho c_p T u_x)$$
This example: $u_x$, $\rho$, $c_p$ are all constant, so we end up with:
$$\rho c_p \frac{\partial T}{\partial t} = k \frac{\partial^2 T}{\partial x^2}
- \rho c_p u_x \frac{\partial T}{\partial x} + \dot{q}$$
Rearrange slightly for constant $\rho c_p u_x$, substitute
$q_x = -k \partial T/\partial x$:
$$\rho c_p \left(\frac{\partial T}{\partial t} +
  u_x \frac{\partial T}{\partial x}\right) =
k \frac{\partial^2 T}{\partial x^2} + \dot{q}$$
Divide by $\rho c_p$:
$$\frac{\partial T}{\partial t} + u_x \frac{\partial T}{\partial x} =
\alpha\frac{\partial^2 T}{\partial x^2} + \frac{\dot{q}}{\rho c_p}$$
Discuss terms: why proportional to $\partial T/\partial x$, competing effects
of positive $\partial^2T/\partial x^2$ and negative $-\partial T/\partial x$.
Graphical explanation.

What introductory math concept does this remind us of?  The substantial
derivative!  Rewrite:
$$\frac{DT}{Dt} = \alpha\frac{\partial^2 T}{\partial x^2} +
\frac{\dot{q}}{\rho c_p}$$
Note that's the time derivative in the frame of reference of the moving solid.
Cool!

Steady-state, no generation:
$$\alpha\frac{\partial^2 T}{\partial x^2} - u_x \frac{\partial T}{\partial x} =
0$$
Simple solution using the characteristic
polynomial, $R=0, u_x/\alpha$.  Result:
$$T = A + B \exp\left(\frac{u_x x}{\alpha}\right)$$
Fit to boundary conditions: $x=0\Rightarrow T=T_M$, $x=\infty\Rightarrow T=T_i$
so use erf-style:
$$\frac{T-T_i}{T_M-T_i} = \exp\left(\frac{u_x x}{\alpha}\right)$$
Lengthscale=$\alpha/u_x$.  Graph, noting that $u_x$ is negative.  Titanium
$\rm\alpha=0.1\frac{cm^2}{s}$, $u_x\sim5\frac{\rm cm}{\rm min} =
\frac{1}{12}\frac{\rm cm}{s}$, so $\alpha/u_x = 1.2 cm$, about 1/2 inch.  So
only the bottom few centimeters are heated at all, even at this low velocity!

Heat flux into the bottom:
$$q_x=-k\frac{\partial T}{\partial x} =
-k(T_m-T_i)\frac{u_x}{\alpha}\exp\left(\frac{u_x x}{\alpha}\right) =
-\rho c_p u_x (T_m-T_i)$$
Note $\rho c_p(T_m-T_i)$ is the enthalpy per unit volume to heat metal to its
melting point.  Mult by $u_x$ for enthalpy per unit area to heat metal coming
at a rate of $u_x$, which is a cool result.

Next time: heat flux required to melt...
\newpage


\section{October 17, 2003: Phase Change}

Ask Andy re retake...

\noindent Mechanics:
\begin{itemize}
\item New version on Stellar (minus the graph), will be used in retake.  Sorry!
\item Test stats first time around: 62-86 within a std dev.  But significant clustering, low 80s and low 60s.
  \begin{center}
    \begin{tabular}{l|ccc|}
      Problem & Mean & Std. Dev. & Max \\ \hline
      1. & 5 & 0 & 5 \\
      2. & 29.41 & 4.02 & 35 \\
      3. & 22.34 & 7.21 & 34 \\
      4. & 17.21 & 3.92 & 25 \\ \hline
      Total & 73.97 & 12.34 & 94 \\ \hline
    \end{tabular}
  \end{center}
\item Yet another error: diffusion equation missing $D$!
\item Magnetic nanodots Monday 3-4 Chipman, Igor Roshchin, UCSD.  (Also mention
  interesting talk on Wednesday.)
\item GLOAT ABOUT YANKEES!
\end{itemize}

\noindent Muddy from last time:
\begin{itemize}
\item Why $H=\rho c_pT$ in quotes?  Well, $\Delta H=\rho c_p\Delta T$, for
  temperature change only.  But $H=\rho c_pT$ is not true, show by graph.
\item What is $H\cdot u_x$?  That's the convective flux, the transfer of heat
  due to motion of a substance.
\item Frames of reference: $DT/Dt$ is the time rate of change for a particle
  moving with the solid (or later, fluid); $\partial T/\partial t$ is the time
  rate of change at a fixed point (in a certain frame).
\item What's the significance of $q_x = -\rho c_pu_x (T_m-T_i)$?  Well, $\rho
  c_p\Delta T$ is the heat per unit volume.  How much heat to raise Ti from the
  initial temp to the melting point.  Times $u_x$ gives the heat/area/time, the
  flux required to raise titanium coming in at that speed.  Think of $u_x$ as
  meters/second, or as $\rm m^3/m^2\cdot s$.
\end{itemize}

\paragraph{Phase change}

Another important concept: heat generated/lost at interface due to phase
change.  If extend the graph beyond $x=0$ into liquid, more flux from liquid
into interface than from interface into solid.  How much more?
$$q_{x,l} - q_{x,s} = -\rho \Delta H_M u_x$$
Example: candle, MIT undergrad; ``Build a man a fire...''

Model of casting limited by conduction through metal, per Albert's recitation;
graphical representation on board.  Analogy to diffusion phase change (silicon
oxidation): $H$ is like $C$, $T$ is like chemical potential $\mu$.  Fast growth
means proportional to undercooling (ask Albert), like reaction-limitation in
oxidation.

\paragraph{Evaporation/condensation}

Also for evaporation, heat flux from gas, plasma, radiation incl. laser
(below), electron beam, etc.  Condensation releases heat at a similar rate.

Evaporation into gas: boundary layer, $J=h_D(C_s-C_{bulk})$.

Evaporation rate into a vacuum: Langmuir equation
$$J = \frac{p_v}{\sqrt{2\pi MRT}}$$
Here the units should work, go through.

Equilibrium pure vapor pressure: Clausius-Clapeyron equation, one form:
$$\log_{10}p_v = -\frac{A}{T} + B + C\log_{10}T (+ DT)$$
Units: torr, conversion factor.  If not pure, then mult by activity.  Either
way, multiply material flux $J$ by $\Delta H_{vap}$ for heat flux influence.
\newpage


\section{October 20, 2003: Radiation}

Mechanics:
\begin{itemize}
\item Test stats first time around: 62-86 within a std dev.  But significant
  clustering, low 80s and low 60s.
  \begin{center}
    \begin{tabular}{l|ccc|}
      Problem & Mean before & Std. Dev. & Max \\
      \hline
      1. & 5 & 0 & 5 \\
      2. & 29.41 & 4.02 & 35 \\
      3. & 22.34 & 7.21 & 34 \\
      4. & 17.21 & 3.92 & 25 \\ \hline
      Total & 73.97 & 12.34 & 94 \\ \hline
    \end{tabular}
  \end{center}
\item Magnetic nanodots today 3-4 Chipman, Igor Roshchin, UCSD.
\end{itemize}

Evaporation cont'd: When to use dense gas, line-of-sight vacuum approxes?  Mean
free path $\lambda$:
$$\lambda=\frac{1}{\sqrt{2}\pi\sigma^2n}$$
$\sigma$ is collision diameter, $n$ is number of molecules per unit volume,
$P/k_BT$ (sketch molecules).  Important thing is the Knudsen number,
$\lambda/L$, given by:
$${\rm Kn} = \frac{\lambda}{L} = \frac{kT}{\sqrt{2}\pi\sigma^2PL}$$
so in $P-T$ space, lines deliniate ``line-of-sight'' r\'{e}gime (Kn$>$1),
``continuum'' r\'{e}gime (Kn$<$0.01).

\paragraph{Radiation!}

Def: spontaneous emission of photons from a hot body.  Emission, absorption,
reflection, transmission.  Cosine distribution: hand-waving skin depth
explanation.

Happens throughout a body, but surface emission follows a cosine distribution:
handwaving explanation of skin depth as a function of angle.

Concept: black body, absorbs all incident radiation, theoretical construct with
some practical application.  Also emits maximum possible radiation.  Handwaving
explanation: zero reflection at the interface.

Defs: $e$ is power emitted per unit area, $e_b$ is power emitted by black body
per unit area, $e_\lambda$ is power per unit wavelength per unit area,
$e_{b,\lambda}$ is power by black body per unit wavelength per unit area.

Emission spectrum of black body:
$$e_{b,\lambda} =
\frac{2\pi h c^2 \lambda^{-5}}{e^{\frac{ch}{k_B\lambda T}} - 1}$$
$h$ is Planck's constant, $c$ is light speed, $k_B$ Boltzmann's constant.
Graph for different $T$.

%NEXT TIME: put $\lambda_{max} T=2.9\times10^{-3}{\rm m\cdot K}$ here; at 1000K,
%2.9$\mu$m=2900 nm; sun at ~5800K is at 500 nm (yellow)---need to be pretty hot
%to peak in the visible spectrum.   (Also next time remove from next lecture.)

How to get $e_b$?  Integrate over all wavelengths.  Fortunately, it's quite
simple:
$$e_b = \int_0^\infty e_{b,\lambda} d\lambda = \sigma T^4$$
The physicists must have jumped for joy when they saw that one.  For our
purposes, it puts radiation within reach of engineers.  Okay, all done, never
have to see that first equation again.

Even better:
$$\rm\sigma = \frac{2\pi^5k_B^4}{15c^2h^3} = 5.67\times10^{-8}\frac{W}{m^2\cdot
  K^4}$$
Note: fourth-power dependence on temperature means this is {\bf MUCH} more
important at high temperature than low temperature.

New defs: emissivity $\epsilon_\lambda = e_\lambda/e_{b,\lambda}$, the fraction
of black body radiation which is emitted; absorptivity $\alpha_\lambda =
a_\lambda/a_{b,\lambda}$.  Cool result: $\epsilon_\lambda = \alpha_\lambda$,
always!  Material property.  Graph resulting emission spectrum.

Grey body approximation: $\epsilon=\alpha=\epsilon_\lambda=\alpha_\lambda=$
constant.  Makes life a lot simpler for us engineers.  Superpose grey spectra
on previous graph.

Resulting emission: $e = \epsilon\sigma T^4$.  Pretty cool.  Likewise average
absorptivity $\alpha$.

Real materials: $\epsilon=f(T)$, $\alpha=f({\rm incident\ spectrum})$.
Example: global warming, CO$_2$ absorbs in the infrared, admits sun in visible.
\newpage


\section{October 22, 2003: More Radiation}

Mechanics:
\begin{itemize}
\item CONGRATS TO ALBERT!
\item Test stats first time around: 62-86 within a std dev.  But significant
  clustering, low 80s and low 60s.
  \begin{center}
    \begin{tabular}{l|ccc|ccc|}
      Problem & Mean before & Std. Dev. & Max & Mean after & Std. dev & Max \\
      \hline
      1. & 5 & 0 & 5 & 5 & 0 & 5\\
      2. & 29.41 & 4.02 & 35 & 34.66 & 0.86 & 35 \\
      3. & 22.34 & 7.21 & 34 & 33.83 & 1.49 & 35 \\
      4. & 17.21 & 3.92 & 25 & 24.59 & 1.02 & 25 \\ \hline
      Total & 73.97 & 12.34 & 94 & 98.07 & 2.36 & 100 \\ \hline
    \end{tabular}
  \end{center}
  ``A'' I consider around 80/89, because of statistics.  Did well, 20\%; not so
  well, only 20\%.
\item Subra on bionano cell mechanics next Monday 4PM 10-250.  Recruiting...
\end{itemize}

\noindent Muddy from last time:
\begin{itemize}
\item Why is $\epsilon$ a function of $T$?  Semiconductor example: silicon has
  band gap, absorbs some near infrared and in visible and higher energy (lower
  wavelength), very little in far infrared.  So at low $T$, low $\epsilon$; at
  higher $T$ (up to melting point), higher $\epsilon$.  Note: can't be heated by
  IR heat lamp.  Also note: liquid silicon has zero band gap!
\end{itemize}

Peak wavelength:
$$\lambda_{max} T = 2.9\times10^{-3}{\rm m\cdot K}$$
1000K, 2.9$\mu$m=2900 nm; sun at ~5800K is at 500 nm (yellow)---need to be
pretty hot to peak in the visible spectrum.

Little table:
\begin{center}
  \begin{tabular}{l|c|c|}
                    & Wavelength & Total/average    \\ \hline
    BB Emission     & $e_{b,\lambda}$ & $e_b=\int_0^\infty e_{b,\lambda}d\lambda$ \\
    Actual emission & $e_\lambda$     & $e(=q)=\int_0^\infty e_\lambda d\lambda$ \\
    Emissivity      & $\epsilon_\lambda=e_\lambda/e_{b,\lambda}$ & $\epsilon(T)=e/e_b$ \\
    Absorptivity    & $\alpha_\lambda\equiv \epsilon_\lambda$ & $\alpha(incident)$ \\ \hline
  \end{tabular}
\end{center}

Fortunately $e_b$ is quite simple:
$$e_b = \int_0^\infty e_{b,\lambda} d\lambda = \sigma T^4,
\sigma = 5.67\times10^{-8}\frac{W}{m^2\cdot K^4}$$
Grey approximation means we stick an average $\epsilon$ in there.  Note:
fourth-power dependence on temperature means this is {\bf MUCH} more important
at high temperature than low temperature.

Averaged properties: $\epsilon=e/e_b$, $\alpha=a/incident$.  Note $\epsilon$
will vary with temperature, $\alpha$ depends on wavelength of incident light.

\paragraph{Radiation viewfactors}

So, the pointof radiative exchange: how much radiation emitted by surface 1
reaches surface 2?  Double integral:
$$Q_{12} = \int_{S1} \int_{S2} \frac{e_1 \cos\theta_1 \cos\theta_2}
{r^2} dA_2 dA_1$$
A really ugly thing!

Okay, but suppose $A$ is at a uniform temperature, $B$ also?  Then can pull out
$e_A$, $\alpha_B$; define $F_{12}$:
$$Q_{12} = e_1 \int_{S1} \int_{S2} \frac{\cos\theta_1 \cos\theta_2}
{r^2} dA_A dA_B$$
$$Q_{12} = e_1 A_1 F_{12}$$
$F_{12}$ is only a function of the shape, not the size; is dimensionless.

Viewfactor Algebra: two principles
$$A_1 F_{12} = A_2 F_{21}$$
Prove from equal temperature.
$$\sum_{i=1}{n} F_{ji} = 1$$
if they form an enclosure.  Simple thing.  With these two, can do complex
stuff.  Simple geom graphs on pp. 396--398.  Note: $F_{11}\neq 0$ if concave.

For coaxial disks of same radius, graph $F_{12}$ vs. $d/r$, values below.

Example: disk and cylinder section height d/4 to d/2 above, viewfactor for
disks d/4 is 0.6, for d/2 is 0.375.  Derive $F_{21}=0.225$ by enclosure
arguments; $F_{12}=\frac{A_2}{A_1}F_{21}=F_{21}$ by this argument.

\paragraph{Total exchange viewfactor: NOT COVERED THIS YEAR}

Reflection can be specular, diffuse.  Here discuss diffuse.  Suppose two grey
bodies forming an enclosure, diffuse reflection at same cosine distribution.
$$Q_{12, net} = e_{b1} \epsilon_1 A_1 F_{12} (1-\epsilon_2) A_2 F_{21}
(1-\epsilon_1) A_1 F_{12} {\rm etc.} - e_{b2} \epsilon_2 A_2 {\rm etc.}$$
Simplifies to:
$$Q_{12, net} = \frac{e_{b1} - e_{b2}}{\frac{1-\epsilon_1}{A_1\epsilon_1} +
  \frac{1}{A_1F_{12}} + \frac{1-\epsilon_2}{A_2\epsilon_2}}$$

Funny thing: like a sum of resistances.  Funnier stil: multiple surfaces
forming an enclosure $\Rightarrow$ resistance diagram!

New concept: zero-flux surface, well-insulated,
reflected+emitted$\simeq$incident.  In that case, no ``current'' through that
resistor, can get the total from surface 1 to 2 bypassing surface R.  Pretty
cool!

Total Exchange Viewfactor: $\bar{F}_{12}$, in this case
$$A_1 \bar{F}_{12} = A_1 F_{12} +
\frac{1}{\frac{1}{A_1 F_{1R}} + \frac{1}{A_2 F_{2R}}}$$
Substitute that in instead of $A_1F_{12}$ in $Q_{12,net}$ equation above.

Done with radiation, with heat transfer, on to fluids!


\chapter{Fluid Dynamics}

\section{October 24, 2003: Intro, Newtonian Fluids}

TODO: look up Wiedmann-Franz, falling film in new textbook.

Mechanics:
\begin{itemize}
\item PS5: Get the spreadsheet from the Stellar site (URL on PS was for PMMA
  properties).
\item $L$ expression may be off: calc'd to $7.9\times10^{-9}$, not
  $2.45\times10^{-8}$.  Maybe a missing $\pi$.
\item Subra on bionano cell mechanics next Monday 4PM 10-250.  Recruiting...
\item Next Tues: MPC Materials Day, on Biomed Mat'ls Apps.  Register:
  http://mpc-web.mit.edu/
\end{itemize}

\noindent Muddy from last time
\begin{itemize}
\item First ``rule'' in viewfactor algebra: $A_1F_{12}=A_2F_{21}$, doesn't it
  depend on $T$?  No, because $F_{ij}$ is based only on geometry.
\item $F_{12}$ for facing coaxial disks with radii $r_1$ and $r_2$ sample
  graph: $F_{12}$ is decreasing with $d/r_1$.
\end{itemize}

\paragraph{Fluid Dynamics!}

Brief introduction to rich topic, of which people spend lifetimes studying one
small part.  You will likely be confused at the end of this lecture, come to
``get it'' over the next two or three.

Categories: laminar, turbulent; tubes and channels; jets, wakes.  Compressible,
incompressible.

Outcomes: flow rates (define), drag force (integral of normal stress), mixing.
Later couple with diffusion and heat conduction for convective heat and mass
transfer.

Start: the 3.185 way.  Momentum field, ``momentum diffusion'' tensor as shear
stress.  Show this using units: momentum per unit area per unit time:
$$\rm\frac{kg\frac{m}{s}}{m^2\cdot s} = \frac{kg}{m\cdot s^2} = \frac{N}{m^2}$$

Two parallel plates, fluid between, zero and constant velocity.
$x$-momentum diffusing in $z$-direction, call it $\tau_{zx}$, one component of
2nd-rank tensor.  Some conservation of math:
$$\rm accumulation = in - out + generation$$
Talking about momentum per unit time, $\rm\frac{kg}{m\cdot s^2}$, locally
momentum per unit volume $\rho\vec{u}$.  Here suppose $u_x$ varies only in the
$z$-direction, no $\tau_{xx}$ or $\tau_{yx}$, no $u_y$ or $u_z$.  Three
conservation equations for three components of momentum vector, here look at
$x$-momentum:
$$V\cdot \frac{\partial(\rho u_x)}{\partial t} =
\left.A\cdot\tau_{zx}\right|_z - \left.A\cdot\tau_{zx}\right|_{z+\Delta z} +
V\cdot F_x$$
Do this balance on a thin layer between the plates:
$$W_xW_z\Delta z\frac{\partial(\rho u_x)}{\partial t} =
\left.W_xW_z\tau_{zx}\right|_z - \left.W_xW_z\tau_{zx}\right|_{z+\Delta z} +
W_xW_z\Delta z F_x$$
Cancel $W_xW_z$ and divide by $\Delta z$, let go to zero:
$$\frac{\partial(\rho u_x)}{\partial t} =
-\frac{\partial\tau_{zx}}{\partial z} + F_x$$

What's generation?  Body force per unit volume, like gravity.  Units: N/m$^3$
(like $\tau$ has $N/m^2$), e.g. $\rho g$.

What's the constitutive equation for $\tau_{zx}$?  Newtonian fluid,
proportional to velocity gradient:
$$\tau_{zx} = -\mu \left(\frac{\partial u_x}{\partial z} +
  \frac{\partial u_z}{\partial x}\right)$$
This defines viscosity $\mu$, which is the momentum diffusivity.  Units: $\rm
N\cdot s/m^2$ or $\rm kg/m\cdot s$, Poiseuille.  CGS units: $\rm g/cm\cdot s$,
Poise = 0.1 Poiseuille.  Water: .01 Poise = .001 Poiseuille.

So, sub constitutive equation in the conservation equation, with $u_y=0$:
$$\frac{\partial(\rho u_x)}{\partial t} =
-\frac{\partial}{\partial z}\left(-\mu\frac{\partial u_x}{\partial z}\right)
 + F_x$$
With constant $\rho$ and $\mu$:
$$\rho\frac{\partial u_x}{\partial t} = \mu\frac{\partial^2 u_x}{\partial z^2}
+ F_x$$
It's a diffusion equation!

So at steady state, with a bottom plate at rest and a top plate in motion in
the $x$-direction at velocity $U$, we have: a linear profile, $u_x=Az+B$.  

Aaliyah tribute: innovative complex beats, rhythmic singing, great performing.
Started at an early age, passing at age 21? 22? a couple of years ago in plane
crash was a major music tragedy.
\newpage


\section{October 27: Simple Newtonian Flows}

Mechanics:
\begin{itemize}
\item Aaliyah: okay, you might have heard of her, but didn't expect from a
  Prof.  MTV...
\item Wiedmann-Franz: new text doesn't offer any help (p. 204, no constant).
\item Subra on bionano cell mechanics next Monday 4PM 10-250.  Recruiting...
\item Tomorrow: MPC Materials Day, on Biomed Mat'ls Apps.  Too late to register
  though :-(
\end{itemize}

\noindent Muddy from last time:
\begin{itemize}
\item Is the velocity in the $x$-direction, or the $z$-direction?
  $x$-direction, but it is quite confusing, $\tau_{zx}$ is flux of $x$-momentum
  in $z$-direction.  Momentum is a vector, so we have {\em three} conservation
  equations: conservation of $x$-momentum, $y$-momentum, $z$-momentum.  This
  vector field thing is a bit tricky, especially the vector gradient.
\item I left out: a Newtonian fluid exhibits linear stress-strain rate
  behavior, proportional.  Lots of nonlinear fluids, non-Newtonian; we'll get
  to later.
\end{itemize}

Intro: may be confused after last time.  This time do a couple more examples
with confined flow, including one cylindrical, hopefully clear some things up.

From last time: parallel plates, governing equation
$$\rho\frac{\partial u_x}{\partial t} = \mu\frac{\partial^2 u_x}{\partial z^2}
+ F_x$$
$$\frac{\partial u_x}{\partial t} = \nu\frac{\partial^2 u_x}{\partial z^2} +
\frac{F_x}{\rho}$$
We have the diffusion equation!  $\nu$ is the momentum diffusivity, like the
thermal diffusivity $k/\rho c_p$ before it.  Note: units of momentum
diffusivity $\nu=\mu/\rho$: $\rm\frac{kg/m\cdot s}{kg/m^3} = m^2/s$!
Kinematic ($\nu$), dynamic ($\mu$) viscosities.

Note on graphics: velocities with arrows, flipping the graphs sideways to match
orientation of the problem.

Cases:
\begin{itemize}
\item Steady-state, no generation, bottom velocity zero, top $U$:
  $$u_x = \frac{U}{L}z,\ \tau_{zx} = -\mu\frac{U}{L}$$
  Shear stress:
  $$\tau_{zx} = -\mu\left(\frac{\partial u_x}{\partial z} + \frac{\partial
      u_z}{\partial x}\right) = -\mu\frac{U}{z}$$
  Thus the drag force is this times the area of the plate.
\item Unsteady, no generation, different velocities from $t=0$.
  $$u_x = U{\rm erfc}\left(\frac{L-z}{2\sqrt{\nu t}}\right),
  \tau_{zx} = -\frac{1}{2\sqrt{\nu t}}\frac{2}{\sqrt{\pi}}
  \exp\left(-\frac{(L-z)^2}{4\nu t}\right)$$
\item New: steady-state, generation, like book's falling film in problem 4.15
  of W$^3$R, with $\theta$ the inclination angle off-normal so $g_x =
  g\sin\theta$, $z$ is the distance from the plane.  The steady-state equation
  reduces to:
  %NEXT TIME: z = distance from free surface (like last year).
  $$0 = \mu\frac{\partial^2 u_x}{\partial z^2} + F_x$$
  $$u_x = -\frac{F_x z^2}{2\mu} + Az + B$$
  BCs: zero velocity at bottom plate at $z=0$, free surface with zero shear
  stress at $z=L$, $F_x=\rho g_x = \rho g\sin\theta$, result: B=0, get
  $$u_x = \frac{g\sin\theta}{2\nu}(2Lz-z^2)$$
  Shear stress:
  $$\tau_{zx} = -\mu\left(\frac{\partial u_x}{\partial z} + \frac{\partial
      u_z}{\partial x}\right) = \rho g\sin\theta (L-z)$$
  This is the weight of the fluid per unit area on top of this layer!
\end{itemize}
Shear stress as the mechanism of momentum flux, each layer pushes on the layer
next to it.  Think of it as momentum diffusion, not stress, and you'll get the
sign right.

Flow rate: $Q$, volume per unit time through a surface.  If width of the
falling film is $W$, then flow rate is:
$$Q=\int_S \vec{u}\cdot\hat{n}dA = \int_{z=0}^L u_x W dz =
\frac{Wg\sin\theta}{\nu}\left[\frac{Lz^2}{2}-\frac{z^3}{6}\right] =
\frac{Wg\sin\theta}{\nu}\frac{L^3}{3}$$
Average velocity is $Q/A$, in this case $Q/LW$:
$$u_{av} = \frac{Q}{LW} = \frac{g\sin\theta L^2}{3\nu};$$
$$u_{max} = \left.u_x\right|_{z=L} = \frac{g\sin\theta L^2}{2\nu}.$$
So average velocity is 2/3 of maximum for falling film, channel flow, etc.
\newpage


\section{October 29: 1-D Laminar Newtonian Wrapup, Summary}

Mechanics:
\begin{itemize}
\item Next Weds 11/5: 3B Symposium!
\end{itemize}

\noindent Muddy from last time:
\begin{itemize}
\item Erf solution: why
  $$u_x=U{\rm erfc}\left(\frac{L-z}{2\sqrt{\nu t}}\right)?$$
  As long as it's semi-infinite, it's an erf/erfc solution (uniform initial
  condition, constant velocity boundary condition), so this works for $t\leq
  L^2/16\nu$.  For erfs, they can start at 0, or somewhere else (zinc diffusion
  couple), and go ``forward'' or ``backward''.  Graph the normal way.  Also:
  $$u_x = U\left[1+{\rm erf}\left(\frac{z-L}{2\sqrt{\nu t}}\right)\right]$$
\item Weight of fluid: consider layer of fluid from $z$ to $L$, it has force in
  $x$- and $z$-directions $F_g=V\rho g\sin\theta$ and $-\cos\theta$
  respectively.  In $x$-direction, shear goes the other way: $F_\tau =
  -\tau_{zx}A$.  So force balance for steady-state (no acceleration):
  $$V\rho g\sin\theta - A\tau_{zx} = 0 \Rightarrow \tau_{zx}= \frac{V}{A} \rho
  g\sin\theta = (L-z) \rho g\sin\theta.$$
  In $z$-direction, this is balanced by pressure:
  $$P = P_{atm} + \rho g\cos\theta(L-z).$$
\end{itemize}

Nice segue into pressure-driven flows.  Suppose fluid in a cylinder, a pipe for
example of length $L$ and radius $R$, $P_1$ on one end, $P_2$ on other.  Net
force: $(P_1-P_2)A_{xs}$, force per unit volume is $(P_1-P_2)V/A_{xs} =
(P_1-P_2)/L$.  Can shrink to shorter length, at a given point, force per unit
volume is $\Delta P/\Delta z\rightarrow\partial P/\partial z$.  This is the
pressure generation term.

So, flow in tube: uniform generation throughout $(P_1-P_2)/L$ (prove next
week), diffusion out to $r=R$ where velocity is zero.  Could do momentum
balance, but is same as diffusion or heat conduction, laminar Newtonian result:
$$\rho\frac{\partial u_z}{\partial t} = \frac{1}{r}\frac{\partial}{\partial r}
\left(r\mu\frac{\partial u_z}{\partial r}\right) + \rho g_z - \frac{\partial
  P}{\partial z}.$$
Here looking at steady-state, horizontal pipe, uniform generation means:
$$u_z = -\frac{F_z r^2}{4\mu} + A\ln r + B = -\frac{P_1-P_2}{4\mu L}r^2 + A\ln
r + B.$$
Like reaction-diffusion in problem set 2 (PVC rod): non-infinite velocity at
$r=0$ means $A=0$ (also symmetric), zero velocity at $r=R$ means:
$$u_z = \frac{P_1-P_2}{4\mu L}(R^2-r^2).$$
What's the flow rate?
$$Q = \int_0^R u_z 2\pi r dr = \int_0^R \frac{P_1-P_2}{4\mu L}(R^2-r^2) 2\pi r
dr$$
$$Q = \frac{\pi(P_1-P_2)}{2\mu L} \left[\frac{R^2r^2}{2} -
  \frac{r^4}{4}\right]_0^R = \frac{\pi(P_1-P_2)R^4}{8\mu L}.$$
H\"{a}gen-Poisseuille equation, note 4th-power relation is extremely strong!
3/4'' vs. 1/2'' pipe...

\paragraph{Summary} Summary of the three phenomena thus far:

\begin{center}
  \begin{tabular}[h]{l|c|c|c|}
                      & Diffusion & Heat conduction & Fluid flow \\ \hline
    What's conserved? & Moles of each species & Joules of energy & momentum \\
    Local density  & $C$ & $\rho c_p T$ & $\rho\vec{u}$ \\
    Units of flux & $\rm\frac{mol}{m^2\cdot s}$ & $\rm\frac{W}{m^2}$ &
      $\rm\frac{kg\frac{m}{s}}{m^2\cdot s} = \frac{N}{m^2}$ \\
    Conservation equation$^*$ &
      $\frac{\partial C}{\partial t} = -\nabla\cdot\vec{J} + G$ &
      $\rho c_p\frac{\partial T}{\partial t} = -\nabla\cdot\vec{q} + \dot{q}$ &
      $\frac{\partial(\rho\vec{u})}{\partial t} = -\nabla P - \nabla\cdot\tau +
      \vec{F}$ \\
    Constitutive equation & $\vec{J} = -D\nabla C$ & $\vec{q} = -k\nabla T$ &
      $\tau = -\eta\left[\nabla\vec{u} + (\nabla\vec{u})^T\right]$ \\
    Diffusivity & $D$ & $\alpha = \frac{k}{\rho c_p}$ &
      $\nu = \frac{\eta}{\rho}$ \\
    %Convective flux & $C\vec{u}$ & $\rho c_pT\vec{u}$ & $\rho\vec{u}\vec{u}$\\
    \hline
  \end{tabular}
\end{center}
$^*$Only considering diffusive fluxes.  $^T$ in fluid constit. denotes the
transpose of the matrix.

New stuff: vector field instead of scalar; very different units; pressure as
well as flux/shear stress and force.

For those taking or having taken 3.11, shear stress $\tau$ relates to stress
$\sigma$ as follows:
$$\sigma = -\tau -PI$$
where $p$ is $(\sigma_{xx}+\sigma_{yy}+\sigma_{zz})/3$, so $\tau_{xx} +
\tau_{yy} + \tau_{zz} = 0$, $\tau_{xy} = \tau_{yx} = -\sigma_{xy} =
-\sigma_{yx}$.  Note that $\tau_{yx} = \tau_{xy}$ almost always, otherwise
infinite rotation...

Also: mechanics uses displacement for $\vec{u}$, acceleration is its {\em
  second} derivative with time.  Simple shear:
$$\rho\frac{\partial^2\vec{u}}{\partial t^2} = \nabla\cdot\sigma + \vec{F}
\Rightarrow \rho\frac{\partial^2u_x}{\partial t^2} =
\frac{\partial\sigma_{xx}}{\partial x^2} + \frac{\partial\sigma_{yx}}{\partial
  y^2} + \frac{\partial\sigma_{zx}}{\partial z^2} + F_x =
G\frac{\partial^2 u_x}{\partial z^2} + F_x.$$
Analogue to momentum diffusivity: $G/\rho$, units m$^2$/s$^2$, $\sqrt{G/\rho}$:
speed of sound!  Remember with a little jig:

\begin{center}
  \em Fluids are diffusive,\\
  With their velocity and viscosity.\\
  But on replacement with displacement,\\
  it will behave, like a wave!
\end{center}
\newpage


\section{October 31: Mechanics Analogy Revisited, Reynolds Number, Rheology}

Mechanics:
\begin{itemize}
\item Next Weds 11/5: 3B Symposium!
\item 3.21 notes on Stellar (with $\eta$ changed to $\mu$).
\item In all of the viewfactor problems of PS6, ignore the graphs dealing with
  "nonconducting but reradiating" surfaces in the text, as we didn't cover
  those this year (or last year).
\item In disc graph, not $r_1/r_2$ but $D/r_1$ and $r_2/D$.
\item In problem 2, "collar" refers to the heat shield, a cylinder above the
  ZrO2 source.
\item In problem 3, the fluid layer is thin enough that we can neglect the
  curvature of the ram and cylinder.  The ram motion is also vastly powerful in
  terms of driving flow than the weight of the fluid, so we can neglect rho*g
  in the fluid.  With these two simplifications, velocity profile between the
  cylinder and ram is linear, making the problem a lot simpler.
\item In problem 4a, Arrhenius means proportional to $\exp(-\Delta G_a/RT)$.
  For part b, think about what happens to glasses as they cool...
\item In problem 5, think about conduction through a multi-layer wall, in terms
  of the interface condition.  Also, just as the multi-layer wall has two
  layers with different values of A and B in the temperature graph T=Ax+B, here
  we have two fluid layers with different values of A and B in the falling film
  solution given in class (that solution is the answer to part a).  The
  Reynolds number will be discussed in tomorrow's lecture.
\end{itemize}

\noindent Muddy from last time:
\begin{itemize}
\item Crossbar on $z$.
\item Why is $A=0$, $B=\frac{P_1-P_2}{4\mu L}R^2$ in tube?  General solution:
  $$u_z = -\frac{P_1-P_2}{4\mu L}r^2 + A\ln r + B,$$
  boundary conditions: $r=0\rightarrow u_z$ not infinite, $r=R\rightarrow
  u_z=0$.  So, nonzero $A$ in $A\ln r$ gives infinite $u_z$ at $r=0$, use $B$
  to exactly cancel first term at $r=R$.

  Note: at axis of symmetry, $\partial u_z/\partial r=0$, like temperature and
  conc; this too would give $A=0$.
\item Mechanics analogy: very rushed, deserves better treatment, even though
  not a part of this class.
\end{itemize}

\paragraph{Reynolds number}

%NEXT TIME: save Re, turb for later; omit from PS6.
Like other dimensionless numbers, a ratio, this time of convective/diffusive
momentum transfer, a.k.a. inertial/viscous forces.  Formula:
$${\rm Re}=\frac{\rho UL}{\eta}$$
Describes dimensionless velocity for dimensionless drag force $f$; also onset
of turbulence.
\begin{itemize}
\item Tubes: $<2100\rightarrow$ laminar (very constrained).
\item Channel, Couette: $<1000\rightarrow$ laminar.
\item Falling film (inclined plane): $<20\rightarrow$ laminar, due to free
  surface.
\end{itemize}
Note didn't give number for turbulent, that's because it depends on entrance
conditions.

\paragraph{Rheology}

Typical Newtonian viscosities:
\begin{itemize}
\item Water: $\rm10^{-3}N\cdot s/m^2$, density $\rm10^3kg/m^3$, kinematic
  viscosity (momentum diffusivity) $\rm10^{-6}m^2/s$.
\item Molten iron: $\rm5\times10^{-3}N\cdot s/m^2$, density
  $\rm7\times10^3kg/m^3$, kinematic viscosity (momentum diffusivity) just under
  $\rm10^{-6}m^2/s$, close to water.  Water modeling...
\item Air: $\rm10^{-5}N\cdot s/m^2$, density $\rm1.9kg/m^3$, kinematic
  viscosity (momentum diffusivity) $\rm5\times10^{-6}m^2/s$, close to water!
\end{itemize}
But, very different effect on surroundings, drag force.  So even though might
flow similarly over a hill, geologists can tell the difference between water
and wind erosion damage.

Liquids: generally inverse Arrhenius; gases (forgot).

Non-Newtonian: graphs of $\tau_{yx}$ vs. $\partial u_x/\partial y$.
Categories:
\begin{itemize}
\item Dilatant (shear-thickening), example: fluid with high-aspect ratio solid
  bits; blood.  More mixing, momentum mixing, acts like viscosity.  Platelet
  diffusivity, concentration near walls...

  Model: power-law, $n>1$.

\item Pseudoplastic (shear-thinning), examples: heavily-loaded semi-solid, many
  polymers get oriented then shear more easily.

  Model: power law, $n<1$.

  Next time: Wierd shear-thinning behavior at low strain rates due to
  fibrinogen content, extremely sensitive to fibrinogen and measures risk of
  cardiovascular disease better than smoking!  (Gordon Lowe)

\item Bingham plastic: finite yield stress, beyond that moves okay, but up to
  it nothing.  Some heavily-loaded liquids, polymer composites, toothpaste;
  semi-solid metals bond together then break free.

  Model: yield stress $\tau_0$, slope $\mu_P$.
\end{itemize}
Power law relation:
$$\tau_{yx}=\mu_0\left(\frac{\partial u_x}{\partial y}\right)^n$$
More than 1-D leads to wierd Tresca, von Mises criteria, etc.
\newpage


\section{November 3, 2003: Navier-Stokes Equations!}

Mechanics:
\begin{itemize}
\item Weds 11/5: 3B Symposium!
\item PS7 material last on Test 2...  Due next Weds.
\item Today: mid-term course evaluations!
\end{itemize}

\noindent Muddy from last time:
\begin{itemize}
\item In fluids, is $\tau$ a shear or normal stress?  Equating it to $\sigma$
  is confusing...  It's a shear stress.  It might have ``normal'' components,
  but if you rotate it right, it's all shear.
\item What's the difference between a liquid and a Newtonian fluid?  Liquids
  are fluids, as are gases, and plasmas.  Definition: finite (nonzero) shear
  strain rate for small stresses.  Bingham plastic not technically a fluid, nor
  in a sense is a strict power-law pseudoplastic (though tends to break down
  near 0).
\end{itemize}

The Navier-Stokes equations!  Pinnacle of complexity and abstraction in this
course.  From here, we explain, we see more examples, we fill in more details.
Probably last time without a motivating process...  So don't be surprised if
you don't understand all of this just now, it's a bunch of math but should be
clearer as we go on.

3.21 ``Are you ready for momentum convection?''

\paragraph{Conservation of mass}

2-D Navier-Stokes: two equations for three unknowns!  Need one more equation,
conservation of mass, only in-out by convective mass flux $\rho\vec{u}$, no
mass diffusion or generation:
$$\frac{\partial\rho}{\partial t} = -\nabla\cdot(\rho\vec{u})$$
$$\frac{\partial\rho}{\partial t} + \vec{u}\cdot\nabla\rho +
\rho\nabla\cdot\vec{u} = 0$$
$$\frac{D\rho}{Dt} + \rho\nabla\cdot\vec{u} = 0$$

Incompressible definition: $D\rho/Dt=0$.  Example: oil and water, discuss
$D\rho/Dt$ and $\partial\rho/\partial t$.  Incompressible, not constant/uniform
density.  Result: $\nabla\cdot\vec{u}=0$.

\paragraph{Eulerian derivation of Navier-Stokes}

Like before; this time add convective momentum transfer.  What's that?  In
convective mass transfer it was $\rho c_p \vec{u}T$.  Now it's
$\rho\vec{u}\vec{u}$.  Wierd outer product second-rank tensor!

But what is momentum convection?  Those $u_y\frac{\partial u_x}{\partial y}$
terms.  We'll come back to those later.  Recall heat transfer with convection:
$$\rho c_p\frac{\partial T}{\partial t} + \nabla\cdot(\rho c_p\vec{u}T) =
-\nabla\cdot\vec{q} + \dot{q}$$
With fluids, it's the same:
$$\frac{\partial(\rho\vec{u})}{\partial t} + \nabla\cdot(\rho\vec{u}\vec{u}) =
-\nabla p - \nabla\cdot\tau + \vec{F}$$
Left side expansion, simplification:
$$\rho\frac{\partial\vec{u}}{\partial t} +
\vec{u}\frac{\partial\rho}{\partial t} +
\rho\vec{u}\nabla\cdot\vec{u} + \rho\vec{u}\cdot\nabla\vec{u} +
\vec{u}\vec{u}\cdot\nabla\rho =
\vec{u}\left[\frac{\partial\rho}{\partial t} + \vec{u}\cdot\nabla\rho +
  \rho\nabla\cdot\vec{u}\right] + \rho\left[\frac{\partial\vec{u}}{\partial t}
  + \vec{u}\cdot\nabla\vec{u}\right]$$
Recall the continuity equation:
$$\frac{\partial\rho}{\partial t} + \vec{u}\cdot\nabla\rho +
\rho\nabla\cdot\vec{u} = 0.$$
Entire first part cancels!  Result:
$$\rho\frac{D\vec{u}}{Dt} = -\nabla p - \nabla\cdot\tau + \vec{F}.$$

Now with Newtonian viscosity, incompressible:
$$\tau = -\mu\left(\nabla\vec{u} + \nabla\vec{u}^T -
  \frac{2}{3}\nabla\cdot\vec{u}I\right).$$
For $x$-component in 2-D:
$$\tau_x = -\mu\left[2\frac{\partial u_x}{\partial x}\hat{\imath} +
  \left(\frac{\partial u_x}{\partial y} +
    \frac{\partial u_y}{\partial x}\right)\hat{\jmath}\right]$$
$$-\nabla\cdot\tau_x = \mu\left[2\frac{\partial^2u_x}{\partial x^2} +
  \frac{\partial^2 u_x}{\partial y^2} +
  \frac{\partial^2 u_y}{\partial x\partial y}\right] =
\mu\left[\nabla^2 u_x + \frac{\partial}{\partial x}
  \left(\frac{\partial u_x}{\partial x} +
    \frac{\partial u_y}{\partial y}\right)\right]$$
So, that simplifies things quite a bit.

Incompressible Newtonian result:
$$\rho\frac{Du_x}{Dt} = -\frac{\partial p}{\partial x} + \mu\nabla^2u_x +F_x$$
Written out in 3-D:
$$\rho\left(\frac{\partial u_x}{\partial t} +
  u_x\frac{\partial u_x}{\partial x} + u_y\frac{\partial u_x}{\partial y} +
  u_z\frac{\partial u_x}{\partial z}\right) =
-\partial p/\partial x + \mu\left(\frac{\partial^2 u_x}{\partial x^2} +
  \frac{\partial^2 u_x}{\partial y^2} + \frac{\partial^2 u_x}{\partial z^2}
\right) + F_x$$
Identify (nonlinear) convective, viscous shear ``friction'' terms, sources.

The full thing in vector notation:
$$\rho\frac{D\vec{u}}{Dt} = -\nabla p + \mu\nabla^2\vec{u} + \vec{F}$$
Pretty cool.  Again, sorta like the diffusion equation.

[For more on this, including substantial derivative, see 2000 David Dussault
material, and ``omitted'' paragraphs from 2001.]
\newpage


\section{November 5, 2003: Using the Navier-Stokes Equations}

Mechanics:
\begin{itemize}
\item Tonight: 3B Symposium!  Dinner 5:30 in Chipman room.
\item New PS7 on Stellar: no TODO, \#4c graph on W$^3$R p. 188.
\item Midterm Course Evals:
  \begin{itemize}
  \item Lectures mostly positive, cards great (one: takes too long); negatives:
    1/3 too fast, 1/2 math-intense; need concept summaries.
  \item TA: split, most comfortable, 1/3 unhelpful, many: recs need more PS
    help.  ``Makes the class not hurt so badly.''
  \item PSes: most like, old PSes a prob, need more probs for test studying.
  \item Test: like policy, but too long.
  \item Text: few helpful different approach, most useless!  But better...
  \item Time: from 2-5 or 3+ to 12-18; ``4 PS + 2-4 banging head against
    wall.''
  \end{itemize}
\end{itemize}

\noindent Muddy from last time:
\begin{itemize}
\item When we started momentum conservation, we had a $P$, but at the end, only
  $\rho$.  Where did the $P$ go?  Still there.  General equations:
  $$\frac{D\rho}{Dt} + \rho\nabla\cdot\vec{u} = 0$$
  $$\rho\frac{D\vec{u}}{Dt} = -\nabla P - \nabla\cdot\tau + \vec{F}$$
  Incompressible, Newtonian, uniform $\mu$:
  $$\nabla\cdot\vec{u} = 0$$
  $$\rho\frac{D\vec{u}}{Dt} = -\nabla P + \mu\nabla^2\vec{u} + \vec{F}$$
\item Fifth equation in compressible flows: $\rho(P)$, e.g. ideal gas $\rho =
  MP/RT$.
\end{itemize}

\paragraph{Convective momentum transfer} That nonlinear $D\vec{u}/Dt$ part that
makes these equations such a pain!

Example: $t=0\rightarrow u_x=1$, $u_y=2x$, sketch, show at $t=0$ and $t=1$.
Convecting eddies in background velocity.

\paragraph{Using the Navier-Stokes Equations}

Handout, start with full equations and cancel terms.  Flow through tube
revisited.  Longitudinal pressure trick with $z$-derivative of $z$-momentum
equation:
$$\frac{\partial^2P}{\partial z^2} = \frac{\partial}{\partial z}
\left[\frac{\mu}{r}\frac{\partial}{\partial r}
  \left(\frac{\partial u_z}{\partial r}\right)\right] =
\frac{\mu}{r}\frac{\partial}{\partial r}
\left(\frac{\partial^2 u_z}{\partial r\partial z}\right) = 0!\ P=
A(r,\theta)z + B(r,\theta)$$
$$0 = -\frac{\partial P}{\partial z} + \frac{\mu}{r}\frac{\partial}{\partial r}
\left(\frac{\partial u_z}{\partial r}\right) \Rightarrow
u_z = \frac{\partial P/\partial z}{4\mu}r^2 + A\ln r + B =
-\frac{P_1-P_2}{4\mu L}z^2 + A\ln r + B.$$
Lateral pressure: if $\theta=0$ points up:
$$\rho g_r=-\rho g\cos\theta=\partial P/\partial r,$$
$$\rho g_\theta=\rho g\sin\theta = (1/r)\partial P/\partial\theta.$$
Resulting pressure:
$$P = - \rho gr\cos\theta + f(z) = f(z) - \rho gh,$$
for $h$ increasing in the upward direction from $z$-axis.  Final result:
$$P = Az - \rho gh + C.$$
\newpage


\section{November 7, 2003: Drag Force}

Mechanics:
\begin{itemize}
\item New PS7 on Stellar: no TODO, \#4c graph on W$^3$R p. 188.
\item Midterm Course Evals:
  \begin{itemize}
  \item Lectures mostly positive, cards great (one: takes too long); negatives:
    1/3 too fast, 1/2 math-intense; need concept summaries.
  \item TA: split, most comfortable, 1/3 unhelpful, many: recs need more PS
    help.  ``Makes the class not hurt so badly.''
  \item PSes: most like, old PSes a prob, need more probs for test studying.
    Bad: last material in last lecture before due!  This time: option Weds/Fri,
    okay?
  \item Test: like policy, but too long, too long delay to retake.
  \item Text: few helpful different approach, most useless!  But better...
  \item Time: from 2-5 or 3+ to 12-18; ``4 PS + 2-4 banging head against
    wall.''
  \end{itemize}
\item Test 2 in less than two weeks...  Shorter wait for retake this time.
\end{itemize}

\noindent Muddy from last time:
\begin{itemize}
\item Redo pressure derivation.
  $$\frac{\partial^2P}{\partial z^2} = \frac{\partial}{\partial z}
  \left[\frac{\mu}{r}\frac{\partial}{\partial r}
    \left(\frac{\partial u_z}{\partial r}\right)\right] =
  \frac{\mu}{r}\frac{\partial}{\partial r}
  \left(\frac{\partial^2 u_z}{\partial r\partial z}\right) = 0!\ P=
  A(r,\theta)z + B(r,\theta)$$
  Lateral pressure: if $\theta=0$ points up:
  $$\rho g_r=-\rho g\cos\theta=\partial P/\partial r,$$
  $$\rho g_\theta=\rho g\sin\theta = (1/r)\partial P/\partial\theta.$$
  Resulting pressure:
  $$P = - \rho gr\cos\theta + f(z) = f(z) - \rho gh,$$
  for $h$ increasing in the upward direction from $z$-axis.  Final result:
  $$P = Az - \rho gh + C.$$

\item What's fully-developed, edge effects in cylindrical coordinates?  Flow
  direction derivatives deal with fully-developed.  If axisymmetric and flow is
  mostly $\theta$, then can use this as ``fully-developed''.

  Edge effects: flow direction, large variation direction, third direction is
  ``edge'' direction.  For tube flow, was $\theta$, so axisymm is equiv to ``no
  edge effects''.  For rod-cup, will be something else.
\end{itemize}

\paragraph{Drag force}

Integrated traction (force per unit area) in one direction.  For tubes, it's in
the $z$-direction, use the shear traction (stress) times area: $F_d = \tau\cdot
A$.  Here, looking at $\tau_{rz}$:
$$u_z = \frac{P_1-P_2}{4\mu L}(R^2-r^2) \Rightarrow
\tau_{rz} = -\mu\frac{\partial u_z}{\partial r} = \frac{P_1-P_2}{2 L} r.$$
$$F_d = 2\pi RL\cdot\left.\tau_{rz}\right|_{r=R} =
2\pi RL \frac{P_1-P_2}{2 L} R = \pi R^2(P_1-P_2).$$
Neat result: this is just the net pressure force.  Okay, what if we know
desired velocity, want to estimate required pressure?  In terms of $u_{av}$:
$$u_{av} = \frac{u_{max}}{2} = \frac{(P_1-P_2)R^2}{8\mu L},$$
$$F_d = \frac{(P_1-P_2)R^2}{8\mu L}\cdot 8\pi\mu L = 8\pi\mu L u_{av}.$$
For any laminar flow, this is the drag force.  Drag is shear traction times
area:
$$F_d = \tau_{rz}\cdot 2\pi RL \Rightarrow \tau_{rz} = \frac{F_d}{2\pi RL} =
\frac{8\pi\mu L u_{av}}{2\pi RL} = \frac{4\mu}{R}u_{av} =
\frac{8\mu}{d}u_{av}.$$

Okay, using $u_{av}$ to correlate with flow rate whether laminar or turbulent.
So what about turbulent?  Density plays a role because of convective terms.
Dimensional analysis:
$$\tau = f(U, \mu, d, \rho)$$
Five parameters, three base units, so two dimensionless parameters.  Four
different nondimensionalizations!
\begin{center}
  \begin{tabular}[h]{c|cc|}
    Keep           & $\pi_\tau$ & $\pi_{other}$ \\ \hline
    $\tau$, $\mu$ & $\frac{\tau}{\rho U^2}$ & $\pi_\mu=\frac{\mu}{\rho Ud}$ \\
    $\tau$, $\rho$ & $\frac{\tau d}{\mu U}$ & $\pi_\rho=\frac{\rho Ud}{\mu}$ \\
    $\tau$, $U$ & $\frac{\tau d^2\rho}{\mu^2}$ & $\pi_U=\frac{\rho Ud}{\mu}$ \\
    $\tau$, $d$ & $\frac{\tau}{\rho U^2}$ & $\pi_d=\frac{\rho Ud}{\mu}$ \\
    \hline
  \end{tabular}
\end{center}
First and last are essentially the same, though last is more familiar (Reynolds
number), so ignore the first.  Third is just a mess, so throw it out.  Second
is a great fit to what's above.  So use last, or second?

With turbulence, there are different curves with different surface roughnesses,
roughly proportional to $U^2$.  If use second, get one flat $\pi_\tau$ laminar,
multiple lines for turbulent $\pi_\tau$.  If use first/last, one line for
laminar, multiple flats for turbulance (p. 188).  So this is generally more
convenient.

Dimensionless $\pi_\tau$ is called (fanning) friction factor $f$ ($f_f$).  The
denominator $\frac{1}{2}\rho U^2$ is the approximate kinetic energy density,
we'll call it $K$, a.k.a. dynamic pressure.  So:
$$\tau=fK,\ F_d = fKA, f=\frac{\tau}{\frac{1}{2}\rho U^2} = f({\rm Re}).$$
Laminar flow friction factor:
$$f=\frac{\tau}{\frac{1}{2}\rho U^2} =
\frac{\frac{8\mu}{d}u_{av}}{\frac{1}{2}\rho U^2} = \frac{16\mu}{\rho Ud} =
\frac{16}{\rm Re}.$$
To calculate drag force:  Reynolds number (and surface roughness) $\rightarrow$
friction factor $f$ $\rightarrow$ $\tau=fK,$ $F_d=fKA$.
\newpage


\section{November 12, 2003: Drag Force on a Sphere}

Mechanics:
\begin{itemize}
\item Test 2 11/19 in 2-143, preview handout today.
\item PS7 extended to Friday.
\end{itemize}

\noindent Muddy from last time:
\begin{itemize}
\item How is $f=16\mu/\rho Ud$?
  $$\tau_{rz} = \frac{8\mu}{d}u_{av}\Rightarrow
  f=\frac{\tau_{rz}}{\frac{1}{2}\rho u_{av}^2} =
  \frac{8\mu u_{av}/d}{\frac{1}{2}\rho u_{av}^2} = \frac{16\mu}{\rho u_{av}d} =
  \frac{16}{\rm Re}.$$
\item What's the point of defining $f$?  We just need $\tau$, right?  Well, $f$
  is easy for laminar flow, for turbulence it's more complicated.  This gives
  us a parameter to relate to the Reynolds number for calculating drag force in
  more general situations.  More examples on the way.
\item Or was it $\vec{t}$, not $f$?  In that case, traction
  $\vec{t}=\sigma\cdot\hat{n}$, force per unit area.  In this case, with
  $n=\hat{r}$, $\vec{t}=(\tau_{rr}+p, \tau_{r\theta}, \tau_{rz})$.  The
  relevant one for the $z$-direction is $\tau_{rz}$, but traction is more
  general, see an example later today.
\item Why was option \#4 the ``better'' graph?  Expanded version: with
  roughness at high Re, $f$ is constant.  So if Re=10$^8$, friction factor is
  {\em only} a function of $e/d$ down to $e/d=10^{-5}$!
\item Please review this process $Q\rightarrow u_{av}\rightarrow{\rm
    Re}\rightarrow f\rightarrow\tau\rightarrow F_d\rightarrow\Delta P$.
  $$\frac{Q}{\pi R^2}=u_{av},\ {\rm Re}=\frac{\rho u_{av}d}{\mu},\ f=f({\rm
    Re}, e/d),\ \tau_{rz}=fK,\ F_{d,z}=\tau_{rz} A=fKA (A=2\pi RL),\ \Delta P=\frac{F_d}{\pi
    R^2}.$$
\end{itemize}

\paragraph{Reynolds Number revisited}

Low velocity: shear stress; high velocity: braking kinetic energy.  Ratio of
forces:
$${\rm Re} = \frac{\rm convective\ momentum\ transfer}{\rm shear\ momentum
  \ transfer} = \frac{\rm inertial\ forces}{\rm viscous\ forces}\simeq
\frac{\rho u_y\frac{\partial u_x}{\partial y}}
{\mu\frac{\partial^2u_x}{\partial y^2}} \simeq
\frac{\rho U U/L}{\mu U/L^2} = \frac{\rho UL}{\mu}.$$

\paragraph{Flow past a sphere}

Motivating process: Electron beam melting and refining of titanium alloys.
Water-cooled copper hearth, titanium melted by electron beams, forms solid
``skull'' against the copper.  Clean heat source, liquid titanium contained in
solid titanium, results in very clean metal.  Mystery of the universe: how does
liquid Ti sit in contact with solid Cu?  Main purpose: removal of hard TiN
inclusions often several milimeters across which nucleate cracks and bring down
airplanes!  (1983 Sioux City, Iowa.)

Set up problem: sphere going one way $u_{sphere}$, fluid other way $u_\infty$,
local disturbance but relative velocity $U=u_\infty-u_{sphere}$, relative veloc
of fluid in sphere frame.  Drag force is in this direction.

For a sphere, drag force is slightly different: it has not only shear, but
pressure component as well.  Traction $\vec{t}=\sigma\cdot\hat{n}$.
Stokes flow: ingore the convective terms, result (pp. 68-71):
$$F_d = 3\pi\mu du_{rel}.$$
At high velocity, similar friction factor concept to tube:
$$F_d=fKA = f\cdot\frac{1}{2}\rho U^2\cdot\frac{1}{4}\pi d^2.$$
Again, $f({\rm Re})$, but not really $\pi_\tau$ because $\tau$ is all over the
place, more of an average.  This time, low Re ($<$0.1) means Stokes flow, can
ignore all convective terms; analytical result in 3.21 notes, drag force:
$$F_d = 3\pi\mu Ud = f\cdot\frac{1}{2}\rho U^2\cdot\frac{1}{4}\pi d^2
\Rightarrow f=\frac{24\mu}{\rho Ud} = \frac{24}{\rm Re}.$$
If faster, though not turbulent, $f$ becomes a constant.  Graph on W$^3$R
p. 153 of $f$ (they call $c_D$) vs. Re.  Constant: about 0.44, that's what I've
known as the drag coefficient.  Cars as low as 0.17, flat disk just about 1,
making dynamic pressure a good estimate of pressure difference.

For rising/sinking particles, set drag force to buoyancy force, solve for
velocity.  If not Stokes flow, you're in trouble, way to do it but it's
complicated.

Can {\em NOT} use this for bubbles.  For those, $F_d=2\pi\mu Ud$ all the way
out to Re=10$^5$!

So, what about precipitation?  Buoyancy force and weight vs. drag force, all
sum to zero:
$$F_w = \frac{1}{6}\pi d^3\rho_{sphere}, F_b = \frac{1}{6}\pi d^3\rho_{fluid},
F_d = {\rm what\ we\ just\ calculated}.$$
After all, if you're not part of the solution, you're part of the precipitate.
(Ha ha)
\newpage


\section{November 14, 2003: Boundary Layers Part I}

Mechanics:
\begin{itemize}
\item Wulff Lecture Tues 4:15 6-120: {\em Information Transport and Computation
    in Nanometer-Scale Structures}, Don Eigler, IBM Fellow.
\item Test 2 11/19 in 2-143.  Solving Fluids Problems provided if needed.
\item PS7 solution error: \#1 replace $L$ with $\delta$, ``length'' with $L$.
  Correction on Stellar.
\end{itemize}

\noindent Muddy from last time:
\begin{itemize}
\item Which are the convective and viscous terms?  In vector Navier-Stokes
  momentum:
  $$\rho\frac{D\vec{u}}{Dt} = \rho\left(\frac{\partial\vec{u}}{\partial t} +
    \vec{u}\cdot\nabla\vec{u}\right) = -\nabla P + \mu\nabla^2\vec{u} +
  \vec{F}.$$
  Convective terms are the $\vec{u}\cdot\nabla\vec{u}$ terms, like $u_x\partial
  u_y/\partial x$, the nonlinear terms which make the equations such a pain to
  solve and create turbulence...  Viscous ones are $\mu\nabla^2\vec{u}$.
\item Why do thin ellipsoids have less drag than spheres, but flat plates have
  more?  Depends on orientation.  P\&G has a few more examples on p. 87.  Neat
  thing: the Stokes flow $f=24/$Re holds for all of them!
\item Are log($f$) vs. log(Re) plots for only turbulent, or both laminar and
  turbulent?  Both, that's the neat thing.  For tube, sphere, and BL, it
  captures everything.
\end{itemize}

\paragraph{Sphere flow wrapup}

Another neat way to think about Re: ratio of inertial to shear forces
$$\frac{\rho U^2\cdot\pi d^2}{\mu dU}\propto \frac{\rho Ud}{\mu}= {\rm Re}.$$
Can {\em NOT} use this flow-past-sphere stuff for bubbles.  For those,
$F_d=2\pi\mu Ud$ all the way out to Re=10$^5$!  Boundary conditions...

\paragraph{``Boundary layers'' in a solid}

Thought experiment with moving solid: extruded polymer sheet (like PS4
extruded rod problem).  Start at high temp, if well-cooled so large Biot then
constant temperature on surface; no generation.  Full equation:
$$\frac{DT}{Dt} = \alpha\nabla^2T$$
Define boundary layer thickness $\delta{x}$ where temperature deviates at least
1\% from far-field.  If $\delta\ll x$, then
$$\frac{\partial^2T}{\partial y^2}>>\frac{\partial^2 T}{\partial x^2}$$
$$u_x\frac{\partial T}{\partial x} = \alpha\frac{\partial^2T}{\partial y^2}$$
Transform: $\tau = x/u_x$, becomes diffusion equation, erf solution:
$$\frac{T-T_s}{T_i-T_s} = {\rm erf}\frac{y}{2\sqrt{\alpha x/u_x}}$$
If we define $\delta$ as where we get to 0.99, then erf$^{-1}(.99)=1.8$, and
$$y=\delta\ {\rm where}\ \frac{\delta}{2\sqrt{\alpha x/u_x}}=1.8$$
$$\delta = 3.6\sqrt\frac{\alpha x}{u_x}$$
Obviously breaks down at start $x=0$, but otherwise sound.

\paragraph{Boundary layers in a fluid}

Now we want to calculate drag force for flow parallel to the plate.

Similar constant IC to solid, infinite BC, call it $U_\infty$.  Difference: BC
at $y=0$: $u_x=u_y=0$.  Have to solve 2-D incompressible steady-state
Navier-Stokes:
$$\frac{\partial u_x}{\partial x}+\frac{\partial u_y}{\partial y} = 0$$
$$u_x\frac{\partial u_x}{\partial x} + u_y\frac{\partial u_x}{\partial y} =
-\frac{\partial p}{\rho\partial x} +
\nu\left(\frac{\partial^2 u_x}{\partial x^2} +
  \frac{\partial^2 u_x}{\partial x^2}\right)$$
$$u_x\frac{\partial u_y}{\partial x} + u_y\frac{\partial u_y}{\partial y} =
-\frac{\partial p}{\rho\partial y} +
\nu\left(\frac{\partial^2 u_y}{\partial x^2} +
  \frac{\partial^2 u_y}{\partial x^2}\right)$$
Then a miracle occurs, the Blassius solution for $\delta\ll x$ is a graph of
$u_x/U_\infty$ vs. $\beta=y\sqrt{U_\infty/\nu x}$; hits 0.99 at ordinate of 5:
$$\delta = 5.0\sqrt{\frac{\nu x}{U_\infty}}.$$
Why 5.0, not 3.6?  Because there must be vertical velocity due to mass
conservation (show using differential mass equation and integral box), carries
low-$x$-velocity fluid upward.  Slope: 0.332, will use next time to discuss
shear stress and drag force.
\newpage


\section{November 17, 2003: Boundary Layers Part II}

Mechanics:
\begin{itemize}
\item Wulff Lecture Tues 4:15 6-120: {\em Information Transport and Computation
    in Nanometer-Scale Structures}, Don Eigler, IBM Fellow.
\item PS7 solution error: \#1 replace $L$ with $\delta$, ``length'' with $L$.
  Correction on Stellar.
\item Zephyr hours tomorrow 9-12, 1-3; also Weds 4-7 PM.
\item Test 2 11/19 in 2-143.  Solving Fluids Problems provided if needed.
\end{itemize}

\noindent Muddy from last time:
\begin{itemize}
\item Nothing!
\end{itemize}

Recall our ``miraculous'' Blassius solution to the 2-D Navier-Stokes equations
(draw the graph)...  At $y=0$, slope: 0.332, so viscous drag:
$$\tau_{yx} = -\mu\frac{\partial u_x}{\partial y} =
-\mu \frac{\partial u_x}{\partial\beta}\frac{\partial\beta}{\partial y}$$
$$\tau_{yx} = -\mu\cdot0.332U_\infty\sqrt{\frac{U_\infty}{\nu x}}$$
Note: a function of $x$ (larger near leading edge), diverges at $x=0$!  But
$\delta\ll x$ does not hold there.

Now set to a friction factor:
$$\tau_{yx} = -0.332\sqrt{\frac{\rho\mu U_\infty^3}{x}} =
f_x \cdot \frac{1}{2}\rho U_\infty^2$$
This time $\tau$ is not constant, so we have different $f_x=\tau/K$ and
$f_L=F_d/KA$.  Let's evaluate both:
$$f_x = 0.664\sqrt{\frac{\mu}{\rho U_\infty x}} =
\frac{0.664}{\sqrt{{\rm Re}_x}}$$

Also, note dimensionless BL thickness:
$$\delta = 5.0\sqrt{\frac{\nu x}{U_\infty}}$$
$$\frac{\delta}{x} = 5.0\sqrt{\frac{\nu}{U_\infty x}} =
\frac{5.0}{\sqrt{{\rm Re}_x}}$$

Lengthwise, global drag force, average friction factor.  Neglect edge effects
again...
$$F_d = \int \tau_{yx} dA = W \int_{x=0}^L tau_{yx} dx$$
$$F_d = W \int_{x=0}^L0.332\sqrt{\frac{\rho\mu U_\infty^3}{x}}dx$$
$$F_d = 0.332 W \sqrt{\rho\mu U_\infty^3}\cdot 2\sqrt{L}$$
$$F_d = 0.664 W \sqrt{\rho\mu U_\infty^3 L}$$
Now for the average friction factor/drag coefficient:
$$f_L = \frac{F_d}{KA} = \frac{0.664 W \sqrt{\rho\mu U_\infty^3 L}}
{\frac{1}{2}\rho U_\infty^2 \cdot WL} =
1.328\sqrt{\frac{\mu}{\rho U_\infty L}} = \frac{1.328}{\sqrt{{\rm Re}_L}}$$
This is what is meant by average and local friction factors on the Test 2
overview sheet.  Don't need to know for test 2, since for a tube they're the
same.  For a sphere, only defined average/global, but for a BL, they're
different.

\paragraph{Entrance Length}

For channel flow between two parallel plates spaced apart a distance $H$, we
can define the entrance length $L_e$ as the point where the boundary layers
from each side meet in the middle.  The twin Blassius functions are close
enough to the parabolic profile that we can say it's fully-developed at that
point.  So we can plug in the boundary layer equation if flow is laminar:
%NEXT TIME: use $u_{av}$ instead of $U_\infty$.
$$x=L_e \Rightarrow \frac{H}{2} = \delta = 5.0\sqrt{\frac{\nu x}{U_\infty}},$$
$$L_e = \frac{H^2U_\infty}{100\nu}.$$
If $L_e\ll L$, then flow is fully-developed for most of the tube, so the
fully-developed part will dominate the drag force and $F_d=\tau\cdot2\pi RL$.

Movie Friday...
\newpage


\section{November 21, 2003: Turbulence}

Mechanics:
\begin{itemize}
\item Turbulence movie: QC151.T8; guide QC145.2.F5 Barker Media.
\end{itemize}

\noindent Muddy from last time:
\begin{itemize}
\item What's the physical significance of ``Blassius''?  Some guy who came up
  with this function to solve a slightly-reduced 2-D Navier-Stokes.  There are
  actually three, for $u_x$, $u_y$ and $p$ in the boundary layer, all
  $\delta\ll x$.  For drag force, only $u_x$ is needed.
\item Is $dBL/d\beta$ only the slope at the initial part of the curve?  If by
  ``initial'', you mean $y=0$ (the bottom), then yes.
\item What's the difference between $f_x$ and $f_L$?  Here, $\tau$ is not
  constant, graph $\tau$ and $\tau_{av}\rightarrow F_d$, show how average is
  twice local at the end.  Hence $f_L$ is twice $f_x$.
\item How about ``Re$_x$'' and ``Re$_L$''?  No significance, just different
  lengthscales, one for local and one for global/average.
\item Difference between $u_{av}$ and $U_\infty$?  Should be no $u_{av}$ for BL
  problems, sorry if I made a writo.  D'oh!  This is wrong, see next lecture's
  notes.
\item What's up with $\delta/x$?  Just a ratio, dimensionless for convenience,
  allows to evaluate $\delta\ll x$.
\item What is velocity profile for entrance length if not laminar?  Get to that
  later (next Monday or so).
\item What happens near leading edge?  Something like the sphere: kinda
  complicated.  Maybe solvable for Stokes flow...
\end{itemize}

\paragraph{Turbulence}

Starting instability, energy cascade.  Vortices grow in a velocity gradient
because of momentum convection, damped due to viscosity; therefore, tendency
increases with increasing Re.

Resulting behavior:
\begin{itemize}
\item Disorder.
\item ``Vorticity'' in flow, 3-D.
\item Lots of mixing, of mass and heat as well as momentum.
\item Increased drag due to momentum mixing, as small vortices steal energy
  from the flow.
\end{itemize}

The movie!

Parviz Moin and John Kim, ``Tackling Turbulence with Supercomputers,'' {\em
Scientific American} January 1997 pp. 62-68. % file sciam.pdf, can put on
% Stellar or link from Scientific American
\begin{quote}
  Turbulence may have gotten its bad reputation because dealing with it
  mathematically is one of the most notoriously thorny problems of classical
  physics.  For a phenomenon that is literally ubiquitous, remarkably little of
  a quantitative nature is known about it.  Richard Feynman, the great Nobel
  Prize-winning physicist, called turbulence ``the most important problem of
  classical physics.''  Its difficulty was wittily expressed in 1932 by the
  British physicist Horace Lamb, who, in an address to the British Association
  for the Advancement of Science, reportedly said, ``I am an old man now, and
  when I die and go to heaven there are two matters on which I hope for
  enlightenment.  One is quantum electrodynamics, and the other is the
  turbulent motion of fluids.  And about the former I am rather optimistic.''
\end{quote}

That article goes on to talk about direct numerical simulation of all of the
details of turbulent flows, of which I am not a great fan.  Why?  as pointed
out in my {\em JOM} article ``3-D or not 3-D'', even if computers continue to
double in computational power every eighteen months, the fifth-power scaling of
complexity with lengthscale in three dimensions (cubic in space times quadratic
in time) means that the resolution of these simulations will double only every
seven-and-a-half years!

To take a simple illustrative example, direct numerical simulation of
turbulence in continuous casting involves Reynolds numbers on the order of one
million, and the smallest eddies are a fraction of a millimeter across and form
and decay in a few milliseconds.  One must account for interaction with the
free surface, including mold powder melting and entrainment, as well as mold
oscillation, and surface roughness of the solidifying metal, since with
turbulent flow, the details of these boundary conditions can make a large
difference in macroscopic behavior.  The formulation alone is daunting, and
computational work required to solve all of the equations on each of the tens
of trillions of grid points over the millions of timesteps required to approach
steady-state will be prohibitively costly for many years, perhaps until long
after Moore's law has been laid to rest (indeed, the roughly four petabytes of
memory required to just store a single timestep would cost about two billion
dollars at the time this is being written).  Furthermore, postprocessing that
many degrees of freedom would not only be computationally difficult, but it is
not clear that our minds would be able to comprehend the resulting complexity
in any useful way, and further, the exercise would be largely pointless, as one
really cares only about coarse-grained averages of flow behavior, and detailed
behavior perhaps at certain interfaces.

%Note: all text with % at the start of the line is a comment and SHOULD NOT BE
%IN THE PRINTED VERSION!

Analysis: for 1 $\rm\frac{m}{s}$ flow through the nozzle 0.1 m in diameter,
with kinematic viscosity of .005/7000$\simeq10^{-6}$ $\rm\frac{m^2}{s}$, this
gives us Re$_d\simeq10^5$.  Using the larger lengthscale $H$ of the caster,
around 1 m, this gives Re$_H\simeq10^6$, this is given in the paragraph above.
Using standard enengy cascade/Kolmogorov microscale analysis, the energy
dissipation rate for the largest turbulent eddies in a tube is given by
$$\epsilon\sim\mu_t\left(\frac{U}{d}\right)^2,$$
where $\mu_t$ is the turbulent viscosity, $U$ the eddy velocity estimated by
the average velocity, and $d$ the eddy size estimated by the tube diameter.
Combining this with the well-known Kolmogorov result for the smallest eddy
lengthscale $\ell$:
$$\ell\sim\sqrt[4]{\frac{\mu^3}{\rho^2\epsilon}}$$
($\rho$ is density) gives the smallest eddy lengthscale as
$$\ell\sim d\frac{1}{\sqrt{{\rm Re}_d}}\sqrt[4]{\frac{\mu}{\mu_t}}.$$
Even using the conservative estimate of $\mu_t=30\mu$ gives $\ell\sim0.1$
mm, in 1 meter cubed this gives a trillion grid points, but you want a few
grid points across each smallest eddy which means about a few$^3\simeq100$
times more grid points, hence ``tens of trillions''.

Put slightly differently, the total rate of energy dissipated in a jet at
steady-state is the rate of kinetic energy input, which is the product of
volumetric kinetic energy given by dynamic pressure and the flow rate:
$$\epsilon V=\frac{1}{2}\rho U^2\cdot\frac{\pi}{4}d^2U= \frac{\pi}{8}\rho
U^3d^2$$
Taking the volume as 1 m$^3$, average velocity of 1 $\rm\frac{m}{s}$, nozzle
diameter of $0.1$ m and density of 7000 $\rm\frac{kg}{m^3}$ gives
$\rm\epsilon\simeq30\frac{W}{m^3}$.  Putting this into the Kolmogorov
lengthscale expression again gives $\ell\sim0.1$ mm.
\newpage


\section{November 24, 2003: Turbulence, cont'd}

Mechanics:
\begin{itemize}
\item Turbulence movie: QC151.T8; guide QC145.2.F5 Barker Media.
\end{itemize}

\noindent Muddy from last time:
\begin{itemize}
\item Difference between $u_{av}$ and $U_\infty$?  I messed up last time, for
  the entrance length situaiton, we we take average velocity as $U_\infty$, the
  initial ``free stream'' velocity.  Sorry to confuse you last time.
\item On that note, just as we have the dimensionless boundary layer thickness,
  we also have the dimensionless entrance length:
  $$L_e=\frac{H^2u_{av}}{100\nu} \Rightarrow
  \frac{L_e}{H}=\frac{{\rm Re}_H}{100}.$$
  So a large Reynolds number means a long entrance length (1000 means ten times
  the channel width), and vice versa.
\end{itemize}

Energy cascade and the Kolmogorov microscale.  Largest eddy Re=$UL/\nu$,
smallest eddy Reynolds number $u\ell/\nu\sim1$.  Energy dissipation, W/m$^3$;
in smallest eddies:
$$\epsilon = \eta\left(\frac{du}{dx}\right)^2 \sim \eta\frac{u^2}{\ell^2}$$
Assuming most energy dissipation happens there, we can solve these two
equations, get smallest eddy size and velocity from viscosity, density and
dissipation:
$$u\sim\ell\sqrt{\frac{\epsilon}{\eta}} \Rightarrow
\frac{\rho\ell^2}{\eta}\sqrt{\frac{\epsilon}{\eta}}\sim 1$$
$$\ell\sim\left(\frac{\eta^3}{\rho^2\epsilon}\right)^\frac{1}{4}$$
This defines the turbulent microscale.  For thermal or diffusive mixing,
turbulence can mix things down to this scale, then molecular diffusion or heat
conduction has to do the rest.  Time to diffusive mixing in turbulence is
approximately this $\ell^2/D$.

So suppose we turn off the power, then what happens?  Smallest eddies go away
fast, then larger ones, until the whole flow stops.  Timescale of smallest is
$\ell^2/\nu$, largest is $L^2/\nu_t$, turbulent effective viscosity.  Get into
modeling and structure later if time is available.

\paragraph{Turbulent boundary layer}

Laminar is good until Re$_x=10^5$, associated boundary layer thickness and
local friction factor:
$$\frac{\delta}{x} = \frac{5.0}{\sqrt{{\rm Re}_x}},
\ f_x=\frac{0.664}{\sqrt{{\rm Re}_x}},
\ f_L=\frac{1.328}{\sqrt{{\rm Re}_L}},$$
In range 10$^5$ to 10$^7$, transition, oscillatory; beyond 10$^7$ fully
turbulent.  Always retains a laminar sublayer against the wall, though it
oscillates as vortices spiral down into it.  New behavior:
$$\frac{\delta}{x} = \frac{0.37}{{\rm Re}_x^{0.2}}$$
So $\delta\sim x^{0.8}$.  Grows much faster.  Why?  Mixing of momentum, higher
effective velocity.  But still a laminar sublayer near the side.

$f_x$ is oscillating all over the place, but what about the new $f_L$?
Disagreement, even for smooth plate:
$${\rm P\&G\ p.\ 38:}\ f_L=\frac{0.455}{(\log{\rm Re}_L)^{2.58}};
\ {\rm BSL\ p.\ 203:}\ f_L=\frac{0.146}{{\rm Re}_L^{0.2}}$$
Either way get some kind of curve in $f$-Re space which jumps in turbulence.
W$^3$R doesn't give an $f_L$, just an $f_x$ on p. 179, which is:
$$f_x=\frac{0.0576}{({\rm Re}_x)^{0.2}}.$$

\paragraph{Time-smoothing}

Time smoothing, or experiment-smoothing and Reynolds stresses: velocity varies
wildly, decompose into $u_x=\bar{u}_x + u'_x$, where $\bar{u}$ is
time-smoothed:
$$\bar{u}_x=\frac{\int_{t_a}^{t_b} u_x dt}{t_b-t_a}.$$
For time-dependent, experiment-average it.

\paragraph{Contest}

Prizes Wednesday for those who catch the two errors in the Navier-Stokes
equations t-shirt!

\paragraph{Not covered this year} The following topics were not covered in
lecture, but are here for your edification if you're interested.

\paragraph{Reynolds stresses}

Then take time-smoothed transport equations:
$$\bar{\frac{\partial(\bar{u_x}+u'_x)}{\partial t}} =
\bar{\frac{\partial\bar{u_x}}{\partial t}} +
\bar{\frac{\partial u'_x}{\partial t}} = \frac{\partial\bar{u_x}}{\partial t} +
0.$$
Same with spatial derivatives, pressure terms.  But one thing which doesn't
time-smooth out:
$$u'_x\frac{\partial u'_x}{\partial x}\neq 0.$$
This forms the Reynolds stresses, which we shift to the right side of the
equation:
$$\tau_{xy}=-\mu\left(\frac{\partial u_x}{\partial y} + \frac{\partial u_y}
  {\partial x}\right) - \bar{\rho u'_xu'_y}$$
Show how it's zero in the center of channel flow, large near the sides, zero
{\em at} the sides.  The resulting mass equation is the same; $x$-momentum
equation:
$$\rho\left(\frac{\partial\bar{u}}{\partial t} +
  \bar{\vec{u}}\cdot\nabla\bar{u}_x\right) =
-\frac{\partial\bar{P}}{\partial x} + \mu\nabla^2\bar{u}_x -
\rho\left(\frac{\partial}{\partial x}(\rho u'_xu'_x) +
  \frac{\partial}{\partial y}(\rho u'_xu'_y) +
  \frac{\partial}{\partial z}(\rho u'_xu'_z)\right) + F_x.$$

\paragraph{Turbulent transport and modeling}

Recall on test: pseudoplastic, Bingham.  Define effective viscosity: shear
stress/strain rate.

On a micro scale, lots of vortices/eddies.  On a macro scale, mixing leads to
higher effective $D_t$, $k_t$, and $\eta_t$ at length scales down to that
smallest eddy size, less so close to walls.  All three turbulent diffusivities
have the same magnitude.

New dimensionless number: Prandtl number is ratio of $\nu$ to
diffusivity, e.g. $\nu/D$ and $\nu/\alpha$.  Pr$_t\simeq1$ for heat and mass
transfer.

Next time: thermal and solutal boundary layers, heat and mass transfer
coefficients, turbulent boundary layer, then natural convection.  Last,
Bernoulli equation, continuous reactors.

Modeling: $K-\ell$ and $K-\epsilon$ modeling ($C_\mu$, $C_1$, $C_2$, $\sigma_K$
and $\sigma_\epsilon$ are empirical constants):
$$K=\frac{1}{2}\rho(u'^2_x+u'^2_y+u'^2_z), \nu_t=C_\mu\frac{K^2}{\epsilon}.$$
$$\frac{DK}{Dt} = \nabla\cdot\left(\frac{\nu_t}{\sigma_K}\nabla K\right) +
\nu_t\nabla\vec{u}\cdot\left(\nabla\vec{u}+(\nabla\vec{u})^T\right) -
\epsilon.$$
$$\frac{D\epsilon}{Dt} = 
\nabla\cdot\left(\frac{\nu_t}{\sigma_\epsilon}\nabla\epsilon\right)
+ C_1\frac{\nu_t\epsilon}{K}\nabla\vec{u}\cdot(\nabla\vec{u}+(\nabla\vec{u})^T)
- C_2\frac{\epsilon^2}{K}.$$


\chapter{Coupled Fluids with Heat and Mass Transfer}

\section{November 26, 2003: Coupled Fluids, Heat and Mass Transfer!}

Mechanics:
\begin{itemize}
\item Congrats to Jenny and David for winning the contest, prize: \$5 Tosci's.
\item PS8 on Stellar, due Fri 12/5.
\item Evaluations next Wednesday 12/3.
\end{itemize}
Muddy from last time:
\begin{itemize}
\item Time smoothing: what are $u_x$, $\bar{u}_x$, $u'_x$?  They are: the real
  velocity, the time-smoothed component of velocity, and the fluctuating
  component of velocity.
\item What timescale can you find from the lengthscale of the smallest eddies?
  How would one go about this?  Two timescales are relevant here: one is
  diffusion timescale $\ell^2/D$, which gives mixing time, and the timescale of
  formation and elimination of these little eddies $\ell^2/\nu$.  When
  attempting direct numerical simulation of turbulence, this tells how small a
  timestep one will need (actually, a fraction of this for accuracy); this also
  describes how long the eddies will last after the mixing power is turned off.
\end{itemize}

\paragraph{Thermal and solutal boundary layers}

Types: forced, natural convection; forced today, natural later.

Recall first BL thought experiment on thick polymer sheet extrusion, hot
polymer sheet $T_\infty$ and cold water $T_s$.  Now it's happening in a liquid,
competing thermal and fluid boundary layers with thicknesses $\delta_u$ and
$\delta_T$.

Fluid:
$$\delta_u = 5.0\sqrt{\frac{\nu x}{U_\infty}}$$
Thermal if flow uniform, same criterion:
$$\delta_T = 3.6\sqrt{\frac{\alpha x}{U_\infty}}$$
Dimensionless:
$$\frac{\delta_T}{x} = \frac{3.6}{\sqrt{\frac{U_\infty x}{\alpha}}} =
\frac{3.6}{\sqrt{{\rm Re}_x{\rm Pr}}}$$
When is flow uniform?  In a solid, or for much larger thermal boundary layer
than fluid, so $\alpha>>\nu$, Pr$<<1$.

Another way to look at it:
$$\frac{\delta_T}{\delta_u} = 0.72{\rm Pr}^{-1/2}$$

Large Prandtl number ($>$.5) means
$$\frac{\delta_T}{\delta_u} = 0.975{\rm Pr}^{-1/3}$$

Liquid metals (and about nothing else) have small Pr; mass transfer Pr is
almost always large.  {\em E.g.} water $\rm\nu = 10^{-6}\frac{m^2}{s} =
10^{-2}\frac{cm^2}{s}$, but $D$ is typically around $\rm10^{-5}\frac{cm^2}{s}$.

Note: blood platelets diffuse at around $D=10^{-9}$, but tumbling blood cells
not only stir and increase diffusivity, but somehow platelets end up on the
sides of blood vessels, where they're needed.  I don't fully understand...

\paragraph{Heat and Mass Transfer Coefficients}

What about $h$?  Start with $h_x$, then $h_L$, as before with $f_x$ and $f_L$.
Let $\beta_T=y\sqrt{U_\infty/\alpha x}$, $\theta=T-T_s/T_\infty-T_s$, graph $\theta$
vs. $\beta_T$ gives $\theta = {\rm erf}(\beta_T/2)$.

Heat conduction into the liquid:
$$q_y = -k\frac{\partial T}{\partial y} =
-k\frac{dT}{d\theta}\frac{d\theta}{d\beta_T}
\frac{\partial\beta_T}{\partial y}$$
$$q_y = k(T_s-T_\infty)\frac{1}{\sqrt{\pi}}\sqrt{\frac{U_\infty}{\alpha x}} =
h_x(T_s-T_\infty)$$
$$h_x = \frac{k}{\sqrt{\pi}}\sqrt{\frac{U_\infty}{\alpha x}} =
\sqrt{\frac{k\rho c_p U_\infty}{\pi x}}$$
Likewise for mass transfer, $\rho c_p$ is effectively one, so:
$$h_{Dx} = \sqrt{\frac{DU_\infty}{\pi x}}$$
Next time: average, dimensional analysis, $\delta_T<\delta_u$ case.
\newpage


\section{December 1, 2003: Nusselt Number, Heat and Mass Transfer Coefficients}

Mechanics:
\begin{itemize}
\item Evals Wednesday.
\end{itemize}
Muddy from last time:
\begin{itemize}
\item In the thermal boundary layer with constant velocity, why is $\partial^2
  T/\partial x^2<<\partial^2 T/\partial y^2$?  That's because $\delta\ll x$, so
  graph $T$ vs. x and vs. y, show $y$-deriv is larger.
\item Is there a physical meaning behind $\delta_T/\delta_u\propto
  {\rm Pr}^{-1/2}$ and $\delta_C/\delta_u\propto{\rm Pr}^{-1/3}$?  Yes, see
  below.
\end{itemize}

\paragraph{Heat and mass transfer coefficients}

Recap last time:
\begin{itemize}
\item Flow and heat/mass transfer: weakly coupled.  So far, all laminar.
\item Case 1: much larger thermal(/concentration) boundary layer (Pr$<$0.1):
  consider T/C BL to have uniform velocity, use same BL formulation as moving
  solid motivating example: erf solution, $\delta_T=3.6\sqrt{\alpha
    x/U_\infty}$.  Here:
  $$\delta_C/\delta_u\ {\rm or}\ \delta_T/\delta_u=0.72{\rm Pr}^{-1/2}.$$
  Physical meaning: grows as sqrt of diffusivity, so ratio is ratio of square
  roots of diffusivity, which is inverse sqrt(Pr).
\item Case 2: smaller thermal(/concentration) boundary layer (Pr$>$5 or so):
  consider T/C BL to have linear velocity, smaller velocity means thicker T/C
  BL.  Here:
  $$\delta_C/\delta_u\ {\rm or}\ \delta_T/\delta_u=0.975{\rm Pr}^{-1/3}.$$
\item Moving on, back to case 1, calculated $q|_{y=0}$ from erf solution:
  $$q_y = k(T_s-T_\infty)\frac{1}{\sqrt{\pi}}\sqrt{\frac{U_\infty}{\alpha x}} =
  h_x(T_s-T_\infty) \Rightarrow
  h_x = \sqrt{\frac{k\rho c_p U_\infty}{\pi x}}.$$
  Likewise for mass transfer:
  $$h_{Dx} = \sqrt{\frac{DU_\infty}{\pi x}}.$$
\end{itemize}
Since that's the local, let's integrate for average, neglecting edge effects:
$$q_{av} = \frac{1}{WL}\int_{x=0}^L h_x (T_s-T_\infty) Wdx =
h_L (T_s-T_\infty)$$
$$\frac{2(T_s-T_\infty)}{L}\left[\sqrt{k\rho c_p U_\infty x/\pi}\right]_{x=0}^L =
h_L (T_s-T_\infty)$$
$$h_L = 2\sqrt{\frac{k\rho c_p U_\infty}{\pi L}} =
\left.2h_x\right|_{x=L}$$

Now for case 2 (high-Prandtl), need different formulation.  Dimensional
analysis of mass transfer:
$$h_D = f(D_{fl}, U, x, \nu)$$
Five parameters, two base units (cm, s), so three dimensionless.  Eliminate $x$
and $D$.  Then one dimensionless is Reynolds ($\pi_U$), one is Prandtl
($\pi_\nu$), what's the third?
$$\pi_{h_D} = \frac{h_D x}{D_{fl}}$$
Looks like the Biot number, right?  But it's not, it's actually quite
different.
$${\rm Bi} = \frac{h_D L}{D_{solid}} = \frac{L/D_{solid}}{1/h} =
\frac{\rm Resistance\ to\ conduction\ in\ solid}{\rm Resistance\ due\ to\ BL
  \ in\ liquid}$$
Uses $L$=solid thickness, $D_{solid}$.  Heat transfer note: you get one extra
dimensionless number, due to heating by viscous friction.

Here, Nusselt \#, $L$=length of plate (in flow direction), the conduction and
BL are in the same medium, use $D_{liquid}$.
$${\rm Nu} = \frac{h_D L}{D_{liquid}} = \frac{L}{D_{liquid}/h_D} \simeq
\frac{L}{\delta_C}\ {\rm or}\ \frac{L}{\delta_T}.$$

Low-Prandtl fit:
$$\frac{h_L L}{k} = 2\sqrt{\frac{U_\infty L}{\pi\alpha}} =
\frac{2}{\sqrt{\pi}} {\rm Re}_L^{1/2}{\rm Pr}^{1/2}$$

Actually, for small to ``medium'' Pr, slight correction:
$${\rm Nu}_x =
\frac{0.564{\rm Re}_x^{1/2}{\rm Pr}^{1/2}}{1+0.90\sqrt{\rm Pr}}$$
$${\rm Nu}_L =
\frac{1.128{\rm Re}_x^{1/2}{\rm Pr}^{1/2}}{1+0.90\sqrt{\rm Pr}}$$

High: ($>$0.6): nice derivation in W$^3$R chapter 19:
$${\rm Nu}_x = 0.332{\rm Re}_x^{1/2} {\rm Pr}^{0.343}$$
$${\rm Nu}_L = 0.664{\rm Re}_L^{1/2} {\rm Pr}^{0.343}$$
Just as there are more correlations for $f$ (friction factor), lots more
correlations for various geometries etc. in handout by 2001 TA Adam Nolte.
Summarize: flow gives Re, props give Pr, gives Nu, gives $h$ (maybe Bi).
\newpage


\section{December 3: Natural Convection}

Mechanics:
\begin{itemize}
\item Course evals today!
\end{itemize}
Muddy from last time:
\begin{itemize}
\item What's the relationship between $h_x$ or $h_L$ and the friction factor?
  Hmm...  Meaning: heat transfer coefficient, kinetic energy transfer
  coefficient.  Types: local, global/average.  Laminar flow variation:
  both$\sim1/\sqrt{x}$, integral$\sim\sqrt{x}$, average$\sim1/\sqrt{x}$.
  Laminar $f_L=2f_x|_{x=L}$, $h_L=2h_x|_{x=L}$.  Dimensionless: $f=f$(Re),
  Nu=$f$(Re,Pr). Different correlations for different geometries.
\item Other Nusselt numbers from sheet by Adam Nolte.  (Note for Re=0 with a
  sphere...)
\end{itemize}

\paragraph{Natural convection}

Hot stuff rises, cold stuff sinks.  Obvious examples: radiators, etc.
Strongly-coupled equations:
$$\frac{D\rho}{Dt} + \nabla\cdot\vec{u} = 0$$
$$\rho\frac{D\vec{u}}{Dt} = -\nabla p + \eta\nabla^2\vec{u} + \rho\vec{g}$$
$$\frac{DT}{Dt} = \alpha\nabla^2 T + \frac{\dot{q}}{\rho c_p}$$
Full coupling comes in the $\rho$ in the fluid flow equations.

Volumetric thermal expansion coefficient:
$$\beta = -\frac{1}{\rho}\frac{d\rho}{dT},\ \rho - \rho_0 = \beta(T-T_0)$$
Note relation to 3.11 thermal expansion coeff:
$$\alpha = \frac{1}{L}\frac{dL}{dT}$$
$$\beta = -\frac{1}{\rho}\frac{d\rho}{dT} = -\frac{V}{M}\frac{d(M/V)}{dT}$$
$$d(1/V) = -dV/V^2,\ dV=d(L^3)=3L^2dL$$
$$\beta = V\frac{dV}{V^2dT} = \frac{3L^2 dL}{VdT} = \frac{3}{L}\frac{dL}{dT} =
3\alpha.$$
Word explanation: heat a solid cube, length increases 1\% in each direction,
volume increases 3\%.  Both have units $1/K$.  Ideal gases:
$$\rho=\frac{P}{RT},\ \beta=-\frac{1}{\rho}\frac{d\rho}{dT} =
-\frac{RT}{P}\left(-\frac{P}{RT^2}\right) = 1/T.$$
Also $\beta_C = -\frac{1}{\rho}\frac{d\rho}{dC}$.

Simplest case: vertical wall, $T_s$ at wall, $T_\infty$ with density
$\rho_\infty$ away from it, $x$ vertical and $y$ horizontal for consistency
with forced convection BL.  Assume:
\begin{enumerate}
\item \label{univisc} Uniform kinematic viscosity $\nu=\nu_\infty$.
\item \label{smalldense} Small density differences: $\rho$ only matters in
  $\rho g$ term, otherwise $\rho_\infty$ for convective terms.
\item \label{stdst} Steady-state.
\item \label{boussinesq} Boussinesq approx: $p\simeq-\rho_\infty gx+const$,
  obvious away from BL, no pressure difference across BL to drive flow.
\item \label{betadt} Also with small density diff: $\Delta\rho/\rho =
  \beta\Delta T$ ($\rho$ is roughly linear with $T$).
\item \label{noedge} No edge effects ($z$-direction).
\end{enumerate}
With assumptions \ref{univisc} and \ref{smalldense}, get momentum equation:
$$\frac{\partial\vec{u}}{\partial t}+\vec{u}\cdot\nabla\vec{u} =
\nu_\infty\nabla^2\vec{u} +
\frac{1}{\rho_\infty}\left(\rho \vec{g}-\nabla p\right).$$
Now for $x$-momentum, steady-state (assumption \ref{stdst}), assumption
\ref{boussinesq} gives:
$$\vec{u}\cdot\nabla u_x = \nu_\infty\nabla^2 u_x +
\frac{-\rho g+\rho_\infty g}{\rho_\infty}$$
Now assumptions \ref{betadt} and \ref{noedge}, $x$-momentum becomes:
$$u_x\frac{\partial u_x}{\partial x} + u_y\frac{\partial u_x}{\partial y} =
\nu_\infty\nabla^2 u_x + g\beta(T-T_\infty)$$
With $T_s>T_\infty$ and $g_x=-g$, this gives driving force in the positive-$x$
direction, which is up, like it's supposed to.  Okay, that's all for today,
more next time.
\newpage

\section{December 5: Wrapup Natural Convection}

Mechanics:
\begin{itemize}
  \item Test 2: before max=90, mean 75.38, std. dev 12.23; after max=100, mean
    95.76, std. dev 6.37.
\end{itemize}
Muddy from last time:
\begin{itemize}
\item D'oh!  Left too early...
\end{itemize}

Last time: assumptions led to equation:
$$u_x\frac{\partial u_x}{\partial x} + u_y\frac{\partial u_x}{\partial y} =
\nu_\infty\nabla^2 u_x + g\beta(T-T_\infty)$$
One more assumption, $\delta_u\ll x$, gives:
$$u_x\frac{\partial u_x}{\partial x} + u_y\frac{\partial u_x}{\partial y} =
\nu_\infty\frac{\partial^2u_x}{\partial y^2} + g\beta(T-T_\infty)$$

New dimensional analysis:
$$h = f(x, \nu, k, \rho c_p, g\beta, T_s-T_\infty)$$
Seven params 4 base units (kg, m, s, K); 3 dimless params.  Again Pr (dim'less
$\rho c_p$), Nu (dim'less $h$), this time Grashof number (dim'less $\beta$).
$${\rm Gr} = \frac{g\beta(T_s-T_\infty)L^3}{\nu^2}$$

Forced convection: ${\rm Nu} = f({\rm Re, Pr})$.

Natural convection: ${\rm Nu} = f({\rm Gr, Pr})$.

Detour: recall falling film
$$u_x = \frac{g\sin\theta(2Lz-z^2)}{2\nu}$$
$$u_av = \frac{g\sin\theta L^2}{3\nu}$$
$${\rm Re} = \frac{u_av \delta}{\nu} = \frac{g\cos\beta\delta^3}{3\nu^2}$$
So Gr is a natural convection Reynolds number, determines the rate of growth of
the BL.

Graphs of dimensionless $T=(T-T_\infty)/(T_s-T_\infty)$, dimensionless
$u_x={\rm Re}_x/2\sqrt{{\rm Gr}_x}$ vs. $y/\sqrt[4]{Gr_x}$ on P\&G p. 232
corresponding to dimensional graphs in W$^3$R p. 313.  Explain velocity BL is
always at least as thick as thermal BL, but thermal can be thinner for large
Pr.

Forced convection: $\delta\propto\sqrt{x}$

Natural convection: $\delta\propto\sqrt[4]{x}$

Note: in P\&G p. 232 plots, Pr=0.72 corresponds to air.

Another Gr interpretation: dimensionless temperature
gradient; for $\theta=\frac{T-T_\infty}{T_s-T_\infty}$:
$$\frac{\partial T}{\partial y} = \frac{\partial T}{\partial\theta}
\frac{\partial\theta}{\partial\frac{y}{x}\sqrt[4]{\frac{Gr_x}{4}}}
\frac{1}{x}\sqrt[4]{\frac{Gr_x}{4}} =
(T_s-T_\infty)f({\rm Pr})\frac{1}{x}\sqrt[4]{\frac{Gr_x}{4}}$$
Note velocity squared proportional to driving force in pipe flow, kinda same
here; heat trans proportional to square root of velocity.  Hence
${\rm Re}_x\propto\sqrt{{\rm Gr}_x}$ for velocity,
$\rm Nu_x\propto\sqrt{{\rm Re}_x}\propto\sqrt[4]{{\rm Gr}_x}$.

Transition to turbulence determined by Ra=GrPr, boundary at 10$^9$.  Laminar,
Ra between 10$^4$ and 10$^9$:
$$\frac{{\rm Nu}_L}{\sqrt[4]{{\rm Gr}_L/4}} =
\frac{\rm 0.902 Pr^{1/2}}{\rm(0.861 + Pr)^{1/4}}$$
Special for $\rm0.6<Pr<10$, laminar:
$${\rm Nu}_L = 0.56({\rm Gr}_L{\rm Pr})^{1/4}$$
Turbulence, Ra between 10$^9$ and 10$^{12}$ (p. 259):
$${\rm Nu}_L = \frac{0.0246 {\rm Gr}_L^{2/5}Pr^{7/15}}
{\rm(1+0.494 Pr^{2/3})^{2/5}}$$
Again, velocity$^{0.8}$ in a way, sorta like turbulent forced convection
boundary layers.
\newpage


\section{December 8: Wrapup Natural Convection, Streamfunction and Vorticity}

Mechanics:
\begin{itemize}
\item Final exam Monday 12/15 in 4-149.  Discuss operation, incl. closed/open
  sections, new diff eq, essay.
\end{itemize}
Muddy from last time:
\begin{itemize}
\item What were we supposed to get out of the last lecture?  Pretty much the
  list given: how natural conv BLs work, calculate $h_{(D)L}$ using Nu$_L$,
  $\delta_u\geq\delta_T$ or $\delta_C$, natural BLs grow more slowly, velocity
  and temperature profiles.
\item What direction is velocity?  Dominant velocity is in $x$-direction, which
  is vertical; upward for hot wall, downward for cold.  What's the difference
  between velocity in the BL, far from it?  Far from it, velocity is zero.
\item Why $\delta_u\geq\delta_T$?  Hot region lifts (or cold region sinks)
  fluid, so all of the hot/cold region (thermal BL) will be moving (in the
  velocity BL).  For large Pr, $\nu>\alpha$, so the momentum diffusion happens
  faster, thin thermal and thick velocity.
\item Dimensionless curves: crazy non-intuitive axis value
  $\frac{y}{x}\sqrt[4]{{\rm Gr}_x/4}$!  Well, not much worse than Blassius:
  $u_x/U_\infty$ vs. $\beta=y\sqrt{U_\infty/\nu x}=\frac{y}{x}\sqrt{{\rm
      Re}_x}$.  But I'll give you that the dimensionless velocity is a bit odd.
\item Where do these things come from?  Okay.  Concretize:
  $$\frac{\frac{u_x x}{\nu}}{2\sqrt{{\rm Gr}_x}} =
  \frac{1}{2}\frac{u_x x}{\nu}\sqrt{\frac{\nu^2}{g\beta\Delta T x^3}} =
  \frac{u_x}{\sqrt{g\beta\Delta Tx}}\Rightarrow
  u_{x,max}=f({\rm Pr})\sqrt{g\beta\Delta T x}.$$
  $$\frac{\delta_u}{x}\sqrt[4]{\frac{{\rm Gr}_x}{4}}=f({\rm Pr}) \Rightarrow
  \delta_u = f({\rm Pr})\frac{x}{\sqrt[4]{{\rm Gr}_x/4}} =
  \sqrt{2}f({\rm Pr})\sqrt[4]{\frac{x^4\nu^2}{g\beta\Delta Tx^3}} =
  \sqrt{2}f({\rm Pr})\sqrt[4]{\frac{x\nu^2}{g\beta\Delta T}}.$$
  These two results are consistent with: $u_{x,max}\propto {\rm thickness}^2$,
  forced convection $\Delta u_x/\Delta y$ goes as $1/\sqrt{{\rm Re}_x}$.
\end{itemize}
Other geometries: Raylegh-Bernard cells in inversion for GrPr greater than
1000.  Solutal buoyancy too, dissolving salt cube.
$$\beta_C = -\frac{1}{\rho}\frac{d\rho}{dC}.$$

Special: nucleate boiling, film boiling, $h$ vs. $T$ with liquid coolant.

If time: BL on rotating disk: $u\propto r$, so uniform BL.  Pretty cool.

Now can calculate (estimate) heat/mass transfer coefficients for forced and
natural convection, laminar or turbulent.  (D'oh!  Forgot this closing part
after the muddy stuff.)

\paragraph{Stream Function and Vorticity}

Vorticity introduced in turbulence video, measure of local rotation,
definition:
$$\omega = \nabla\times\vec{u}$$
2-D scalar, 3-D vector.  Some formulations give 2-D NS in terms of $u_x$,
$u_y$, $\omega$.  Also, vorticity particle methods: bundles of vorticity
moving, combining, annihilating.

Other application: crystal rotation in semisolid rheology.

Stream function, for incompressible flow where $\nabla\cdot\vec{u}=0$:
$$u_x = \frac{\partial\Psi}{\partial y},
\ u_y = -\frac{\partial\Psi}{\partial x}$$
Collapses velocity components into one parameter.  Look at $\Psi=Ax$,
$\Psi=By$, $\Psi=Ax+By$, $\Psi^2=x^2+y^2$.  Cool.

Gradient is normal to flow direction.  Streamlines: curves of constant $\Psi$,
parallel to flow direction.  If spaced apart same difference in $\Psi$, then
$$\left|\vec{u}\right| \propto {\rm distance\ between\ streamlines}$$

Aero-astros look out at wing and see streamline, Mech Es see structure, Mat
Scis see a giant fatigue specimen...

Visualizing 2-D flows, giving approximate regions of large and small velocity.
DON'T CROSS THE STREAMS!

Concept: flow separation, difference between jet and inlet.  Breathing through
nose.  (D'oh!  Forgot to mention breathing through the nose.)

\paragraph{Decisions...} Finish the term with the Bernoulli equation, or
continuous flow reactors?  Bernoulli wins the vote.
\newpage


\section{December 10, 2003: Bernoulli, Semester Wrapup}

TODO: get rooms for review sessions!

Mechanics:
\begin{itemize}
\item Review sessions: me Friday 2 PM, Albert Sunday evening.
\end{itemize}
No muddy cards from last time.

\paragraph{Bernoulli Equation}

W$^3$R chap 6: control volume integral derivation based on first law of
thermodynamics.  Interesting, I do somewhat different, based on Navier-Stokes;
I like to think mine is more straightforward, but you can read W$^3$R if
needed.

Also called ``inviscid flow''.  Motivation: tub with hole, pretty close to zero
friction factor, velocity is infinity?  No.  Something other than viscosity
limits it.

Navier-Stokes, throw out viscous terms:
$$\rho\frac{D\vec{u}}{Dt} = -\nabla p + \rho\vec{g}$$
Change coordinates to local streamline frame: $\hat{s}$ in direction of flow,
$\hat{n}$ in direction of curvature (perpendicular in 2-D, complicated in 3-D).

Flow only in $s$-direction, $s$-momentum equation for $\vec{g} = -g\hat{z}$:
$$\rho\left(\frac{\partial u_s}{\partial t}
  + u_s\frac{\partial u_s}{\partial s}\right)
= -\frac{\partial p}{\partial s} + \rho g_z\frac{\partial z}{\partial s}$$
Steady-state, constant $\rho$:
$$\frac{\partial \left(\frac{1}{2}\rho u_s^2\right)}{\partial s}
+ \frac{\partial p}{\partial s}- \rho g_z\frac{dz}{ds} = 0$$
Integrate along a streamline:
$$\frac{1}{2}\rho V^2 + p + \rho gz = constant$$
In other words:
$$KE + P + PE = constant$$
This is the Bernoulli equation.

Example 1: draining tub with a hole in the bottom.  Set $z=0$ at the bottom:
PE=$\rho gh$ at top, $P$ at bottom corner is that plus atmospheric pressure,
$\frac{1}{2}\rho V^2$ beyond outlet (further accelerating).  Potential energy
becomes pressure $\Delta P=\rho gh$, then becomes kinetic $V=\sqrt{2 gh}$.

Illustrate how changes with long tube $h_2$ down from bottom: $\rho gh$ at top,
$P_0$ at base in corner, $\frac{1}{2}\rho V^2 + P_1$ at base over spout,
$\frac{1}{2}\rho V^2 -\rho gh_2$ at tube end.  Three equations in three
unknowns.  Solves to $P_1=\rho gh$, $V^2=2g(h+h_2)$, $P_1 = -\rho gh_2$.  Can
also fill in the table...

Conditions:
\begin{itemize}
\item No shear or other losses (not nearly fully-developed)
\item No interaction with internal solids, etc.
\item No heat in or out, mechanical work on fluid (pumps, etc.)
\item No sudden expansion (jet$\longrightarrow$turbulent dissipation,
  separation complicates stuff)
\item No turbulence
\item No combustion (mixing$\longrightarrow$effective viscosity)
\item {\em Yes} sudden contraction.
\end{itemize}

Note time-to-drain problem on final of three years ago (that was the ``derive
and solve a new equation'' problem of 2000), tendency for diff eqs and thought
problems...

\paragraph{Semester summary}

You've come a very long way!  Mentioned linear to multiple nonlinear
PDEs, understanding of solution.  More generally, learned to start with a
simple conservation relation: accum = in - out + gen, turn into really powerful
results, on macro or micro scale, for diffusion, thermal energy, mass,
momentum, even kinetic energy.

Covered all topics in fluid dynamics and heat and mass transfer, in MechE,
ChemE, aero-astro.  If want to go on, take graduate advanced fluid dynamics or
heat/mass transfer, will be bored in undergrad class.

Also done some computation; for more depth with or without programming
experience, try 22.00J/3.021J!  (Shameless plug...)

Thank Albert for a terrific job as a TA!

\paragraph{Last muddy questions}

\begin{itemize}
\item What is the relevance of the boundary layer thickness to the Bernoulli
  equation?  The boundary layer is a region where there is quite a bit of
  shear, and sometimes turbulence.  If it is thin relative to the size of the
  problem (e.g. relative to the diameter of the tube), then most of the fluid
  will have negligible shear.
\item Why such a wierd coordinate system in Bernoulli example 2?  Why not just
  make $z=0$ at the bottom of the tube?  You could do that too, and it would
  work equally well, it just differs by a constant in the potential energy; the
  way we did it is just more consistent with the first example:
  \begin{center}
    \begin{tabular}{c|ccc|}
      Point & KE                & P                   & PE \\ \hline
      1     & $\sim$0           & $p_{atm}$           & $\rho g(h_1+h_2)$ \\
      2     & $\sim$0           & $p_{atm}+\rho gh_1$ & $\rho gh_2$ \\
      3     & $\rho g(h_1+h_2)$ & $p_{atm}-\rho gh_2$ & $\rho gh_2$ \\
      4     & $\rho g(h_1+h_2)$ & $p_{atm}$           & 0 \\ \hline
    \end{tabular}
  \end{center}
\end{itemize}

\paragraph{Batch and Continuous Flow Reactors}

For those interested.

Basic definitions, motivating examples.  Economics: batch better for
flexibility, continuous for quality and no setup time (always on).

Two types: volumetric and surface reactors.  Volume V, generation due to
chemical reaction; we'll discuss first-order $A\longrightarrow B$, so
$$G=-kC_A$$

For a volume batch reactor, start with $C_{A,in}$, dump into reactor, it goes:
$$\rm accum = generation$$
$$V\frac{dC_A}{dt} = -VkC_A$$
$$\ln(C_A) = -kt + A$$
$$\frac{C_{A,out}}{C_{A,in}} = \exp\left(-kt\right)$$

For mass transfer-limited surface batch reactor, say
$$\rm accum = out$$
$$V\frac{dC_{A,out}}{dt} = -Ah_dC_A$$
$$\frac{C_{A,out}}{C_{A,in}} = \exp\left(-\frac{h_DA}{V}t\right)$$

Two extremes in continuous reactor behavior with flow rate $Q$: plug flow and
perfect mixing.

Plug flow is like a mini-batch with $t_R=V/Q$, draw plug in a pipe, derive:
$$\frac{C_{A,out}}{C_{A,in}} = \exp\left(-\frac{kV}{Q}\right)$$
With a surface, the $V$s cancel, left with
$$\frac{C_{A,out}}{C_{A,in}} = \exp\left(-\frac{h_DA}{Q}\right)$$

Perfect mixing: in, out, gen, no accum, out at $C_{A,out}$ reactor conc:
$$0 = QC_{A,in} - QC_{A,out} - kVC_{A,out}$$
$$\frac{C_{A,out}}{C_{A,in}} = \frac{Q}{Q+kV} = \frac{1}{1+\frac{kV}{Q}}$$
With area:
$$\frac{C_{A,out}}{C_{A,in}} = \frac{1}{1+\frac{h_DA}{Q}}$$

Say target conversion is 0.01, given volume $V$, homogeneous with constant $k$.
\begin{itemize}
\item Batch:
  $$t_R = \frac{1}{k}\ln(C_{A,in}/C_{A,out}) = \frac{4.6}{k}$$
  prodection rate is
  $$\frac{V}{\frac{4.6}{k} + t_{change}} = \frac{kV}{4.6 + kt_{change}}$$
\item Plug:
  $$Q=\frac{kV}{\ln(C_{A,in}/C_{A,out})} = \frac{kV}{4.6}$$
  Better than batch, likely better quality too, less flexible.
\item Perfect mixing:
  $$Q = \frac{kV}{C_{A,in}/C_{A,out}-1} = \frac{kV}{99}$$
  Much smaller than either of the others!
\end{itemize}

Dead zones and effective volumes!

How to tell: tracers, Peclet number.

Other examples: catalytic combustion (that dimensional analysis problem in
PS3), alveoli/breathing (continuous/batch mixed).  Batch: generally better
conversion in same volume (see why); continuous: consistent quality, no setup
time.

Steelmaking: batch, but folk want to make continuous.
\end{document}
